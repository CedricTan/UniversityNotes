\documentclass[12pt, letterpaper]{article}
\usepackage[margin=1in]{geometry}
\usepackage{graphicx}
\usepackage{amsmath}
\usepackage{amssymb}
\usepackage{tikz}
\usepackage{hyperref}
\hypersetup{
	colorlinks = true,
	linkcolor = black,
	urlcolor = blue
}

\title{
	{GV101 Into to PolSci}\\
	{\large{Professor Simon Hix}}\\
	{\large{Lent Term Revision Document}}
}
\author{Cedric Tan}
\date{May 2019}
\begin{document}
\maketitle
\begin{abstract}

	This is a revision document for selected Lent Term Topics for the GV101 course. This is specific to the GV101 exam in May. The notes are mine fully and may not be authentic to the lecturer's as they have been modified.

	The format of this material is usually recounted lecture by lecture. Material may be merged together if it fits appropriately though this is unlikely in this course.

\end{abstract}
\newpage
\tableofcontents
\newpage

\section{How Government Works}
We will begin with a discussion on the workings of government which is an overarching theme on political institutions
\subsection{Consequences of Democratic Institutions}
There are two fundamental ways in which Democracy should work:
\begin{enumerate}
	\item Majoritarian
	\item Consensus
\end{enumerate}
The choices on these electoral rules has a huge impact on who gets to govern. The tension between majoritarian and consensus democracy is between \textbf{a guarantee of coherent stable choices (group transitivity)} and \textbf{a guarantee of freedom to form their own preferences (universal admissibility).} Below I will explain the two types of visions associated with the two main forms of democracy.
\subsubsection{The Majoritarian Vision}
Key features:
\begin{itemize}
	\item Elections are a choice between alternatives
	\item Elected party has responsibility over policy etc.
	\item Two models exist:
		\begin{itemize}
			\item Trustee model: politicians have autonomy
			\item Delegate mode: politicians have to stick to the constitution
		\end{itemize}
\end{itemize}
Further to that, other features for citizens include:
\begin{itemize}
	\item Ability to decide on performance and whether or not to reward or punish the party in power (clarity of accountability)
	\item However, you are only able to assert this influence every election
	\item Policy is only determined by the majority you vote in, there is no influence whatsoever from minority parties
	\item Voters need to vote in a clear majority for this system to be effective
\end{itemize}

\subsubsection{The Consensus Vision}
Key features:
\begin{itemize}
	\item Elections as an opportunity to choose a wide range of representatives
	\item Representatives are chosen by belief that they would be effective for particular issues or views
	\item Consensus is based on the trustee model of representation:
		\begin{itemize}
			\item Autonomy to bargain
			\item Constantly shifting majorities
			\item Continuously shift in accordance with citizen's preferences
		\end{itemize}
\end{itemize}
Further to that, other key features in the decision-making process include:
\begin{itemize}
	\item No privileged status in the decision-making process by any one party
	\item As many people as possible should be able to govern
\end{itemize}

\subsection{Institutions}
Below is a table of the institutions and how they differ between Majoritarian and Consensus governments.
\begin{center}
	\begin{tabular}{c|cc}
	Institution & Majoritarian & Consensus\\
	\hline
	Electoral System & Majoritarian & Proportional\\
	Party System & Two parties & Many parties\\
	Government Type & Single-party Majority & Coalition/Minority\\
	Federalism & Unitary & Federal\\
	Bicameralism & Unicameral & Bicameral\\
	Constitutionalism & legislative supremacy & higher law\\
	Regime Type & Parliamentary & Presidential\\
	\hline
	\end{tabular}
\end{center}
There is almost a dichotomy between representation of as many views as possible in a meaningful manner versus efficiency and action

\subsection{Political Representation}
Hannah Pitkin describes four different views of political representation:
\begin{enumerate}
	\item \textbf{Formalistic Representation:} how representatives are authorised and held accountable
	\item \textbf{Substantive Representation:} how representatives act for the people and promote interests
	\item \textbf{Descriptive Representation:} the extent to which representatives resemble their constituencies
	\item \textbf{Symbolic Representation:} focuses on the synbolic ways representatives stand up for
\end{enumerate}
The idea is that Descriptive and Symbolic forms of representation focus on \textbf{who} is being represented whilst Substantive representation focuses on \textbf{actions taken} by these representatives.

\subsubsection{Formalistic Representation}
Formalistic representation is about authorisation and accountability.\\\\
\textbf{Authority}
\begin{itemize}
	\item \textbf{Majoritarian:} majority authorises the distribution of power; policy-making decisions by the minority is considered illegitimate
	\item \textbf{Consensus:} dispersion of power as an important factor: direction proportion to electoral size, authority as distributed accordingly
\end{itemize}
\textbf{Accountability}
\begin{itemize}
	\item \textbf{Retrospective voting:} the ability of voters to sanction the ruling part based on their performance
	\item \textbf{Clarity of responsibility:} the ability to identify who the responsible people are for certain policies. This is required to have accountability. Higher concentrations of power lead to increased clarity whilst lower concentrations means more dispersion and a subsequent lack of clarity within the system
	\item \textbf{Accountability:} the extent to which we can attribute blame or praise for certain actions that were carried out.
		\begin{itemize}
			\item Majoritarian systems have \textbf{high levels} of accountability
			\item Consesus systems has \textbf{lower levels} of accountability
		\end{itemize}
	\item Institutions such as \textbf{bicameralism} or \textbf{federalism} also reduce clarity within the system due to a further dispersion of power
\end{itemize}

\subsubsection{Substantive Representation}
Substantive representation focuses on the actors taking actions in line with the ideological interests which they represent. The higher the substantive representation, the more in line the interests they represent and subsequent policy.

There are two key concepts to recognise:
\begin{enumerate}
	\item Ideological Congruence: the extent to which actions representatives do are in line with the interests of the people at a point in time (static capture of alignment and representation)
	\item Ideological Responsiveness: this is how quickly representatives change their behaviour to become more congruent with the interests of their people over time (dynamic and directional form of representation)
\end{enumerate}
\textbf{Congruence}\\
Judged by the ideological distance between the government and the \textbf{median} voter
\begin{itemize}
	\item \textbf{Majoritarian:} representatives tend to be congruent with the majority
	\item \textbf{Consensus:} representatives tend to be congruent with as many people as possible
\end{itemize}
\textbf{Responsiveness}\\
Conditions necessary for responsiveness include representatives \textbf{wanting to be more congruent} and also the representatives \textbf{having the ability to become more congruent.}
\begin{itemize}
	\item \textbf{Majoritarian:} higher responsiveness due to ability to enact change more easily to stay in power
	\item \textbf{Consensus:} lower responsiveness due to dispersion of authority and perhaps a strict alignment with party interests. There is also less clarity of responsibility and more veto players in a consensus system
\end{itemize}

\subsubsection{Descriptive Representation}
Descriptive representation is about whether or not representatives resemble who they represent. This could be on the category of \textbf{race, gender, religion or class.} Here are some key features:
\begin{itemize}
	\item Descriptive representation is valued more highly in \textbf{consensus based government} than \textbf{majoritarian based ones}
	\item Plausibly inferior to \textbf{substantive representation}
	\item This is focused more on \textbf{who people are} rather than \textbf{what they do}
	\item Cannot be held accountable by descriptive characteristics especially if they are \textbf{morally arbitrary}
	\item Critics argue that it can promote group essentialism, an exclusivity which is not conducive to cooperation
	\item However, descriptive representation can often lead to \textbf{substantive representation}
	\item Large district magnitudes lead to more descriptive representation
	\item It is a particularly pertinent issue with regards to women's representation
\end{itemize}

\subsubsection{Symbolic Representation}
Symbolic representation is about what representatives stand for. Key features include:
\begin{itemize}
	\item A dynamic, performative and constitutive process
	\item Involves a back and forth claims-making process between the representatives and the represented
	\item It is, however, \textbf{understudied} compared to other forms of representation
\end{itemize}

\subsection{Veto Players}
Veto players are already covered in the Short Answer Question document. Refer to the GitHub to have access to those. Key things to remember though are:
\begin{itemize}
	\item More veto players means less policy change
	\item Bigger policy distances between veto players means less policy change
\end{itemize}



\subsection{Types, Pros and Cons}
Below is a table on the regime types that can be related to the Majoritarian and Consensus visions:
\begin{center}
	Regime Type\\
	\begin{tabular}{c|c|c}
		& Parliamentary & Presidential\\
		\hline
		Single Party & Majoritarian & Consensus\\
			     & Westminster & USA\\
		\hline
		Coalition & Consensus & Super-Consensus\\
			  & Cont. Europe & Lat-Am Model\\
	\end{tabular}
\end{center}
Below is another table to illustrate some common-found pros and cons of Majoritarian versus Consensus Democracy systems:
\begin{center}
	\begin{tabular}{c|c|c}
		& Majority & Consensus\\
		\hline
		Pros & Decisive Government & Slower decisions\\
		     & Clear responsibility & Broad compromises\\
		     & Electoral Promises Kept & Protection of Minorities\\
		\hline
		Cons & Decisions too quick & Decisions too slow\\
		     & Elective dictatorship & No clear responsibility\\
		     & No compromises & Electoral promises broken\\
		     & Threat to minority interests & Vetoes by minorities
	\end{tabular}
\end{center}

\newpage
\section{Regimes: Presidents, Ministers and Parliaments}
How a democracy's regime is classified depends on the relationship between the government, the legislature and the president. The type of government formed depends on policy-seeking or office-seeking candidates.
Presidential democracies include more \textbf{more minority governments} but fewer \textbf{coalition governments than parliamentary ones.} Semi-presidential democracies, however, are understudied.
\subsection{Classification Questions}
Classification is based on how \textbf{people are kept in power.} Hence, there are two essential questions:
\begin{itemize}
	\item Is the government responsible to the elected legislature?
	\item Is the head of state popularly elected for a fixed term in office?
\end{itemize}

\subsection{Overview}
Below is an overview of Regime Types, their Political Implications and Pros and Cons:

\subsubsection{Functional Overview}
There exists two main powers of government.
\begin{itemize}
	\item Leadership (e.g. Foreign Policy, Budget and Legislative duties)
	\item Management (e.g. Implementation of this policy)
\end{itemize}
Further to that, there are elements of government as well:
\begin{center}
	\begin{tabular}{cc}
		Head & Prime-minister/President\\
		Cabinet Ministers & Finance, Education etc.\\
		Junior Ministers & Outside the cabinet\\
		Civil Servants & Deputies and agencies
	\end{tabular}
\end{center}
Further, a breakdown of the different models are given below:
\begin{itemize}
	\item \textbf{Presidential:} Separation of powers where the President appoints the cabinet and votes elect the legislature. The President and cabinet cannot dissolve congress and congress cannot remove the executive
	\item \textbf{Parliamentary:} Fusion of powers where both the PM and Cabinet can dissolve Parliament and the legislature can call a vote of no-confidence
	\item \textbf{Semi-Presidential:} President appoints the PM which appoints the cabinet. They can both be removed by the President or Parliament.
\end{itemize}

\subsubsection{Political Implications}
\textbf{Policy Making}
\begin{itemize}
	\item \textbf{Presidential:}\\
		Legislature controlled by congress (Agenda Setter)\\
		President has veto power (Veto Player)\\
		President commands majority and can set legislative agenda\\
		However, cannot force party cohesion\\
		Coalitions are built issue by issue and still require compromise\\
		In a divided government, parliament dominates OR gridlock exists
	\item \textbf{Parliamentary:}\\
		Government (Agenda Setter)\\
		Majority in government or Median Voter (Veto Player)\\
		The government has a monopoly over agenda setting\\
		Party cohesion is maintained through carrots (promotions) and sticks (no promotions, no-confidence votes)\\
		Except this carrot reward system doesn't work if candidates are:
		\begin{itemize}
			\item Rejected: non-promoted backbenches
			\item Ejected: former ministers
			\item Dejected: policy outliers
		\end{itemize}
		Variations in consensus versus majoritarian parliaments affect power\\
		\textbf{Majoritarian} has stronger government power\\
		\textbf{Proportional} has weaker government power
	\item \textbf{Semi-Presidential:}\\
		President via Government (Agenda Setter)\\
		Majority in parliament (Veto Player)\\
		Unified government, similar to presidential\\
		Co-habitation can also occur where the parliament governs
\end{itemize}

\subsubsection{Pros and Cons Overview}
Here is a table of some of the pros and cons associated with Presidential and Parliamentary systems:
\begin{center}
	\begin{tabular}{c|c|c}
		& Presidential & Parliamentary\\
		\hline
		Pros & Directly Accountable & One Election\\
		     & Working Parliament & Powerful Executive\\
		     & Checks and Balances & Cohesive Unit\\
		     & Deliberative Decision-making & Mandate to Govern\\
		\hline
		Cons & Grid Lock & Indirectly Accountable\\
		     & Weak Executive & Weak Parliament (Talk Shop)\\
		     & Weak Parties & Powerful Party Whips\\
		     & Regime Instability & Policy Change can be too quick
	\end{tabular}
\end{center}



\subsubsection{Government Responsibility}
Legislative responsibility means that a legislative majority has the constitutional power to remove the government from office without cause. This is done through a vote of no confidence, a constructive vote of no confidence which includes a suggested replacement and a vote of confidence which is initiated by governments confident that they will stay in power.
Presidential democracies are \textbf{defined by the absence of legislative responsibility. The legislature cannot remove the government without cause}

\subsubsection{Head of State}
A Head of State is popularly elected if they are elected through a process where voters either:
\begin{itemize}
	\item Cast ballots directly for the candidate
	\item Cast ballots to elect an assembly (electoral college) that elects a head of state
\end{itemize}

\subsubsection{Summary}
Below is a summary on the differences between regime types.
\begin{itemize}
	\item \textbf{Presidential:} does not depend on a legislative majority to exist
	\item \textbf{Parliamentary:} depends on legislative majority, \textbf{the Head of State is not popularly elected}
	\item \textbf{Semi-presidential:} depends on legislative majority, \textbf{the Head of State is popularly elected}
\end{itemize}

\subsection{Presidential Regimes: Making and Breaking}
This section will go over presidential democracies, how they form, what types there are and what types of compositions they can have.
\subsubsection{Formation Process}
\begin{itemize}
	\item Comprises of the president and the cabinet
	\item No requirement of a legislative majority to stay in office e.g. Republican President but Democratic Senate
	\item The president is always the \textbf{formateur which leads the formation of a coalition government}
\end{itemize}
Coalitions form in two main ways:
\begin{enumerate}
	\item Portfolio Coalition: legislators form a coalition related to the parties in the cabinet
	\item Legislature Coalition: bloc voting occurs for a piece of legislation
\end{enumerate}
\subsubsection{Presidential Cabinets}
Here are some features of cabinets leading to different types:
\begin{itemize}
	\item Can rule with minority cabinet but implicit legislative majority
	\item Coalition governments are thought to be exceptional cases in presidential governments dependent on policy/office seeking objectives
	\item Presidential decree: \textbf{order by the president that has the force of law}
	\item Weak decree power creates more incentives for coalitions
	\item Coalition governments may be more unstable in presidential democracies
	\item Coalition governments may survive longer but not be as effective
	\item Portfolio coalitions can outlive legislative coalitions
\end{itemize}

\subsubsection{Composition of Presidential Cabinets}
\begin{itemize}
	\item Less partisan ministers and lower cabinet proportionality
	\item Some look more parliamentary
\end{itemize}

\subsection{Institutions and Democratic Survival}
\subsubsection{Perils of Presidentialism}
Historical evidence points to less stability for democracy in presidential systems and has led to studies on the so-called \textbf{Perils of Presidentialism.} Perils are listed as such:
\begin{itemize}
	\item Difficult for citizens to identify who is responsible for policies as there is a \textbf{low clarity of responsibility} due to the separation of powers
	\item Presidentialism is thought to slow the policymaking process as policies must work their way through the legislature and be accepted by the president which means its tougher for a cabinet with minority control
	\item Produces a pattern of executive recruitment different from parliamentary systems which might result in nepotism
	\item Difficult to produce comprehensive policy due to the complex bargaining and lack of clarity
\end{itemize}
Further to that, Juan Linz provides 6 factors to consider. They are listed below with brief explanations:
\begin{enumerate}
	\item \textbf{Paradox of Presidentialism:} inability to have legitimacy and the suspicion of the personalisation of power
	\item \textbf{Zero-sum:} winner takes all mentality due to the strength of executive power
	\item \textbf{Style:} lack of a neat differentiation of roles within the government
	\item \textbf{Dual Legitimacy:} clarity of responsibility: legitimacy from legislative or electorate?
	\item \textbf{Stability Issues:} deadlock, legislative vs executive, minorities and majorities vetoing policy
	\item \textbf{Time Factor:} rushed politics with hasty implementation due to time constraint
\end{enumerate}



\subsubsection{Instability}
We begin with the concept of \textbf{Immobilism:} a situation in parliamentary democracies in which government coalitions are so weak and unstable that they are incapable of reaching an agreement on new policy. An example of this is France and their highly fragmented legislature leading to government immobilism.

Another question of stability, when comparing presidential to parliamentary, is whether or not one system is more stable than the other. Stephen and Skach argue that parliamentary systems are more stable:
\begin{itemize}
	\item Essence of Parliamentary Systems is mutual dependence
	\item Essence of Presidential Systems is mutual independence
	\item Mutual dependence encourages reconciliation
	\item Mutual independence encourages antagonism
\end{itemize}
Hence, democratic over-achievers are three times more likely to be parliamentary regimes as the drive to stay democratic is stronger due to this mutual dependence.

\subsubsection{Presidentialism and Multipartism}
The Perils of Presidentialism might only be a result of timing as we have adopted the regime system at a wrong time. Legislature fragmentation can occur as a result of:
\begin{itemize}
	\item Parliamentary cabinet instability
	\item Presidential democratic instability
\end{itemize}
Legislative and executive gridlock has no constitutional means of resolution. Possible methods are dissolving government or a vote of no confidence.
Further, inability to find legal ways out of deadlock causes instability. Institutional choice matters much more in poorer countries than richer ones as there are lower margins for error.
Increased veto players leads to less democratic stability as well.

\subsection{In Sum}
Three main ways in which democracies organise the relationship between the executive (government) and the legislature (parliament): presidential, parliamentary and mixed/semi-presidential.
\begin{itemize}
	\item \textbf{Presidential Systems} are characterised by separately elected leaders but powerful parliaments and weak political parties, and presidents are particularly weak if they do not command a majority in the parliament.
	\item \textbf{Parliamentary Systems} are characterised by powerful governments, weak parliaments and powerful parties
	\item \textbf{Semi-Presidential Systems} have powerful presidents if their government commands a parliamentary majority, but weak presidents if the majority in parliament (and the government) is from the opposing side
\end{itemize}
Further, the table below helps with some key features in Presidential and Parliamentary systems:
\begin{center}
	\begin{tabular}{c|c|c}
		& Presidential & Parliamentary\\
		\hline
		\textbf{How HoS} & Independent & Appointed by\\
		\textbf{is chosen} & Elections & Elections\\
		\textbf{How Government} & Approval by & Elections by\\
		\textbf{is chosen} & President & Seats\\
		\textbf{Removals} & No/No (Independence & Yes/Yes (No Conf.\\
			 & unless impeachment) & or Dissolve Parliament)\\
		\textbf{Fixed Term} & Yes (Fast Policy) & No (Slow Policy)\\
		\textbf{Agenda Setter} & Depends on Majority & Government\\
		\textbf{Veto players} & President and Congress & Median Voter in Legislature\\
		\textbf{Cohesion} & No Cohesion & Carrot and Stick system\\
		\textbf{Gridlock} & High propensity & Low propensity
	\end{tabular}
\end{center}

\newpage
\section{Cabinets, Coalitions and Single-Party Governments}



\end{document}

