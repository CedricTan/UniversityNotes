\documentclass[12pt, letterpaper]{article}
\usepackage[margin=1in]{geometry}
\usepackage{graphicx}
\usepackage{amsmath}
\usepackage{amssymb}
\usepackage{tikz}
\usepackage{hyperref}
\hypersetup{
	colorlinks = true,
	linkcolor = black,
	urlcolor = blue
}

\title{
	{GV101 Into to PolSci}\\
	{\large{Professor Simon Hix}}\\
	{\large{Lent Term Revision Document}}
}
\author{Cedric Tan}
\date{May 2019}
\begin{document}
\maketitle
\begin{abstract}

	This is a revision document for selected Lent Term Topics for the GV101 course. This is specific to the GV101 exam in May. The notes are mine fully and may not be authentic to the lecturer's as they have been modified.

	The format of this material is usually recounted lecture by lecture. Material may be merged together if it fits appropriately though this is unlikely in this course.

\end{abstract}
\newpage
\tableofcontents
\newpage

\section{How Government Works}
We will begin with a discussion on the workings of government which is an overarching theme on political institutions
\subsection{Consequences of Democratic Institutions}
There are two fundamental ways in which Democracy should work:
\begin{enumerate}
	\item Majoritarian
	\item Consensus
\end{enumerate}
The choices on these electoral rules has a huge impact on who gets to govern. The tension between majoritarian and consensus democracy is between \textbf{a guarantee of coherent stable choices (group transitivity)} and \textbf{a guarantee of freedom to form their own preferences (universal admissibility).} Below I will explain the two types of visions associated with the two main forms of democracy.
\subsubsection{The Majoritarian Vision}
Key features:
\begin{itemize}
	\item Elections are a choice between alternatives
	\item Elected party has responsibility over policy etc.
	\item Two models exist:
		\begin{itemize}
			\item Trustee model: politicians have autonomy
			\item Delegate mode: politicians have to stick to the constitution
		\end{itemize}
\end{itemize}
Further to that, other features for citizens include:
\begin{itemize}
	\item Ability to decide on performance and whether or not to reward or punish the party in power (clarity of accountability)
	\item However, you are only able to assert this influence every election
	\item Policy is only determined by the majority you vote in, there is no influence whatsoever from minority parties
	\item Voters need to vote in a clear majority for this system to be effective
\end{itemize}

\subsubsection{The Consensus Vision}
Key features:
\begin{itemize}
	\item Elections as an opportunity to choose a wide range of representatives
	\item Representatives are chosen by belief that they would be effective for particular issues or views
	\item Consensus is based on the trustee model of representation:
		\begin{itemize}
			\item Autonomy to bargain
			\item Constantly shifting majorities
			\item Continuously shift in accordance with citizen's preferences
		\end{itemize}
\end{itemize}
Further to that, other key features in the decision-making process include:
\begin{itemize}
	\item No privileged status in the decision-making process by any one party
	\item As many people as possible should be able to govern
\end{itemize}

\subsection{Institutions}
Below is a table of the institutions and how they differ between Majoritarian and Consensus governments.
\begin{center}
	\begin{tabular}{c|cc}
	Institution & Majoritarian & Consensus\\
	\hline
	Electoral System & Majoritarian & Proportional\\
	Party System & Two parties & Many parties\\
	Government Type & Single-party Majority & Coalition/Minority\\
	Federalism & Unitary & Federal\\
	Bicameralism & Unicameral & Bicameral\\
	Constitutionalism & legislative supremacy & higher law\\
	Regime Type & Parliamentary & Presidential\\
	\hline
	\end{tabular}
\end{center}
There is almost a dichotomy between representation of as many views as possible in a meaningful manner versus efficiency and action

\subsection{Political Representation}
Hannah Pitkin describes four different views of political representation:
\begin{enumerate}
	\item \textbf{Formalistic Representation:} how representatives are authorised and held accountable
	\item \textbf{Substantive Representation:} how representatives act for the people and promote interests
	\item \textbf{Descriptive Representation:} the extent to which representatives resemble their constituencies
	\item \textbf{Symbolic Representation:} focuses on the synbolic ways representatives stand up for
\end{enumerate}
The idea is that Descriptive and Symbolic forms of representation focus on \textbf{who} is being represented whilst Substantive representation focuses on \textbf{actions taken} by these representatives.

\subsubsection{Formalistic Representation}
Formalistic representation is about authorisation and accountability.\\\\
\textbf{Authority}
\begin{itemize}
	\item \textbf{Majoritarian:} majority authorises the distribution of power; policy-making decisions by the minority is considered illegitimate
	\item \textbf{Consensus:} dispersion of power as an important factor: direction proportion to electoral size, authority as distributed accordingly
\end{itemize}
\textbf{Accountability}
\begin{itemize}
	\item \textbf{Retrospective voting:} the ability of voters to sanction the ruling part based on their performance
	\item \textbf{Clarity of responsibility:} the ability to identify who the responsible people are for certain policies. This is required to have accountability. Higher concentrations of power lead to increased clarity whilst lower concentrations means more dispersion and a subsequent lack of clarity within the system
	\item \textbf{Accountability:} the extent to which we can attribute blame or praise for certain actions that were carried out.
		\begin{itemize}
			\item Majoritarian systems have \textbf{high levels} of accountability
			\item Consesus systems has \textbf{lower levels} of accountability
		\end{itemize}
	\item Institutions such as \textbf{bicameralism} or \textbf{federalism} also reduce clarity within the system due to a further dispersion of power
\end{itemize}

\subsubsection{Substantive Representation}
Substantive representation focuses on the actors taking actions in line with the ideological interests which they represent. The higher the substantive representation, the more in line the interests they represent and subsequent policy.

There are two key concepts to recognise:
\begin{enumerate}
	\item Ideological Congruence: the extent to which actions representatives do are in line with the interests of the people at a point in time (static capture of alignment and representation)
	\item Ideological Responsiveness: this is how quickly representatives change their behaviour to become more congruent with the interests of their people over time (dynamic and directional form of representation)
\end{enumerate}
\textbf{Congruence}\\
Judged by the ideological distance between the government and the \textbf{median} voter
\begin{itemize}
	\item \textbf{Majoritarian:} representatives tend to be congruent with the majority
	\item \textbf{Consensus:} representatives tend to be congruent with as many people as possible
\end{itemize}
\textbf{Responsiveness}\\
Conditions necessary for responsiveness include representatives \textbf{wanting to be more congruent} and also the representatives \textbf{having the ability to become more congruent.}
\begin{itemize}
	\item \textbf{Majoritarian:} higher responsiveness due to ability to enact change more easily to stay in power
	\item \textbf{Consensus:} lower responsiveness due to dispersion of authority and perhaps a strict alignment with party interests. There is also less clarity of responsibility and more veto players in a consensus system
\end{itemize}

\subsubsection{Descriptive Representation}
Descriptive representation is about whether or not representatives resemble who they represent. This could be on the category of \textbf{race, gender, religion or class.} Here are some key features:
\begin{itemize}
	\item Descriptive representation is valued more highly in \textbf{consensus based government} than \textbf{majoritarian based ones}
	\item Plausibly inferior to \textbf{substantive representation}
	\item This is focused more on \textbf{who people are} rather than \textbf{what they do}
	\item Cannot be held accountable by descriptive characteristics especially if they are \textbf{morally arbitrary}
	\item Critics argue that it can promote group essentialism, an exclusivity which is not conducive to cooperation
	\item However, descriptive representation can often lead to \textbf{substantive representation}
	\item Large district magnitudes lead to more descriptive representation
	\item It is a particularly pertinent issue with regards to women's representation
\end{itemize}

\subsubsection{Symbolic Representation}
Symbolic representation is about what representatives stand for. Key features include:
\begin{itemize}
	\item A dynamic, performative and constitutive process
	\item Involves a back and forth claims-making process between the representatives and the represented
	\item It is, however, \textbf{understudied} compared to other forms of representation
\end{itemize}

\subsection{Veto Players}


\end{document}

