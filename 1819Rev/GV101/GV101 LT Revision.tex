\documentclass[12pt, letterpaper]{article}
\usepackage[margin=1in]{geometry}
\usepackage{graphicx}
\usepackage{amsmath}
\usepackage{amssymb}
\usepackage{tikz}
\usepackage{hyperref}
\hypersetup{
	colorlinks = true,
	linkcolor = black,
	urlcolor = blue
}

\title{
	{GV101 Into to PolSci}\\
	{\large{Professor Simon Hix}}\\
	{\large{Lent Term Revision Document}}
}
\author{Cedric Tan}
\date{May 2019}
\begin{document}
\maketitle
\begin{abstract}

	This is a revision document for selected Lent Term Topics for the GV101 course. This is specific to the GV101 exam in May. The notes are mine fully and may not be authentic to the lecturer's as they have been modified.

	The format of this material is usually recounted lecture by lecture. Material may be merged together if it fits appropriately though this is unlikely in this course.

\end{abstract}
\newpage
\tableofcontents
\newpage

\section{How Government Works}
We will begin with a discussion on the workings of government which is an overarching theme on political institutions
\subsection{Consequences of Democratic Institutions}
There are two fundamental ways in which Democracy should work:
\begin{enumerate}
	\item Majoritarian
	\item Consensus
\end{enumerate}
The choices on these electoral rules has a huge impact on who gets to govern. The tension between majoritarian and consensus democracy is between \textbf{a guarantee of coherent stable choices (group transitivity)} and \textbf{a guarantee of freedom to form their own preferences (universal admissibility).} Below I will explain the two types of visions associated with the two main forms of democracy.
\subsubsection{The Majoritarian Vision}
Key features:
\begin{itemize}
	\item Elections are a choice between alternatives
	\item Elected party has responsibility over policy etc.
	\item Two models exist:
		\begin{itemize}
			\item Trustee model: politicians have autonomy
			\item Delegate mode: politicians have to stick to the constitution
		\end{itemize}
\end{itemize}
Further to that, other features for citizens include:
\begin{itemize}
	\item Ability to decide on performance and whether or not to reward or punish the party in power (clarity of accountability)
	\item However, you are only able to assert this influence every election
	\item Policy is only determined by the majority you vote in, there is no influence whatsoever from minority parties
	\item Voters need to vote in a clear majority for this system to be effective
\end{itemize}

\subsubsection{The Consensus Vision}
Key features:
\begin{itemize}
	\item Elections as an opportunity to choose a wide range of representatives
	\item Representatives are chosen by belief that they would be effective for particular issues or views
	\item Consensus is based on the trustee model of representation:
		\begin{itemize}
			\item Autonomy to bargain
			\item Constantly shifting majorities
			\item Continuously shift in accordance with citizen's preferences
		\end{itemize}
\end{itemize}
Further to that, other key features in the decision-making process include:
\begin{itemize}
	\item No privileged status in the decision-making process by any one party
	\item As many people as possible should be able to govern
\end{itemize}

\subsection{Institutions}
Below is a table of the institutions and how they differ between Majoritarian and Consensus governments.
\begin{center}
	\begin{tabular}{c|cc}
	Institution & Majoritarian & Consensus\\
	\hline
	Electoral System & Majoritarian & Proportional\\
	Party System & Two parties & Many parties\\
	Government Type & Single-party Majority & Coalition/Minority\\
	Federalism & Unitary & Federal\\
	Bicameralism & Unicameral & Bicameral\\
	Constitutionalism & legislative supremacy & higher law\\
	Regime Type & Parliamentary & Presidential\\
	\hline
	\end{tabular}
\end{center}
There is almost a dichotomy between representation of as many views as possible in a meaningful manner versus efficiency and action

\subsection{Political Representation}
Hannah Pitkin describes four different views of political representation:
\begin{enumerate}
	\item \textbf{Formalistic Representation:} how representatives are authorised and held accountable
	\item \textbf{Substantive Representation:} how representatives act for the people and promote interests
	\item \textbf{Descriptive Representation:} the extent to which representatives resemble their constituencies
	\item \textbf{Symbolic Representation:} focuses on the synbolic ways representatives stand up for
\end{enumerate}
The idea is that Descriptive and Symbolic forms of representation focus on \textbf{who} is being represented whilst Substantive representation focuses on \textbf{actions taken} by these representatives.

\subsubsection{Formalistic Representation}
Formalistic representation is about authorisation and accountability.\\\\
\textbf{Authority}
\begin{itemize}
	\item \textbf{Majoritarian:} majority authorises the distribution of power; policy-making decisions by the minority is considered illegitimate
	\item \textbf{Consensus:} dispersion of power as an important factor: direction proportion to electoral size, authority as distributed accordingly
\end{itemize}
\textbf{Accountability}
\begin{itemize}
	\item \textbf{Retrospective voting:} the ability of voters to sanction the ruling part based on their performance
	\item \textbf{Clarity of responsibility:} the ability to identify who the responsible people are for certain policies. This is required to have accountability. Higher concentrations of power lead to increased clarity whilst lower concentrations means more dispersion and a subsequent lack of clarity within the system
	\item \textbf{Accountability:} the extent to which we can attribute blame or praise for certain actions that were carried out.
		\begin{itemize}
			\item Majoritarian systems have \textbf{high levels} of accountability
			\item Consesus systems has \textbf{lower levels} of accountability
		\end{itemize}
	\item Institutions such as \textbf{bicameralism} or \textbf{federalism} also reduce clarity within the system due to a further dispersion of power
\end{itemize}

\subsubsection{Substantive Representation}
Substantive representation focuses on the actors taking actions in line with the ideological interests which they represent. The higher the substantive representation, the more in line the interests they represent and subsequent policy.

There are two key concepts to recognise:
\begin{enumerate}
	\item Ideological Congruence: the extent to which actions representatives do are in line with the interests of the people at a point in time (static capture of alignment and representation)
	\item Ideological Responsiveness: this is how quickly representatives change their behaviour to become more congruent with the interests of their people over time (dynamic and directional form of representation)
\end{enumerate}
\textbf{Congruence}\\
Judged by the ideological distance between the government and the \textbf{median} voter
\begin{itemize}
	\item \textbf{Majoritarian:} representatives tend to be congruent with the majority
	\item \textbf{Consensus:} representatives tend to be congruent with as many people as possible
\end{itemize}
\textbf{Responsiveness}\\
Conditions necessary for responsiveness include representatives \textbf{wanting to be more congruent} and also the representatives \textbf{having the ability to become more congruent.}
\begin{itemize}
	\item \textbf{Majoritarian:} higher responsiveness due to ability to enact change more easily to stay in power
	\item \textbf{Consensus:} lower responsiveness due to dispersion of authority and perhaps a strict alignment with party interests. There is also less clarity of responsibility and more veto players in a consensus system
\end{itemize}

\subsubsection{Descriptive Representation}
Descriptive representation is about whether or not representatives resemble who they represent. This could be on the category of \textbf{race, gender, religion or class.} Here are some key features:
\begin{itemize}
	\item Descriptive representation is valued more highly in \textbf{consensus based government} than \textbf{majoritarian based ones}
	\item Plausibly inferior to \textbf{substantive representation}
	\item This is focused more on \textbf{who people are} rather than \textbf{what they do}
	\item Cannot be held accountable by descriptive characteristics especially if they are \textbf{morally arbitrary}
	\item Critics argue that it can promote group essentialism, an exclusivity which is not conducive to cooperation
	\item However, descriptive representation can often lead to \textbf{substantive representation}
	\item Large district magnitudes lead to more descriptive representation
	\item It is a particularly pertinent issue with regards to women's representation
\end{itemize}

\subsubsection{Symbolic Representation}
Symbolic representation is about what representatives stand for. Key features include:
\begin{itemize}
	\item A dynamic, performative and constitutive process
	\item Involves a back and forth claims-making process between the representatives and the represented
	\item It is, however, \textbf{understudied} compared to other forms of representation
\end{itemize}

\subsection{Veto Players}
Veto players are already covered in the Short Answer Question document. Refer to the GitHub to have access to those. Key things to remember though are:
\begin{itemize}
	\item More veto players means less policy change
	\item Bigger policy distances between veto players means less policy change
\end{itemize}



\subsection{Types, Pros and Cons}
Below is a table on the regime types that can be related to the Majoritarian and Consensus visions:
\begin{center}
	Regime Type\\
	\begin{tabular}{c|c|c}
		& Parliamentary & Presidential\\
		\hline
		Single Party & Majoritarian & Consensus\\
			     & Westminster & USA\\
		\hline
		Coalition & Consensus & Super-Consensus\\
			  & Cont. Europe & Lat-Am Model\\
	\end{tabular}
\end{center}
Below is another table to illustrate some common-found pros and cons of Majoritarian versus Consensus Democracy systems:
\begin{center}
	\begin{tabular}{c|c|c}
		& Majority & Consensus\\
		\hline
		Pros & Decisive Government & Slower decisions\\
		     & Clear responsibility & Broad compromises\\
		     & Electoral Promises Kept & Protection of Minorities\\
		\hline
		Cons & Decisions too quick & Decisions too slow\\
		     & Elective dictatorship & No clear responsibility\\
		     & No compromises & Electoral promises broken\\
		     & Threat to minority interests & Vetoes by minorities
	\end{tabular}
\end{center}


\newpage
\section{Regimes: Presidents, Ministers and Parliaments}
How a democracy's regime is classified depends on the relationship between the government, the legislature and the president. The type of government formed depends on policy-seeking or office-seeking candidates.
Presidential democracies include more \textbf{more minority governments} but fewer \textbf{coalition governments than parliamentary ones.} Semi-presidential democracies, however, are understudied.
\subsection{Classification Questions}
Classification is based on how \textbf{people are kept in power.} Hence, there are two essential questions:
\begin{itemize}
	\item Is the government responsible to the elected legislature?
	\item Is the head of state popularly elected for a fixed term in office?
\end{itemize}

\subsection{Overview}
Below is an overview of Regime Types, their Political Implications and Pros and Cons:

\subsubsection{Functional Overview}
There exists two main powers of government.
\begin{itemize}
	\item Leadership (e.g. Foreign Policy, Budget and Legislative duties)
	\item Management (e.g. Implementation of this policy)
\end{itemize}
Further to that, there are elements of government as well:
\begin{center}
	\begin{tabular}{cc}
		Head & Prime-minister/President\\
		Cabinet Ministers & Finance, Education etc.\\
		Junior Ministers & Outside the cabinet\\
		Civil Servants & Deputies and agencies
	\end{tabular}
\end{center}
Further, a breakdown of the different models are given below:
\begin{itemize}
	\item \textbf{Presidential:} Separation of powers where the President appoints the cabinet and votes elect the legislature. The President and cabinet cannot dissolve congress and congress cannot remove the executive
	\item \textbf{Parliamentary:} Fusion of powers where both the PM and Cabinet can dissolve Parliament and the legislature can call a vote of no-confidence
	\item \textbf{Semi-Presidential:} President appoints the PM which appoints the cabinet. They can both be removed by the President or Parliament.
\end{itemize}

\subsubsection{Political Implications}
\textbf{Policy Making}
\begin{itemize}
	\item \textbf{Presidential:}\\
		Legislature controlled by congress (Agenda Setter)\\
		President has veto power (Veto Player)\\
		President commands majority and can set legislative agenda\\
		However, cannot force party cohesion\\
		Coalitions are built issue by issue and still require compromise\\
		In a divided government, parliament dominates OR gridlock exists
	\item \textbf{Parliamentary:}\\
		Government (Agenda Setter)\\
		Majority in government or Median Voter (Veto Player)\\
		The government has a monopoly over agenda setting\\
		Party cohesion is maintained through carrots (promotions) and sticks (no promotions, no-confidence votes)\\
		Except this carrot reward system doesn't work if candidates are:
		\begin{itemize}
			\item Rejected: non-promoted backbenches
			\item Ejected: former ministers
			\item Dejected: policy outliers
		\end{itemize}
		Variations in consensus versus majoritarian parliaments affect power\\
		\textbf{Majoritarian} has stronger government power\\
		\textbf{Proportional} has weaker government power
	\item \textbf{Semi-Presidential:}\\
		President via Government (Agenda Setter)\\
		Majority in parliament (Veto Player)\\
		Unified government, similar to presidential\\
		Co-habitation can also occur where the parliament governs
\end{itemize}

\subsubsection{Pros and Cons Overview}
Here is a table of some of the pros and cons associated with Presidential and Parliamentary systems:
\begin{center}
	\begin{tabular}{c|c|c}
		& Presidential & Parliamentary\\
		\hline
		Pros & Directly Accountable & One Election\\
		     & Working Parliament & Powerful Executive\\
		     & Checks and Balances & Cohesive Unit\\
		     & Deliberative Decision-making & Mandate to Govern\\
		\hline
		Cons & Grid Lock & Indirectly Accountable\\
		     & Weak Executive & Weak Parliament (Talk Shop)\\
		     & Weak Parties & Powerful Party Whips\\
		     & Regime Instability & Policy Change can be too quick
	\end{tabular}
\end{center}



\subsubsection{Government Responsibility}
Legislative responsibility means that a legislative majority has the constitutional power to remove the government from office without cause. This is done through a vote of no confidence, a constructive vote of no confidence which includes a suggested replacement and a vote of confidence which is initiated by governments confident that they will stay in power.
Presidential democracies are \textbf{defined by the absence of legislative responsibility. The legislature cannot remove the government without cause}

\subsubsection{Head of State}
A Head of State is popularly elected if they are elected through a process where voters either:
\begin{itemize}
	\item Cast ballots directly for the candidate
	\item Cast ballots to elect an assembly (electoral college) that elects a head of state
\end{itemize}

\subsubsection{Summary}
Below is a summary on the differences between regime types.
\begin{itemize}
	\item \textbf{Presidential:} does not depend on a legislative majority to exist
	\item \textbf{Parliamentary:} depends on legislative majority, \textbf{the Head of State is not popularly elected}
	\item \textbf{Semi-presidential:} depends on legislative majority, \textbf{the Head of State is popularly elected}
\end{itemize}

\subsection{Presidential Regimes: Making and Breaking}
This section will go over presidential democracies, how they form, what types there are and what types of compositions they can have.
\subsubsection{Formation Process}
\begin{itemize}
	\item Comprises of the president and the cabinet
	\item No requirement of a legislative majority to stay in office e.g. Republican President but Democratic Senate
	\item The president is always the \textbf{formateur which leads the formation of a coalition government}
\end{itemize}
Coalitions form in two main ways:
\begin{enumerate}
	\item Portfolio Coalition: legislators form a coalition related to the parties in the cabinet
	\item Legislature Coalition: bloc voting occurs for a piece of legislation
\end{enumerate}
\subsubsection{Presidential Cabinets}
Here are some features of cabinets leading to different types:
\begin{itemize}
	\item Can rule with minority cabinet but implicit legislative majority
	\item Coalition governments are thought to be exceptional cases in presidential governments dependent on policy/office seeking objectives
	\item Presidential decree: \textbf{order by the president that has the force of law}
	\item Weak decree power creates more incentives for coalitions
	\item Coalition governments may be more unstable in presidential democracies
	\item Coalition governments may survive longer but not be as effective
	\item Portfolio coalitions can outlive legislative coalitions
\end{itemize}

\subsubsection{Composition of Presidential Cabinets}
\begin{itemize}
	\item Less partisan ministers and lower cabinet proportionality
	\item Some look more parliamentary
\end{itemize}

\subsection{Institutions and Democratic Survival}
\subsubsection{Perils of Presidentialism}
Historical evidence points to less stability for democracy in presidential systems and has led to studies on the so-called \textbf{Perils of Presidentialism.} Perils are listed as such:
\begin{itemize}
	\item Difficult for citizens to identify who is responsible for policies as there is a \textbf{low clarity of responsibility} due to the separation of powers
	\item Presidentialism is thought to slow the policymaking process as policies must work their way through the legislature and be accepted by the president which means its tougher for a cabinet with minority control
	\item Produces a pattern of executive recruitment different from parliamentary systems which might result in nepotism
	\item Difficult to produce comprehensive policy due to the complex bargaining and lack of clarity
\end{itemize}
Further to that, Juan Linz provides 6 factors to consider. They are listed below with brief explanations:
\begin{enumerate}
	\item \textbf{Paradox of Presidentialism:} inability to have legitimacy and the suspicion of the personalisation of power
	\item \textbf{Zero-sum:} winner takes all mentality due to the strength of executive power
	\item \textbf{Style:} lack of a neat differentiation of roles within the government
	\item \textbf{Dual Legitimacy:} clarity of responsibility: legitimacy from legislative or electorate?
	\item \textbf{Stability Issues:} deadlock, legislative vs executive, minorities and majorities vetoing policy
	\item \textbf{Time Factor:} rushed politics with hasty implementation due to time constraint
\end{enumerate}



\subsubsection{Instability}
We begin with the concept of \textbf{Immobilism:} a situation in parliamentary democracies in which government coalitions are so weak and unstable that they are incapable of reaching an agreement on new policy. An example of this is France and their highly fragmented legislature leading to government immobilism.

Another question of stability, when comparing presidential to parliamentary, is whether or not one system is more stable than the other. Stephen and Skach argue that parliamentary systems are more stable:
\begin{itemize}
	\item Essence of Parliamentary Systems is mutual dependence
	\item Essence of Presidential Systems is mutual independence
	\item Mutual dependence encourages reconciliation
	\item Mutual independence encourages antagonism
\end{itemize}
Hence, democratic over-achievers are three times more likely to be parliamentary regimes as the drive to stay democratic is stronger due to this mutual dependence.

\subsubsection{Presidentialism and Multipartism}
The Perils of Presidentialism might only be a result of timing as we have adopted the regime system at a wrong time. Legislature fragmentation can occur as a result of:
\begin{itemize}
	\item Parliamentary cabinet instability
	\item Presidential democratic instability
\end{itemize}
Legislative and executive gridlock has no constitutional means of resolution. Possible methods are dissolving government or a vote of no confidence.
Further, inability to find legal ways out of deadlock causes instability. Institutional choice matters much more in poorer countries than richer ones as there are lower margins for error.
Increased veto players leads to less democratic stability as well.

\subsection{In Sum}
Three main ways in which democracies organise the relationship between the executive (government) and the legislature (parliament): presidential, parliamentary and mixed/semi-presidential.
\begin{itemize}
	\item \textbf{Presidential Systems} are characterised by separately elected leaders but powerful parliaments and weak political parties, and presidents are particularly weak if they do not command a majority in the parliament.
	\item \textbf{Parliamentary Systems} are characterised by powerful governments, weak parliaments and powerful parties
	\item \textbf{Semi-Presidential Systems} have powerful presidents if their government commands a parliamentary majority, but weak presidents if the majority in parliament (and the government) is from the opposing side
\end{itemize}
Further, the table below helps with some key features in Presidential and Parliamentary systems:
\begin{center}
	\begin{tabular}{c|c|c}
		& Presidential & Parliamentary\\
		\hline
		\textbf{How HoS} & Independent & Appointed by\\
		\textbf{is chosen} & Elections & Elections\\
		\textbf{How Government} & Approval by & Elections by\\
		\textbf{is chosen} & President & Seats\\
		\textbf{Removals} & No/No (Independence & Yes/Yes (No Conf.\\
			 & unless impeachment) & or Dissolve Parliament)\\
		\textbf{Fixed Term} & Yes (Fast Policy) & No (Slow Policy)\\
		\textbf{Agenda Setter} & Depends on Majority & Government\\
		\textbf{Veto players} & President and Congress & Median Voter in Legislature\\
		\textbf{Cohesion} & No Cohesion & Carrot and Stick system\\
		\textbf{Gridlock} & High propensity & Low propensity
	\end{tabular}
\end{center}

\newpage
\section{Cabinets, Coalitions and Single-Party Governments}
We begin this section with an overview of the types of government that exist:
\begin{itemize}
	\item \textbf{Majority Government:} majority of the seats which is equal to 50\% + 1\%
	\item \textbf{Minority Government:} does not have a majority of 50\% + 1\% seats
	\item \textbf{Single Party:} all controlling seats are controlled by 1 party
	\item \textbf{Coalition:} seats are from multiple parties
		\begin{itemize}
			\item Minimum Winning Coalitions are when you take 1 party away and there is no more legislative majority
			\item Surplus Majority Coalitions are composed with more than the number required to maintain a majority
		\end{itemize}
\end{itemize}
An example government is the current UK government in 2019 which is a \textbf{Single Party Minority}

\subsection{Coalition Formation Processes}
The Coalition Formation Process starts with the \textbf{Formateur} which creates the cabinet if no majority is present.

According to \textbf{Gamson's Law,} the cabinet portfolio is distributed according to proportion of contributing seats per party to legislative majority. Hence, if your party contributes more seats to the majority, say 70\% of the seats, then your party should control 70\% of the cabinet as well.

Further, Coalitions are formed on two bases:
\begin{itemize}
	\item \textbf{Office Seeking:}\\
		\textit{Assumption:} Parties try to maximise cabinet seats\\
		\textit{Proposition:} Only minimum winning coalitions should form
	\item \textbf{Policy Seeking:}\\
		\textit{Assumption:} Parties try to seek policy outcomes\\
		\textit{Proposition:} coalitions should only form between connected parties
\end{itemize}
These two, however, are not necessarily \textbf{mutually exclusive} since if we get the two together, we would get a \textbf{minimum connected coalition.}

\subsubsection{Minority Governments}
Minority governments can be stable, if the party or parties in government are in the centre in which case a majority coalition will not be preferred by the other parties in the legislature to the minority government. This is because the radical wings on the Left Right Spectrum, adhering to Axelrod's policy seeking theory, would not want to form a majority coalition with the opposing parties on the other side of the wing.

\subsection{Consequences of Coalition and Single Party}
Below are a list of questions related to performance that might be asked of these types of governments:
\begin{itemize}
	\item \textbf{Duration:} are single-party governments more stable than coalition governments?
	\item \textbf{Policy-making:} is policy-making easier/faster in single-party governments than in coalition governments?
	\item \textbf{Accountability and Clarity of Responsibility:} are single-party governments more accountable than coalition governments?
	\item \textbf{Representation:} are coalition governments more representative than single-party governments?
\end{itemize}

\subsubsection{Duration}
\begin{itemize}
	\item Single-party majority lasted around 3 years
	\item Minimal winning coalition lasted also around 3 years
\end{itemize}
Do not generalise from bad coalitions that all coalitions will not last long. It depends on the type of coalition you form.

\subsubsection{Policy Making}
This depends on the type of cabinet you have, refer to the SAQ document for more information:
\begin{itemize}
	\item \textbf{Single-party government:} if there is party cohesion, the leader of the majority party would be the dictator as they are the agenda setter and control a majority to pass legislation
	\item \textbf{Coalition Government:} members of the coalition are veto players, there can be compromise but also possible gridlock
\end{itemize}

\subsubsection{Accountability}
Also known as clarity of responsibility, this is how clear it is for voters to know which party is responsible for government policies i.e. who to reward or blame
\begin{itemize}
	\item \textbf{Single-party government:} responsibility for policy is clear
	\item \textbf{Coalition government:} responsibility for policy is less clear and the more parties there are, the less responsibility individual parties have, popular policies credit can be claimed but then unpopular shifted away
\end{itemize}
A good example of this is Hellwig and Samuels (2008) who studied the effect of the clarity of responsibility. The regression output is shown below:
\begin{center}
	\noindent\fbox{\begin{minipage}{.85\textwidth}
			\textbf{Dependent Variables:}
			\begin{itemize}
				\item \% Vote for PM/Pres Party
			\end{itemize}
			\textbf{Independent Variables}
			\begin{itemize}	
				\item Previous Vote: \% vote in previous
				\item Economy: GDP growth in year before
				\item Re-election: Dichotomous
				\item Age of Democracy: Number of years
			\end{itemize}
			\begin{center}
				\begin{tabular}{ccc}
					\hline
					\hline
					Independent Var & High Clarity & Low Clarity\\
					\hline
					Previous Vote & 0.45** & 0.71**\\
						      & (0.18) & (0.06)\\
					Economy & 0.55* & 0.28\\
						& (0.38) & (0.28)\\
					Re-election & 7.96* & 6.31*\\
						    &(4.66) & (2.08)\\
					Age of Democracy & 0.44** & 0.02\\
							 & (0.18) & (0.09)\\
					Age of Democracy$^2$ & -0.005** & 0.00\\
							     &(0.002) & (0.001)\\
					Constant & 13.60 & 5.19*\\
						 &(9.13) & (2.88)\\
					R$^2$ & 0.29 & 0.52\\
					F-Statistic & 4.90** & 46.54**\\
					N & 108 & 318\\
					\hline
					\hline
				\end{tabular}
			\end{center}
			\begin{itemize}
				\item High Clarity (Single-Party Majority in Parliamentary) $\rightarrow$ positive relationship because we can reward the government for performing well economically
				\item Low Clarity (Coalitions, Presidentials etc) $\rightarrow$ no relationship whatsoever as we cannot attribute blame or praise due to lack of clarity on who has done what
			\end{itemize}
			\end{minipage}}
		\end{center}
Thus, Hellwig and Samuels do show that clarity of responsibility matters a lot in rewarding of punishing parties.

\subsubsection{Representation}
One way of thinking about representation: the close a government is to the median voter, the more representative it is. \textbf{A representative government} is a government which contains the party the median voter supported in the election.

Huber and Powell (1994) first introduced \textbf{Citizen-Government} distance. They found that the average distance of a coalition government to the median voter is closer than majority parties.

\subsection{In Sum}
Below are the key takeaways from this week:
\begin{itemize}
	\item Democracies can have single-party or coalition governments and majority or minority governments.
	\item Office Seeking (Riker 1962) predict that minimum-winning coalitions will form and that cabinet seats will be allocated in proportion to parties' seat-shares
	\item Policy Seeking (Axelrod 1970) predict that connected coalitions will form between parties next to each other on a policy dimension and that parties will bargain about the content of a coalition agreement
	\item Single-party governments \textbf{tend to be more stable, more decisive and more accountable}
	\item Coalition governments \textbf{tend to be more consensual and more representative without much loss in decision-making}
\end{itemize}

\newpage
\section{Federalism and Decentralisation}
Balancing Democracy and Diversity is a difficult task. Below are a collection of quotes on this topic:
\begin{itemize}
	\item Small nations have always been the cradle of liberty; and the fact that many of them have lost their liberty by becoming larger shows that their freedom was more a consequence of their small size than of the character of their people ... The federal system was created with the intention of combining the different advantages which result from the magnitude and the littleness of nations.\\
		\textbf{- Alexis de Tocqueville 1835-40}
	\item Federalism is the main alternative to empire as a technique for aggregating large areas under one government ... The essential institutions of federalism are ... a government of the federation and a set of governments of the member units, in which both kinds of governments rule over the same territory and people and each kind has the authority to make some decisions independently of the other.\\
		\textbf{- William H. Riker 1987}
	\item For how long will English constituencies and English Honourable members tolerate ... at least 119 Honourable members from Scotland, Wales and Northern Ireland exercising an important, and probably often decisive, effect on English politics while they themselves have no say in the same matters in Scotland, Wales and Northern Ireland?\\
		\textbf{- Tam Dalyell Labour MP for West Lothian 14th Nov. 1977}
\end{itemize}

\subsection{Models of Territorial Organisation of the State}
Below are examples of how states can be organised.
\subsubsection{Federalism Overview}
A definition of Federalism has 4 key components (Elazar 1997; Bednar 2009)
\begin{enumerate}
	\item \textbf{Geopolitical division:} the country is divided into regional governments that are constitutionally recognised and that cannot be unilaterally abolished by the central government
	\item \textbf{Independence:} regional and central governments have independent bases of authority e.g. separate elections, courts, laws etc.
	\item \textbf{Direct Governance:} policy-making is divided between the regional and central governments, such that each has some 'exclusive competences' over their citizens, e.g. as set out in a 'catalogue of competences'
	\item \textbf{Territorial Representation:} regional sub-units are represented in the upper chamber of the central legislature, and so have power over central government policy
\end{enumerate}
\subsubsection{Unitary State Overview}
A unitary state has features as follows:
\begin{enumerate}
	\item Geopolitical divisions decided by central government
	\item May have independent elections, but not separate courts or laws
	\item No direct governance
	\item No territorial representation in central legislature
\end{enumerate}

\subsubsection{Devolution/Decentralisation within a Unitary State}
Below are features of the half-way solutions:
\begin{enumerate}
	\item Existence and powers of geopolitical divisions are decided by central government
	\item Some sub-units have independent courts and legal traditions
	\item Some sub-units have direct governance, i.e. exclusive power over some policies
	\item Over-representation of (some) territorial sub-units in upper house
\end{enumerate}
Where only some regional sub-units have exclusive policy-making power and special representation, this is sometimes called \textbf{Asymmetric Federalism.}

\subsection{Examples}
Below are some examples:
\subsubsection{USA}
Federalism design in the USA:
\begin{itemize}
	\item \textbf{Geopolitical Division:} 50 states recognised by the US Constitution
	\item \textbf{Independence:} separate state elections, constitutions, courts, laws etc.
	\item \textbf{Direct Governance:} constitution preserves state rights over all policies not explicitly allocated to the federal government and is protected by the Supreme Court
	\item \textbf{Territorial Representation:} each state has 2 members of the Senate who are directly elected\\
		Malapportionment - issues of Democracy:\\
		Wyoming has 280k people per Senator\\
		California has 18.63m people per Senator
\end{itemize}

\subsubsection{Germany}
Federalism to prevent too much centralised power:
\begin{itemize}
	\item \textbf{Geopolitical Division:} 16 states (Lander) recognised by the German constitution
	\item \textbf{Independence:} each state has its own elections, constitutions, judges, laws etc.
	\item \textbf{Direct Governance:} there is a catalogue of competences in the constitution, if they conflict i.e. both central and state government have the competency, central overrules
	\item \textbf{Territorial Representation:} the state government sit in upper house with votes by population proportion
\end{itemize}

\subsubsection{India}
\begin{itemize}
	\item \textbf{Geopolitical Division:} 28 states and 7 union territories, set up by constitution and legislative statute
	\item \textbf{Independence:}\\
		\textit{States:} separate governments, elections, laws, courts etc.\\
		\textit{Unions:} governed directly from the center
	\item \textbf{Direct Governance:} Constitutional division of powers:\\
		\textit{Union:} defence, foreign affairs, citizenship, income and company taxes\\
		\textit{State:} police, justice, health, agriculture, money lending, land taxes etc.\\
		\textit{Concurrent:} marriage, education, labour rights, media etc.
	\item \textbf{Territorial Representation:} 28 states and 2 of the union territories directly elect members of the upper house in proportion to population
\end{itemize}

\subsubsection{UK}
Unitary system
\begin{itemize}	
	\item \textbf{Geopolitical Division:} UK divided into 4 nations: Scotland, Wales, England and Northern Ireland.\\
		England is divided into regions, counties and local councils.\\ 
		All set up by UK legislative statutes
	\item \textbf{Independence:}\\
		Scottish parliament and Welsh, Northern Ireland and London assemblies are elected\\
		Scotland has separate legal tradition and courts\\
		English regional assemblies (non-elected) were abolished in 2008-10
	\item \textbf{Direct Governance:}\\
		Scotland: direct power in some areas e.g. education, limited tax powers
		Wales, Northern Ireland, London: policy implementation powers, no tax powers
	\item \textbf{Territorial Representation:} Scotland, Wales and Northern Ireland are slightly over-represented in the House of Commons
\end{itemize}

\subsubsection{France}
\begin{itemize}
	\item \textbf{Geopolitical Division:} 13 regions, 96 departments, 342 arrondissements, 3883 cantons and 36569 communes all set up by legislative statute
	\item \textbf{Independence:} all levels of government are elected but no independent legal authority
	\item \textbf{Direct Governance:}\\ 
		Regions have no legislative authority but can raise taxes (but then receive less from central government)\\
		Some discretion on the implementation of laws/spending on secondary education, public transport, universities and business subsidies
	\item \textbf{Territorial Representation:} no separate territorial representation. Upper house is indirectly elected, the system is biased to favour rural areas
\end{itemize}

\subsection{Why Decentralise}
Three key arguments will be given:
\begin{enumerate}
	\item Democratic Accountability/Checks and Balances
	\item Ethnic Divisions/Divergent Policy Preferences
	\item Fiscal Federalism
\end{enumerate}

\subsubsection{Democratic Accountability, Checks and Balances}
Principle of Subsidiarity\\ 
A central authority should have a subsidiary function, performing tasks which cannot be performed effectively at a more local level.

\textbf{Question:} What does effective even mean? This question is \textbf{open to interpretation} as there are externalities of decentralising most policies, such as environmental protection, education, transport etc.

\textbf{Vertical Checks and Balances}\\
Dividing powers between the centre and sub-states is analytically the same as dividing powers between the executive and the legislature (presidentialism) or between two legislative chambers (bicameralism).

\textbf{Question:} How much should a national majority be constrained by local preferences or interests?

Harold Lasky believes that change is endogenous to the system and dependent on what type of checks and balances exist within it. When asked why he supported federalism in the USA, the checks and balances would allow for change to grow.

In the UK, it would be a top-down approach since the checks and balances would prevent federalism and growth of opinion in the UK from the bottom-up.

\subsubsection{Ethnic Divisions}
\textbf{Ethnic Divisions}\\
If ethno-linguistic groups in a society are geographically concentrated, then decentralisation of power can give these groups autonomy over the issues they care about e.g. education, media, language etc.\\
\textbf{Divergent Policy Preferences}\\
More generally, some geographically concentrated social groups might have significantly divergent policy preferences from the national majority, and so prefer decentralised powers on these issues.
\begin{itemize}
	\item Median Scottish voter is to the left of the median UK voter
	\item Median Catalan voter is to the right of the median Spanish voter
\end{itemize}

\subsubsection{Fiscal Federalism}
This is a normative theoretical framework for understanding which functions and instruments should be centralised and which should be decentralised.\\
\textbf{Central Government}
\begin{itemize}
	\item Macro-economic stabilisation e.g. interest rates, currency intervention
	\item Income redistribution e.g. pensions, welfare spending
	\item National public goods e.g. defence
\end{itemize}
\textbf{Regional/Local Government}
\begin{itemize}
	\item Responsible for provision of goods and services whose consumption is limited to their own jurisdictions
	\item Schools, hospitals, roads, local public housing as examples
\end{itemize}
However, for the local government, one could question if these goods are really \textbf{local} if people across states can come and use them etc. That question is also open to interpretation.

\subsection{Consequences of Decentralisation}
Below are a list of consequences that we go into:
\begin{enumerate}
	\item Accommodating/Exacerbating Ethnic Conflict
	\item Market-Preserving Federalism
	\item Parties and Decentralisation
	\item Malapportionment
\end{enumerate}

\subsubsection{Ethnic Conflict}
Conflicting evidence!\\
\textbf{Decreased Separatist Demands}\\
Some voters strategically support separatists, then when some decentralisation is granted, they go back to mainstream/national parties e.g. Basque Country, Belgium in 90s and Canada\\
\textbf{Increased Separatist Demands}\\
Limited autonomy is granted, separatists win regional election and use their powers to demonstrated that they can be trusted in the government which leads to more demands for autonomy and independence e.g. Scotland, Malaysia, Catalonia and Belgium in 2000s

\subsubsection{Market-Preserving Federalism}
Weingast (1995) argues that federalism drives economic growth. This is because \textbf{thriving markets require political institutions that credibly commit the state to honour economic and political rights.}

Federalism can also lead to \textbf{regulatory competition} between states for better policies and better regulatory standards (known as the California effect)

Federalism can also lead to a \textbf{race to the bottom} as states cut their standards, welfare costs and taxes to attract business (known as the Delaware effect)

Hence, there are two sides of the same coin which can lead to prosperous growth or growth at the cost of good and proper regulation.

\subsubsection{Parties and Decentralisation}
Preferences over decentralisation should follow from preferences over policy. Toubeau and Wagner (2015) argue that parties attitudes to decentralisation is influenced by these parties positions on:
\begin{itemize}
	\item \textbf{Economic redistribution/efficiency:} decentralisation limits the redistributive capacity of central government and increases local economic accountability $\rightarrow$ economically right parties are more pro-decentralisation
	\item \textbf{Cultural identity:} decentralisation of power undermines national cultural homogeneity $\rightarrow$ socially liberal parties are more pro-decentralisation
\end{itemize}


\subsubsection{Malapportionment}
One common consequence is this unfair representation due to federalism. This is because representation is based on territorial units rather than being based on the representation of people.

\newpage
\subsection{In Sum}
\begin{itemize}
	\item There are growing demands for more decentralisation in many established democracies
	\item Federalism is a formal and permanent territorial division of power between the centre and the states
	\item Decentralisation of power can lead to:
		\begin{itemize}
			\item More political accountability
			\item More checks and balances
			\item Decreased Ethnic conflicts
			\item Policy innovation
			\item Better economic performance
		\end{itemize}
	\item But it can also lead to:
		\begin{itemize}
			\item Policy gridlock
			\item Increased separatist demands
			\item Concerns about over-representation
			\item Negative policy spillovers
			\item Pressure to reduce taxation and regulation
		\end{itemize}
\end{itemize}
Further concepts to remember:
\begin{itemize}
	\item \textbf{Congruent Federalism:} territorial units share the same political cultures
	\item \textbf{Incongruent Federalism:} units have different political sub cultures
	\item \textbf{Devolution:} unitary state granting powers (this is not Federalism)
\end{itemize}

\newpage
\section{Public Spending and Economic Equality}
This section goes into the variations in public spending and how they are affected by democracy. Outline is as follows:
\begin{itemize}
	\item Variations in public spending
	\item Relationship between Public Spending and Equality
	\item Democracies vs Non-democracies
	\item Between Democracies
\end{itemize}

\subsection{Variations in Spending}
OECD Countries do spend from 30\% to 50\% of their GDP on public spending. A large amount is allocated to Social Protection which is redistribution of wealth. Further, as countries get richer, there is an upwards trend in public spending. Some countries are off this trend though, such as the USA and Ireland.

Even amongst big countries there is big variation. A way in which we can evaluate this further is the pre and post-tax \textbf{Gini Coefficients.} This shows how unequal a society was before taxation and then public spending. \textbf{Refer to lecture slides for graphs.}

\subsection{Democracies vs Non-democracies}
Do democracies lead to more redistribution? Meltzer and Richard (1981), Boix (2003) and Larcinese (2007)
\subsubsection{Democracies}
\begin{itemize}
	\item Level of public spending, in a simple democracy, should be decided by the median voter.
	\item But, lower turnout in elections means less public spending due to skewed turnout whereby those that turnout are richer than the median hence unrepresentative of all eligible to vote (Larcinese 2007)
	\item This is because \textbf{the median voter is richer than the median citizen}
	\item Boix also shows that democracies only redistribute more wealth than non-democracies when you have high electoral turnout!
\end{itemize}

\subsubsection{Non-Democracies}
\begin{itemize}
	\item Level of public spending decided by ruling elite
	\item Ruling elite much richer than median voter
	\item Elites fear democracy because of a higher redistribution of wealth
\end{itemize}
Hence, we can plausibly conclude that there is more redistribution in democracies than non-democracies.



\subsection{Between Democracies}
What about these factors and their effects?
\begin{itemize}
	\item Institutions: Pres vs Parl
	\item Institutions: Proportional vs Majoritarian
	\item Voter Turnout, Geography and Ethnic Diversity?
	\item Do parties even matter?
	\item 'Taxing the Rich' Scheve and Stasavage (2016)
\end{itemize}

\subsubsection{Effect of Regime Type}
Here is a brief history outline:
\begin{itemize}
	\item Status Quo c. 1930: no democracy has a welfare state
	\item 1930s and 40s: economic depression and war
	\item Citizens and the median voter demand increased public spending:
		\begin{itemize}
			\item New Deal by Roosevelt and Truman in the USA
			\item European Social Democrats arise post WW2
		\end{itemize}
\end{itemize}
\textbf{Presidential:}
\begin{itemize}
	\item More veto players means it is difficult to change existing policies
	\item Weak and non-cohesive parties make it difficult for centre-left to act cohesively
\end{itemize}
\textbf{Parliamentary:}
\begin{itemize}
	\item Fewer veto players and more cohesive parties
	\item Centre-left can come to power and set up a welfare state
\end{itemize}
The expectation is that presidential systems will have lower welfare spending than parliamentary systems.

\subsubsection{Electoral System}
Two key ways in which we can approach this:
\begin{enumerate}
	\item Persson and Tabellini (2003):
		\begin{itemize}
			\item Majoritarian: single-party government, spending close to preferences of the median voter as they rely on this voter to maintain power
			\item Proportional: coalition government, parties have different spending priorities which increases spending
		\end{itemize}
	\item Iversen and Soskice (2003), Chang (2008), \textit{inter alia}:
		\begin{itemize}
			\item Majoritarian: centre-left voters concentrated in cities, left parties either compromise or centre-right win elections which leads to lower welfare spending
			\item Proportional: centre-left form coalitions with liberals and elections are fought on general public goods rather than local goods leading to higher welfare spending
		\end{itemize}
\end{enumerate}
Hence, the expectation is that majoritarian electoral systems have lower welfare spending than proportional electoral systems.

\subsubsection{Geographic and Ethnic Diversity}
Alesina and Glaeser (2004) talk about geographic and ethnic factors in addition to the role of political institutions.\\\\
\textbf{Geographic Disparity:}
\begin{itemize}
	\item Low population density and capital not the biggest city
	\item Weaker trade union movements and weaker socialist parties, the USA as a huge country makes it difficult
	\item Leads to lower welfare spending
\end{itemize}
Works the opposite as well!\\\\
\textbf{Ethnic and Linguistic Diversity:}
\begin{itemize}
	\item Less national solidarity
	\item Median voter opposed to transfers to look after poor members of a different ethnic or linguistic group
	\item Leads to lower welfare spending
\end{itemize}
Again, works the opposite way as well such that homogeneity may breed more solidarity and more public spending of closer kin.

\subsubsection{Do Parties Matter?}
Two theses:
\begin{enumerate}
	\item \textbf{Parties Don't Matter:}\\
		In democracy, party policies converge on the median voter (Downs)\\
		Parties compete on the competence and leadership ability of their leaders\\
		Whichever party is in government will have the same policy
	\item \textbf{Parties Do Matter:}\\
		Parties represent particular electoral constituencies with different public spending preferences\\
		Working class leads to more welfare spending\\
		Middle-Upper Class leads to less welfare spending
\end{enumerate}
The electoral system should matter, in that convergence on the median is likely to be stronger in Majoritarian systems than in PR systems. Hence, parties should \textbf{matter more in PR systems}.

Blais et al. (1993) show that parties do matter but not so much. The difference is moderate across both time and space. Their data shows that public spending increases by only 2\% which seems less than one might think a priori. They state in their conclusion:
\begin{center}
	\fbox{\begin{minipage}{.8\textwidth}
		\begin{center}
		\textbf{A change in the composition of government is not systematically followed by a shift in public spending.}
		\end{center}
	\end{minipage}}
\end{center}

\subsubsection{Taxing the Rich}
Most of the variation has been across time instead of across country. Ref. Piketty Graph. Hence they give some normative justifications for taxation as a means for redistribution:
\begin{itemize}
	\item \textbf{Equal Sacrifice:}\\
		We should all pay the same percentage of our incomes\\
		Rich pay more in total because they earn more
	\item \textbf{Ability to Pay:}\\
		People who earn more are more able to pay more\\
		Progressive taxation with higher income groups
\end{itemize}
However, evidence suggest that countries have only managed to implement high marginal tax rates on high income groups after periods of war. This suggests that only when there is mass mobilisation during wartime is there sufficient public support for forcing high income groups to pay more as \textbf{compensation for not fighting.}

Evidence does show increased income tax and then a gradual decline of taxation across these crucial periods of war.

\subsection{In Sum}
Richer countries tend to spend more redistributing wealth than poorer countries and public spending does reduce income inequality. This can be seen through the pre and post redistribution Gini coefficients. However, there is a big variation between rich countries of this level of redistribution.

Democracies redistribute more via welfare spending that non-democracies but democracies don't redistribute as much as one might expect:
\begin{itemize}
	\item Lower turnout $\rightarrow$ less redistribution
	\item Presidential systems redistribute \textbf{less} than parliamentary systems
	\item Majoritarian electoral systems redistribute \textbf{less} than PR systems
	\item Geographic disparity and ethno-linguistic diversity leads to \textbf{less redistribution}
	\item Left-wing parties in government redistribute \textbf{more} than right-wing parties in government \textbf{although not by much}
	\item Only after ware have countries been able to implement heavy progressive taxation
\end{itemize}
Relationships are very difficult to disentangle though!!


\newpage
\section{Politics in Ethnically Divided Societies}
The standard view in political science is that democracy only works \textbf{if there are cross-cutting cleavages} where social divisions do not reinforce each other. If they do reinforce, identities can become rigid and hard to govern.\\\\
Rabushka and Shepsle (1972) state that: \textit{Is the resolution of intense but conflicting preferences in a deeply divided society manageable in a democratic framework? We think not.}\\\\
Lijphart (1977) states that: \textit{it may be difficult, but it is not at all impossible to achieve and maintain stable democratic government in a plural society.}\\\\
The outline for this section is as follows:
\begin{itemize}
	\item Definitions and Theories
	\item Explaining Ethnic Conflict and theories of identity
	\item Democracy in Ethnically-Divided Societies
		\begin{itemize}
			\item Lijphart vs Horowitz
			\item Examples including Northern Ireland and Nigeria
			\item Electoral Systems and ethnic Voting by Huber
		\end{itemize}
\end{itemize}

\subsection{Definitions and Theories}
Below is an overview of a definition we associate with this theory and how we can measure fractionalisation.
\subsubsection{Definition}
Jim Fearon (2003) defines Ethnic Groups:
\begin{itemize}
	\item Membership in the group is reckoned primarily by descent by both members and non-members
	\item Members are conscious of group membership and view it as important
	\item members share some distinguishing cultural features such as language, religion and customs which they share in common
	\item These cultural features are held as valuable
	\item Group has a homeland
	\item Group has a shared history
	\item Group is potentially stand-alone in a conceptual sense i.e. not a caste like European Nobility
\end{itemize}
One way in which you can measure it is by graphing largest ethnic groups to the second largest ethnic group. Here are the types:
\begin{itemize}
	\item \textbf{Homogeneous:} one dominant ethnic group
	\item \textbf{Plural:} a 50/50 split which can lead to serious divide in politics
	\item \textbf{Heterogeneous:} many different groups found in society
\end{itemize}

\subsubsection{Ethnic Fractionalisation}
The Fractionalisation Index is as follows:
\begin{center}
	$F \equiv 1 - \Sigma ^n _{i=1} p^2_i$\\
	Where p is population share of an ethnic group
\end{center}
The index ranges from 0 to 1. If there are two groups and they occupy 50\% of the population each, the index would be 1. The higher the fractionalisation, the closer the index is to 1.

Most Advanced Democracies are also now multi-ethnic societies with the UK having a fractionalisation index of 0.3 and in London alone, an index level of 0.7 e.g. Sadiq Khan vs Zac Goldsmith.

\subsection{Theories of Ethnic Identity}
Three theories:
\begin{enumerate}
	\item \textbf{Primordial} (Smith 1991)\\
		National and ethnic groups are ancient, natural, permanent and objective phenomena. Institutions and policies have little ability to modify these identities and should instead seek to accommodate ethnic identity.
	\item \textbf{Constructivist} (Anderson 1983)\\
		National and ethnic groups are \textbf{constructed} via collective histories and experiences such as civil wars, nation building. Institutions and policies have some/weak effects on identity formation.
	\item \textbf{Instrumental} (Posner 2004)\\
		Ethnic identification is chosen by individuals in response to changing economic and political incentives e.g. the political mobilisation of one group. Institutions and policies have strong effects on identity formation.
\end{enumerate}

\subsubsection{Example: Chewas and Tumbukas (Posner 2004)}
\textbf{In Zambia} both Chewas and Tumbukas are small groups relative to the country as a whole.

\noindent\textbf{In Malawi} both groups are significant ethnic groups relative to the country as a whole.

Posner stats that political salience of a cultural cleavage depends not on the nature of the cleavage itself (since it is identical in both countries) but on the sizes of the groups it defines and whether or not they will be useful for political mobilisation. Hence, since Malawi has big groups which have more political weight, we see more conflict.

This shows that \textbf{context is extremely important.}

\subsection{Democracy and Divisions}

\subsubsection{Problems of Majoritarian Democracy}
Majoritarian democracies are based on winner-takes-all principles i.e. the Westminster Model.\\\\
\textbf{Lijphart (1985) states:} \textit{the core problem of majoritarianism is its potential for majority dictatorship and the permanent exclusion of ethnic minorities}\\\\
\textbf{Sisk (1996) states:} \textit{simple majority rule results in minimum winning coalitions that tend to exclude a significant minority; when minority preferences are intense and there is little chance of the minority becoming a majority, a recipe for conflict exists}

\subsubsection{Lijphart vs Horowitz}
These two present solutions to the issues present in democracy. They are listed below:\\\\
\textbf{Arend Lijphart}\\
Ethnic identities are permanent i.e. \textbf{Primordial} so institutions need to be designed to accommodate these identities:
\begin{itemize}
	\item Incongruent federalism
	\item PR electoral system
	\item Power-sharing executive
	\item Parliamentary
\end{itemize}
\textbf{Donald Horowitz}\\
Identities are malleable i.e. \textbf{Constructivist or Instrumental} so institutions should be designed to force elites to appeal across ethnic divisions:
\begin{itemize}
	\item Preferential Voting e.g. Alternative or Single-Transferable
	\item Cross-group coalitions
	\item Presidential
\end{itemize}

\subsubsection{Lijphart's Consociationalism}
Four characteristics:
\begin{enumerate}
	\item Sharing of executive power (each ethnic group represented in cabinet)
	\item Group autonomy (in education, language rights etc.)
	\item Proportionality (in parliament, civil service, policy etc.)
	\item Mutual Veto (for each ethno-linguistic-religious group) such that no one group imposes over another
\end{enumerate}
\textbf{Assumption:} Ethno-political instability can be prevented through the institutionalised guarantee of political representation for all major ethnic groups in a society.

\subsubsection{Horowitz's Integrative Power-Sharing}
Five elements:
\begin{enumerate}
	\item Dispersion of power to take the heat off a single focal point
	\item Territorial devolution to emphasise intra-ethnic competition
	\item Institutions that create incentives for inter-ethnic cooperation e.g. STV and AV
	\item Policies that encourage alignments based on alternative social alignments
	\item Redistribution of resources to reduce disparities between groups
\end{enumerate}
Vote-pooling electoral systems offer more effective incentives for inter-ethnic coalitions and the reduction of ethnopolitical tensions than the security offered to ethnic minorities in consociational power-sharing agreements. This is approaching the divisions in society with an optimistic outlook for cooperation between groups.

\subsubsection{Example: Northern Ireland}
Beginning with characteristics:
\begin{itemize}
	\item \textbf{Societal Structure}\\
		Protestants: approx 55\%\\
		Catholics: approx 45\%
	\item \textbf{Identity}\\
		Protestants identity with the UK (Unionists)\\
		Catholics identify with the Republic of Ireland (Republicans/Nationalists)
	\item \textbf{Geographic Divisions}\\
		Between counties/cities in Northern Ireland\\
		Within cities e.g. Catholic and Protestant estates in Belfast
	\item \textbf{History of Conflict}\\
		1960s $\rightarrow$ emerging conflict with British troops deployed in 1969\\
		1972 $\rightarrow$ Bloody Sunday with 14 civil rights marchers killed\\
		1970s to 1990s $\rightarrow$ escalation of conflict with IRA/UVF/UDA paramilitaries
\end{itemize}
\textbf{Good Friday Agreement}
\begin{itemize}
	\item \textbf{Northern Irish Assembly}\\
		Devolved powers elected by STV in multi-member districts (Horowitz)
	\item \textbf{Power-sharing Executive - Lijphart}\\
		First minister from largest party\\
		Deputy First minister from second largest part\\
		Cabinet ministers appointed in proportion to parties strength i.e. not chosen by minister
	\item \textbf{Consensual Assembly - Lijphart}\\
		All members have to declare as either \textbf{unionist, nationalist or other}\\
		Key decisions need either parallel consent ($>50\%$ of both communities)\\
		Or weighted majority (60\% overall and $>40\%$ in both communities)\\
		Committee chairs, public bodies e.g. police appointed to ensure large representation from both communities
\end{itemize}

\subsubsection{Example: Nigeria}
Beginning with characteristics:
\begin{itemize}
	\item \textbf{Societal Structure}\\
		250 ethnic groups with a population of 170m\\
		3 main groups making 68\%: Hausa: 29\%, Yoruba: 21\% and Ibo: 18\%\\
		Groups are geographically concentrated\\
		50\% Muslim, mostly in the North\\
		50\% Christian, mostly in the Centre and South
	\item \textbf{Recent Political History}\\
		1960: Independence from UK\\
		1963: Federal System set up with devolved powers\\
		1960s: Military coups and civil war\\
		1999: Return to democracy with a presidential system\\
		Ethnic conflicts continue:
		\begin{itemize}
			\item North: rising Islamic fundamentalism
			\item South East: conflict over oil extraction and exploitation
		\end{itemize}
\end{itemize}
\textbf{Nigerian Politics Today}
Below is an overview of the system today. It is more of a Horowitz success story.
\begin{itemize}
	\item \textbf{President}\\
		Muhammadu Buhari\\
		Directly elected in a two-round run-off election\\
		Needs a majority or plurality of national vote with at least 25\% in two thirds of all federal states
	\item \textbf{National Assembly}\\
		House of Reps: 360 members in single-member constituencies (FPTP)\\
		Senate: 109 members, elected in 36 3-seat constituencies (each state) plus 1 seat in the national capital (Abuja)
	\item \textbf{Main Political Parties}
		\begin{itemize}
			\item People's Democratic Party (PDP): centre-right, national party
			\item All Progressive Congress (APC): centre-left on economics, centre-right on social
			\item All-Nigeria Perople's Party (APP): centre-right support mainly in north
			\item Action Congress of Nigeria (ACN): centre-right, mainly in South
			\item Congress for Progress Change (CPC): centrist, mainly in North
			\item All Progressives Grand Alliance (APGA): centre-left, mainly in South
		\end{itemize}
	\item \textbf{Federalism}\\
		With significant autonomy for states e.g. Sharia Law in some northern states
\end{itemize}

\subsection{Electoral Systems and Ethnic Voting}
\textbf{Lijphart states:} PR is \textbf{good} because it promotes ethnic representation whereas majoritarian systems lead to ethnic conflict.
\\\\
\noindent\textbf{Horowitz states:} PR is \textbf{bad} because it promotes ethnic polarisation, whereas majoritarian systems should promote less ethnic voting
\\\\
\noindent Hence, Huber asks: is there more ethnic voting under PR than majoritarian systems?

\subsubsection{Measurement of Ethnic Voting}
Two main types of ethnic voting are present:
\begin{itemize}
	\item \textbf{Party Based Ethnicization}\\
		Parties are catering towards one particular ethnic group e.g. Belgium with its French parties and Dutch parties
	\item \textbf{Group Based Ethnicization}\\
		Parties are catering towards multiple ethnic groups but limited amounts. Hence there is group based ethnic voting where groups vote for a particular party
\end{itemize}

\subsubsection{Conclusion for Huber}
\textbf{Key takeaways}
\begin{itemize}
	\item PR is indeed associated with lower levels of civil conflict
	\item However, it is not for Lijphart's reasons
	\item It is because of easy party formation that parties are formed to appeal on bases other than ethnic divisions
	\item This diminishes the salience of ethnicity
\end{itemize}
 
\subsection{In Sum}
Most countries in the world either already have multi-ethnic societies or are becoming multi-ethnic societies due to immigration. This presents a range of challenges for democratic politics:
\begin{itemize}
	\item Conflictual political preferences
	\item Minority rights vs common norms
	\item Representation of minorities in politics and policy-making
	\item Competition for public resources
\end{itemize}
There are also some empirical regularities:
\begin{itemize}
	\item Multicultural/pluralist policies are correlated with more tolerant societies
	\item Power Sharing can sometimes work in ethnically-divided societies
	\item Conflict can often be instrumental i.e. only salient when mobilized
	\item PR systems are better than majoritarian systems in ethnically-divided societies
\end{itemize}

\newpage
\section{Past Paper Questions}

\subsection{Consensus vs Majoritarian}
\subsubsection{Consensus Democracy vs Majoritarian Democracy}
\textbf{Question:} Consensus Democracy is better than majoritarian democracy. Discuss.
\\\\



\subsection{Presidential vs Parliamentary}

\subsubsection{Gridlock in Systems}
\textbf{Question:} Presidential systems, as opposed to parliamentary ones, result in gridlock. Discuss
\\\\



\subsubsection{Stability in Regimes}
\textbf{Question:} Parliamentary systems are more conducive to stable democracy than presidential systems. Discuss.
\\\\
Referencing Perils of Presidentialism by Juan Linz

\subsection{Cabinets and Coalitions}

\subsubsection{Coalitions}
\textbf{Question:} Coalition governments are the best type of government for every country. Discuss.
\\\\
\subsection{Federalism and Decentralisation}

\subsubsection{Democracy and Diversity}
\textbf{Question:} Federalism is the most appropriate way of balancing democracy and diversity. Discuss
\\\\
\subsection{Public Spending and Inequality}

\subsubsection{Varying Spending}
\textbf{Question:} Which institutions can best explain variations in public spending across countries?
\\\\

\subsubsection{Inequality}
\textbf{Question:} Why is economic inequality higher in some countries than in others?
\\\\



\subsection{Politics in Ethnically Diverse Societies}
\subsubsection{Tension and Decentralisation}
\textbf{Question:} Does decentralisation of power help reduce ethnic tensions?
\\or\\
\textbf{Question:} Is decentralisation the best way to manage ethnic tensions in a divided society?
\\\\

\subsubsection{Diversity and Provision}
\textbf{Question:} Does ethnic diversity undermine the provision of public goods?
\\\\

\end{document}

