\documentclass[12pt, letterpaper]{article}
\usepackage[margin=1in]{geometry}
\usepackage{graphicx}
\usepackage{amsmath}
\usepackage{amssymb}
\usepackage{tikz}
\usepackage{pgfplots}
\usepackage{tcolorbox}
\pgfplotsset{compat=1.16}
\graphicspath{ {figures_ST107/} }
\tcbuselibrary{theorems}

\newtcbtheorem[number within=section]{theo}{Note}%
{colback=white!5,colframe=black!35!black,fonttitle=\bfseries}{th}

\title{
	{ST107 Quantitative Methods}\\
	{\large{Professor James Abdey}}\\
	{\large{Revision Document}}
}
\author{Cedric Tan}
\date{May 2019}

\begin{document}
\maketitle
\abstract{

This is a review of mathematics lectures by James Abdey in Lent Term of 2019. The notes are mine fully and may not be authentic to the lecturer's as they have been modified.

The format of this material is usually recounted slide by slide but some slides may be merged together as the material fits appropriately with one another.
}

\newpage
\tableofcontents
\newpage

\section{Data Viz and Descriptive Statistics}
This section covers critical points to remember for the exam.
\begin{theo}{Variables}{dvds_1}
	There are two types of variables considered within the course:
	\begin{enumerate}
		\item \textbf{Discrete:} thing you can count such as the number of passengers on a flight and the number of telephone calls received each day in a call centre
		\item \textbf{Continuous:} things you can measure such as height, weight and time all to several decimal places
	\end{enumerate}
\end{theo}
Data can also be presented in a variety of different ways such as Dot Plots, Histograms along with Stem and Leaf diagrams. All of these could be asked in the exam.

\begin{theo}{Classifying Variables}{dvds_2}
	There are two main ways of classifying variables:
	\begin{itemize}
		\item \textbf{Measurable:} where there is a generally recognised method of measuring the variable of interest e.g. cm for height
		\item \textbf{Categorical:} where no such method exists
		\begin{itemize}
			\item \textit{Ordinal:} categorised into a sensible order
			\item \textit{Nominal:} categorised by names
		\end{itemize}
	\end{itemize}
	Further, there are two key properties of a data set:
	\begin{enumerate}
		\item \textbf{Measures of Location:} a central point about which the data tend, also known as measures of central tendency
		\item \textbf{Measures of Dispersion:} a measure of the variability of the data i.e. how spread out the data are about the central point, also known as measures of spread
	\end{enumerate}
\end{theo}



\end{document}