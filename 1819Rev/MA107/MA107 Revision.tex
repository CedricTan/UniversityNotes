\documentclass[12pt, letterpaper]{article}
\usepackage[margin=1in]{geometry}
\usepackage{graphicx}
\usepackage{amsmath}
\usepackage{amsfonts}
\usepackage{amssymb}
\usepackage{tikz}
\usepackage{pgfplots}
\pgfplotsset{compat=1.16}
\graphicspath{ {figures_MA107/} }

\title{
	{MA107 Quantitative Methods}\\
	{\large{Professor James Ward}}\\
	{\large{Revision Document}}
}
\author{Cedric Tan}
\date{May 2019}

\begin{document}
\maketitle
\abstract{

This is a review of mathematics lectures by James Ward in Michaelmas Term of 2018. The notes are mine fully and may not be authentic to the lecturer's as they have been modified.

The format of this material is usually recounted slide by slide but some slides may be merged together as the material fits appropriately with one another.

This material tries to go over key concepts but not all lecture slides. This means topics which were found to be more complex for me or requires formulas that are critical to problem-solving success.
}

\newpage
\tableofcontents
\newpage

\section{Mathematical Terms and Notation}
This covers sets and functions
\subsection{Sets and their Members}
\textbf{A set is a collection of objects which is defined in a precise way so that any given object is either in the set or not in it.}
\begin{itemize}
	\item We call the objects in a set its members or elements
	\item We use curly brackets when indicating a set by listing its members\\
		\textbf{Example:} \{1, 2, 3, 4, 5, ....\}
	\item In particular, sets may have a finite or infinite number of members
\end{itemize}
Some sets are important to remember as they have special names:
\begin{itemize}
	\item $\mathbb{N}$ is the set of \textbf{natural numbers} i.e. \{1, 2, 3, 4, ... \}
	\item $\mathbb{Z}$ is the set of \textbf{integers} i.e. \{..., -2, -1, 0, 1, 2, ...\}
	\item $\mathbb{R}$ is the set of \textbf{real numbers} visualised as the real line
	\item $\varnothing$ is the \textbf{empty set} i.e. \{ \}
\end{itemize}
If we want the set of objects in another set, say $A$, which have a certain property, we use the notation:
\begin{center}
	$\{ x \in A\;|\; property\;x\;must\;satisfy\}$
\end{center}
i.e. this is the set of all members of the set A that have the property.
\\\\
Lastly, we define $\mathbb{R_+}$ to be the set of non-negative real numbers, i.e.
\begin{center}
	$\mathbb{R_+} = \{x \in \mathbb{R}\; | \; x \geq 0 \}$
\end{center}
\subsection{Comparison of Sets}
We use the notation $'S \subseteq T'$ to denote that $S$ is a \textbf{subset} of $T$ i.e. that all members of the set $S$ are also members of the set $T$.
\begin{itemize}
	\item \textbf{Example:} $\mathbb{N} \subseteq \mathbb{Z},\; \mathbb{N} \subseteq{R}$
	\item \textbf{Example:} $\mathbb{R_+} \subseteq \mathbb{R}, \; \mathbb{R} \nsubseteq \mathbb{R_+}$
\end{itemize}
\subsection{New sets from old}
The \textbf{union} of two sets, $S$ and $T$ denoted $S \cup T$ is defined as:
\begin{center}
	$ S\cup T = \{x\;|\; x \in S\; or\; x \in T\}$
\end{center}
The \textbf{intersection} of two sets $S$ and $T$ denoted $S \cap T$ is defined as:
\begin{center}
	$S \cap T = \{x\;|\; x \in S\; and\; z \in T\}$
\end{center}
The \textbf{difference} of two sets $S$ and $T$ denoted $S \setminus T$ is defined as:
\begin{center}
	$S\backslash T = \{x\;|\; x \in S\; and\; x \notin T\}$
\end{center}
The \textbf{Cartesian Product} of two sets $S$ and $T$ denoted $S \times T$ is defined as:
\begin{center}
	$S \times T = \{(x, y)\;|\;x \in S\; and\; y \in T\}$
\end{center}
\subsubsection{Cartesian}
If we take the Cartesian product of a set with itself i.e. $S \times S$, we usually denote this $S^2$. This is import as if we take the Cartesian product of $\mathbb{R}$ itself we get:
\begin{center}
	$\mathbb{R}^2 = \mathbb{R\times R} = \{(x,y) \; | \; x \in \mathbb{R} \; and \; y \in \mathbb{R}\}$
\end{center}
This is how we denote the $xy$-plane in terms of sets e.g. $\mathbb{R}^2_+$ would represent the set of all points in the \textbf{non-negative quadrant} of the $xy$-plane. 

\section{Functions and Demand and Supply}
\subsection{Demand}
If a product is offered for sale at price $p$ (per unit) what quantity $q$ will the market demand (i.e. purchase)?
We use a demand model function to understand this concept:
\begin{center}
	We denote the demand function as $q^D$\\
	It is related to price because it is a function of price: $q = q^D(p)$
\end{center}
If the curve is suitably well-behaved we can ask the \textbf{inverse question:} if a quantity $q$ is demanded, what price $p$ will the product sell at?
\begin{center}
	We denote the inverse demand function as $p^D$\\
	It is related to quantity because it is a function of quantity: $p = p^D(q)$
\end{center}
For example, if we modelled the demand curve by the straight line $5p + q = 40$ we can figure out these two functions be rearranging:
\begin{itemize}
	\item $q^D(p) = 40-5p\; as \; q = 40 - 5p$
	\item $p^D(q) = \frac{40-q}{5}\; as \; p = \frac{40-q}{5}$
\end{itemize}
\subsection{Supply}
If a product can be placed for sale at price $p$ (per unit) what quantity $q$ will the supplier put on the market (i.e. produce)?
We use a supply model function to understand this concept:
\begin{center}
	We denote the demand function as $q^S$\\
	It is related to price because it is a function of price: $q = q^S(p)$
\end{center}
If the curve is suitably well-behaved we can do the inverse again:
\begin{center}
	We denote the inverse supply function as $p^S$\\
	This is a function of quantity: $p = p^S(q)$
\end{center}
The example is below with the supply curve $5p - q = 40$:
\begin{itemize}
	\item $q^S(p) = 5p-40 \; as \; q = 5p-40$
	\item $p^s(q) = \frac{40+q}{5} \; as \; p = \frac{40+q}{5}$
\end{itemize}
Hence, from the above two, we can begin to model some functions.
Further, well-behaved curves mean that the curve will give us a proper inverse function.
The inverses can be verified through using composite functions e.g. $(p^S \circ q^S)(p) = p$
\subsection{Equilibrium}
The reason we use these models is to inform us on what $q$ and $p$ will actually be achieved in the market. We have two sets:
\begin{center}
	$D = \{(q,p) \;|\; q = q^D(p)\}$ and $S = \{(q,p)\;|\; q = q^S(p)\}$
\end{center}
This tells us the factors affecting the demand and supply of a product in the market. To find the equilibrium i.e. where everything being supplied is being purchased, we need an intersection of these two sets:
\begin{center}
	$E = D \cap S$
\end{center}
This is called \textbf{the equilibrium set} of the market.

\section{Taxation}
Two questions on supply and demand sets:
\begin{enumerate}
	\item If market conditions change, what happens to the equilibrium?
	\item If markets start away from the equilibrium, can it move towards it?
\end{enumerate}
\subsection{Excise Tax}
This is an example of how market conditions can change.
The theory goes as follows: A market is governed by a supply function $q^S(p)$ and the demand function $q^D(p)$. If a tax of $T$ is imposed per unit on the product sold, the supplier will pay $T$ per unit sold.
If a price of $p$ is charged, the supplier gets $p-T$ per unit from the sale.
The price $p$ is the price per unit seen by consumers and $p-T$ is the effective price per unit seen by suppliers.

The market is now disturbed and we get new but related supply and demand functions:
\begin{itemize}
	\item \textbf{New Demand Function:} $q^D_T(p)=q^D(p)$
	\item \textbf{New Supply Function:} $q^S_T(p) = q^S(p-T)$
\end{itemize}
Only the supply function changes if an excise tax is introduced.
We might want to find the new equilibrium which can be solved by:
\begin{center}
	$q^D(p^*_T) = q^S(p^*_T-T) \;or\; q^D_T(p^*_T)=q^S_T(p^*_T)$
\end{center}
An example is given below:
\begin{center}
	\fbox{\begin{minipage}{.8\textwidth}
			\begin{center}
	$D = 5p+q = 40$ and $S = 15p-2q = 20$\\
	$q^D(p) = 40-5p$ and $q^S(p) = \frac{15}{2}p-10$\\
	Equating the two we can get $p^*$ and $q^*$\\
	$p^* = 4$ and $q^* = 20$\\
	Now suppose that we introduce an excise tax into the market to discourage consumption or gain government revenue\\
	The new demand function: $q^D_T(p) = q^D(p) = 40-5p$\\
	The new supply function: $q^S_T(p) = q^S(p-T) = \frac{15}{2}(p-T)-10$\\
	To solve for the equilibrium we equate: $40-5p^*_T=\frac{15}{2}(p^*_T-T)-10$\\
	Solving we can get $p^*_T = 4+\frac{3}{5}T$ and using $q^D_T = 40-5(4+\frac{3}{5}T) = 20-3T$\\
	Hence $E = (20-3T,4+\frac{3}{5}T)$ in the form $E = (q,p)$
\end{center}
\end{minipage}}
\end{center}
From this example, we can see that quantity in the market is reduced by $3T$ and the price increases by $\frac{3}{5}T$ which means that \textbf{not all the tax is passed onto consumers.}

Moreover, the tax cannot be arbitrarily high if we want our market to function as the equilibrium quantity must be positive. Hence:
\begin{center}
	$q^*_T > 0 \rightarrow 20-3T > 0 \rightarrow T<\frac{20}{3}$
\end{center}
Thus the maximum tax that can be charged in this market is $\frac{20}{3}$ units of tax.

Tax revenue can also be maximised through optimisation:
\begin{center}
	$R_T = q^*_T\times T$\\
	$R_T = (20-3T)T = 20T-3T^2$\\
	Differentiating: $\frac{\delta T}{\delta R} = 20 - 6T$\\
	Finding maximum: $6T = 20,\; T = \frac{10}{3}$\\
	Revenue: $20(\frac{10}{3}) - 3(\frac{10}{3})^2 = \frac{100}{3}$\\
	Alternatively you can solve this by completing the square to get $-3(T-\frac{10}{3})^2 + \frac{100}{3}$
\end{center}
Sometimes you will also see Tax as a percentage of the price. This would lead to a slightly different equation than $p-T$:
\begin{center}
	Suppliers see it as: $p-rp=p(1-r)$\\
	So the supply function would be $q^S_r(p)=q^S(p(1-r))$
\end{center}

\section{Recurrence}
We want to solve recurrence of the form $y_t=ay_{t-1}+b$. This is for $t \geq 1$.
\begin{itemize}
	\item If $a=1$ then this is an \textbf{arithmetic progression}
	\item If $b=0$ then this is a \textbf{geometric progression}
\end{itemize}
\subsection{Step 1: Time Independent Solution}
Let's see if there is a time independent solution $y^*$ i.e. we have $y_t = y^*$ for all $t \geq 0$.
This means that $y_t$ is a constant independent of t. If this is the case we have:
\begin{center}
	$y_t = y^*$ and $y_{t-1} = y^*$
\end{center}
That means our equation becomes $y^* = ay^* + b \rightarrow (1-a)y^* = b \rightarrow y^* = \frac{b}{1-a}$.
This works as long as $a \neq 1$
\subsection{The Discrepancy}
This is the difference from the solution we seek and the time independent solution we have such that $z_t = y_t - y^*$. This means we have:
\begin{center}
	$z_t = y_t - y^* \rightarrow z_0 = y_0 - y^*$\\
	This means we have found: $z_t = a^tz_0 \rightarrow y_t - y^* = a^t(y_0-y^*)$\\
	More neatly: $y_t=y^*+(y_0-y*)a^t$\\
	\textbf{This is the time dependent solution to the recurrence equation:}\\
	$y_t = ay_{t-1}+b$\\
	This can help us solve any recurrence equation of the same form.
\end{center}
What happens in the long term depends on the geometric variable:
\begin{itemize}
	\item $a > 1,\;  a^t \rightarrow \infty\; as \; t\rightarrow \infty$
	\item $0 < a < 1,\; a^t \rightarrow 0\; as \; t\rightarrow \infty$
	\item $-1 < a < 0, \; a^t \rightarrow 0\; as \; t\rightarrow \infty$ but oscillates around 0
	\item $-1 < a, \; a^t \rightarrow 0\; as \; t\rightarrow \infty$ but oscillates around 0
\end{itemize}]
\section{Applications}
\subsection{Annuity}
Investment decisions and annuities can be calculated with a recurrence equation:
\begin{itemize}
	\item $y_t = (1+r)y_{t-1}-I$ where $(1+r)$ is the interest rate and $I$ is the amount taken out of the annuity
	\item This is of the form $y_t = ay_{t-1}+b$ which we can solve
	\item $a=(1+r)$ and $b=(-)I$
	\item $y^* = \frac{b}{1-a} = {-I}{1-(1+r)} = \frac{I}{r}$
	\item With our solution of form $y_t=y^* + (y_0 - y^*)a^t$
	\item We have: $y_t = \frac{I}{r} + (P - \frac{I}{r})(1+r)^t$
\end{itemize}
Through this we can figure out our Principal required for a set $I$ to be taken out or the $I$ we can take out for a given Principal:
\begin{itemize}
	\item $P = \frac{I}{r}[1-\frac{1}{(1+r)^n}]$ which is form 1
	\item $I = \frac{Pr(1+r)^n}{(1+r)^n-1}$ which is form 2
\end{itemize}
Both of these can be figured out from the recurrence equation above.
\subsection{Compounding}
Compounding intervals can be different and provide different outcomes:
\begin{itemize}
	\item \textbf{Annually:} $P(1+r)^1$ where $1$ is the number of years
	\item \textbf{Semi-annually:} $P(1+\frac{r}{2})^2$ where $2$ is the number of compounds in a year $\rightarrow$ after $n$ years we have $P(1+\frac{r}{2})^{2n}$
	\item \textbf{In General 1:} $P(1+\frac{r}{m})^m$ where $m$ is the number of compounding intervals
	\item \textbf{In General 2:} $P(1+\frac{r}{m})^{mn}$ where $n$ is years
	\item \textbf{In General 3:} Where $m \rightarrow \infty$ we get: $Pe^{rn}$ where $e^r$ is the rate of growth for a given interest rate compounded continuously
\end{itemize}

\subsection{Present Values}
\begin{itemize}
	\item Capital Growth would be a function of the Principal such that: $C(P) = Pe^{rn}$
	\item Present Value would be a function of Capital Growth such that: $P(C) = Ce^{-rn}$
\end{itemize}
Simple, present value can be found from a given amount at the end of a period such that: $\frac{x}{(1+r)^n}$.
However, this does not take into account present values of annuities. This is more complex and can be seen below:
\begin{center}
	Payment 1 of an annuity has the value of $\frac{I}{(1+r)}$\\
	Payment 2 of an annuity has the value of $\frac{I}{(1+r)^2}$\\
	The sum of all payments: $\frac{I}{(1+r)}+\frac{I}{(1+r)^2}+\dots + \frac{I}{(1+r)^n}$\\
	This is a geometric series with first term $\frac{I}{1+r}$ and common ratio $\frac{1}{1+r}$\\
	Hence we can find it with the formula: $\frac{I}{1+r}(\frac{1-(\frac{1}{1+r})^n}{1-\frac{1}{1+r}})$\\
	More simply: $\frac{I}{r}(1-\frac{1}{(1+r)^n})$
\end{center}

\subsection{Cobweb Model}
In general, the Cobweb Model explains how markets converge to the equilibrium or how it may oscillate away or around it.
\begin{center}
	$q_t=q^S(p_{t-1})$ and $p_t=p^D(q_t)$ if $p_0 \neq p^*$ for $t \geq 1$\\
	Supply Function: $q^S(p) = mp-n$\\
	Demand Function: $q^D(p) = k-lp$\\
	Equilibrium price: $q^S(\bar{p}) = q^D(\bar{p}) \rightarrow m\bar{p} -n = k - l\bar{p} \rightarrow \bar{p} = \frac{k+n}{m+l}$
	Equilibrium quantity: $\bar{q} = k - l\frac{k+n}{m+l} = \frac{km-ln}{m+l}$\\
	$q^S(p) = mp-n \rightarrow q_t = mp_{t-1} - n$\\
	$q^D(p) = k-lp \rightarrow p^D(q)=\frac{k-q}{l} \rightarrow p_t = \frac{k-q_t}{l}$\\
	$p_t = \frac{k-(mp_{t-1}-n)}{l} = -\frac{m}{l}p_{t-1}-+\frac{k+n}{l}$\\
	This gives us a form in a recurrence equation which we can then solve with a time independent solution: $p^* = \frac{k+n}{m+l}$ which is unsurprisingly the equilibrium.
\end{center}
From this:
\begin{itemize}
	\item If $m < l$ then $p_t$ will oscillate decreasingly to $\bar{p}$ so the market is stable
	\item If $m = l$ then $p_t$ will oscillate constantly around $\bar{p}$ so the market is cyclic
	\item If $m > l$ then $p_t$ will oscillate increasingly around $\bar{p}$ so the market is unstable
\end{itemize}

\newpage
\section{Differentiation and Optimisation}

\subsection{Differentiation Basics}
There are four ways in which a suitably well defined function can have a point $x = c$ which makes $f'(c)=0$
\begin{itemize}
	\item Point of inflection from positive to positive gradient
	\item Point of inflection from negative to negative gradient
	\item Local minimum i.e. negative gradient to positive
	\item Local maximum i.e. positive gradient to negative
	\end{itemize}
We can also use the second derivative test to see whether or not the points are \textbf{mins} or \textbf{maxs.}
\begin{itemize}
	\item $f''(c) > 0$ then we have a local minimum
	\item $f''(c) < 0$ then we have a local maximum
	\item $f''(c) = 0$ this tells us \textbf{nothing}
\end{itemize}
So our procedure for optimising a function is:
\begin{enumerate}
	\item Find the stationary points using the \textbf{first order condition}
	\item Find out what they are using out \textbf{tests} 
	\item Evaluate the function at the stationary points
	\item Evaluate the function at the end-points of the interval (if any)
	\item Determine the global maximum or global minimum (if any)
\end{enumerate}

\subsection{Optimisation}
\subsubsection{Profit Maximisation}
Suppose a firm incurs as cost $C(q)$ when it supplies an amount $q$ of its product. It is also a monopoly so the price will be determined by the inverse demand function $p^D(q)$. This means that revenue would be $qp^D(q)$. Such that the profit function is:
\begin{center}
	$\Pi(q) = R(q) - C(q) = qp^D(q)-C(q)$
\end{center}
How do we maximise this function? Given our cost curve is $C(q) = q^3 - 10q^2 +25q + 10$
and our inverse demand function is $p^D(q) = 10-q$
\begin{center}
	$\Pi(q) = q(10-q) - (q^3 -10q^2 + 25q + 10)$\\
	$\Pi(q) = 10q-q^2 - q^3 + 10q^2 -25q -10$\\
	$\Pi(q) = -q^3 +9q^2 -15q -10$\\
	$\Pi '(q) = -3q^2 + 18q - 15$\\
	Factorised: $(-3q + 3)(q - 5)$ and knowing our constraint $0 \leq q \leq 10$\\
	Testing for Mins and Maxs with $\Pi ''(q) = -6q + 18$\\
	$3q=3 \therefore q= 1$ and $q=5$\\ 
	For $q=1,\; -6+18 > 0 \therefore local\;minimum$ so $-(1)^3+9(1)^2 - 15(1) - 10 = -15$\\
	For $q=5,\; -30+18<0 \therefore local\;maximum$ so $ -(5)^3+9(5)^2 -15(5) - 10 = 15$\\
	Test end points: $0 + 0 + 0 -10 =-10$ and $-(10)^3+9(10)^2-15(10)-10 = -260$\\
	Hence Global Max: $15$ and Global Min: $-260$
\end{center}
\subsubsection{Discounting}
Suppose we have the option to buy an antique. Market information about this antique suggests that its value in dollars will be given by: $V(t) = 2000 + 500t$ if we were to buy it now and sell it after $t$ years. We also have a bank account which pays continuously compounded interests at a rate of $5\%$ per annum.
Should we buy the antique? And, if we do, what is the best time to sell it?
\begin{center}
	With interest at $5\%$ we can see our present value of investing the $2000$ principal in a bank account:\\
	$C(P) = Pe^{0.05t}$ and Present Value: $P(C) = e^{-0.05t}$\\
	So if I know that, I can get $V(t)$ in $t$ years time from selling our antique:\\
	$P(t) = V(t)e^{-0.005}$ which is the function we want to maximise
\end{center}
From the above we get a local maximum of $t=16$ from $P(t)=(2000+500t)(e^{-0.05t})$ where we can apply the product rule to get $-100e^{-0.05t} -25te^{-0.05t}+500e^{-0.05t}=(400-25t)(e^{-0.05t})$. This gives us a value of $t=16$. What maximises our value?
\begin{center}
	When $t=16$ we get $[2000+500(16)][e^{-0.05(16)}]= 4,493.29 > P(0)=2000$ when $t=0$\\
	Hence we should buy the antique and sell it after 16 years.
\end{center}

\subsection{Efficient Small Firm}
By small we mean that its output is not large enough to affect the market price and so is it's a \textbf{price taker.} So, if the prevailing price is $p$ and the firm is supplying a quantity $q$ then its profit is given by:
\begin{center}
	$\Pi(q) = pq - C(q)$\\
	$\Pi$ is really a function of two independent variables $p$ and $q$ but because the firm is small, the only variable it has control over is $q$.
\end{center}
There are two questions that the firm should ask:
\begin{enumerate}
	\item Given the prevailing price $p$, should it bother to enter the market?
	\item Given $p$, if it should enter, how does it select the optimal level of output $q$?
\end{enumerate}
Given the profit function, we can differentiate to optimise:
\begin{center}
	$\Pi'(q) = p - C'(q)$\\
	If $q^*$ satisfies the equation $\Pi'(q)=0$ i.e. $p=C'(q^*)$ a Sty. pt. exists at $q=q^*$\\
	Also as $q \geq0$ look at the end point where $q=0$
\end{center}
There are a few issues though:
\begin{itemize}
	\item Can we be sure that the stationary point at $q^*$ is a local maximum?
	\item If it is, what if $q^*$ leads to a profit which is negative?
	\item In such coses, should we produce $q^*$ to cut losses or should we refrain from production entirely?
\end{itemize}
One thing we notice is that the Profit Function is actually a function of $p$ which is not under the control of the firm. We can change this so that it becomes a function of $q^*$ rather than $p$.
\begin{center}
	$\Pi_* = q^*C'(q^*)-C(q^*)$ when we use $p=C'(q^*)$
\end{center}
The above is indirect profit where we can treat $q^*$ as an independent variable which is not dependent on $p$. Below are some critical points on where the firm can produce and what this level of production actually means.
\begin{itemize}
	\item At $q^*=0, \; \Pi_*(0) = -C(0)$ where we do not produce at all
	\item At $q_s >0, \; \Pi_*(q_s) =  -C(0)$ where we cover our fixed costs
	\item At $q_b, \; \Pi_*(q_b) =  0$ where we break even and begin to make a profit
\end{itemize}
However, we remember that $q^*$ is strictly a function of $p$ so we can find the effect of prevailing market prices as they correspond with $p = C'(q^*)$. We use the same types of quantities as above:
\begin{itemize}
	\item $p_s = C'(q_s)$ so for prices $0 \leq p \leq p_s$ we should supply nothing
	\item If $p \geq p_s$ we should start supplying $q^* \geq q_s$ as $C'(q^*) \geq C'(q_s)$
	\item If $p_b = C'(q_b)$ we know for prices $p > p_b$ we will actually make a profit
\end{itemize}
This gives us the supply function:
\begin{center}
	$q^s(p)=
	\begin{cases}
		0 & if \; 0\leq p \leq p_s\\
		q^*(p) & if \; p\geq p_s
	\end{cases}$
\end{center}
\subsubsection{Small Firm Comment 1}
Given a cost function, we can call its derivative the marginal cost.
\begin{center}
	$C'(q) \simeq \frac{C(q+h)-c(q)}{h}$ for $h$ close to $0$.
\end{center}
This is the per unit increase in cost.

\subsubsection{Small Firm Comment 2}
The start-up point $q_s > 0$ is where we just cover our fixed costs i.e. the profit we make is equal to the profit from making nothing:
\begin{center}
	$\Pi_*(q_s) = \Pi_*(0)$\\
	In our model we have: $q_sC'(q_s) - C(q_s) = -C(0)$\\
	$\Pi_*(q)=qC'(q)-C(q)$ and re-arranging this we get: $C'(q_s)=\frac{C(q_s)-C(0)}{q_s}$\\
	If we think about costs: costs = variable costs + fixed costs\\
	$C(q_s) -C(0) = $ variable costs
\end{center}
Hence: marginal cost = average variable cost.
\subsubsection{Small Firm Comment 3}
At the break even point we have:
\begin{center}
	$\Pi_*(q_b)=0$\\
	In our current model: $q_bC'(q)-C(q_b)=0 \rightarrow C'(q_b)=\frac{C(q_b)}{q_b}$
\end{center}
Hence: marginal cost = average cost

\subsubsection{Small Firm Comment 4}
If the firm is not a price taker and it is a monopoly, then we can simply use the equation:
\begin{center}
	$\Pi(q) = qp^D(q)-C(q)$ where $C(q)$ is the cost of production
\end{center}
It is easy to maximise the profit of this firm!

\section{Partial Differentiation}
This section is about looking at changes independently which may contribute to a total overall change. The key thing to remember is that when we differentiate a function $f(x,y)$ we get functions of both $x$ and $y$ such that we get:
\begin{itemize}
	\item \textbf{Two} first-order partial derivatives
	\item \textbf{Four} second-order partial derivatives
\end{itemize}
We call $f_{xy}$ and $f_{yx}$ the mixed second order derivatives. These have a special property:
\begin{center}
	\Large{$\frac{\delta^2 f}{\delta y \delta x} = \frac{\delta^2f}{\delta x \delta y}$}
\end{center}
When differentiating, the variables which you are not differentiating with respect to are treated as constants.
\subsection{Optimising Two Variables}
There are three stationary points when looking at the optimisation of two variables:
\begin{enumerate}
	\item Minimum where both variables are minimised
	\item Maximum where both variables are maximised
	\item Saddle-point where one variable is maximised and the other minimised
\end{enumerate}
Can we use second-order partial derivatives to figure out the nature of stationary points? Not necessarily. Some cases are special where saddle points can give confusing signals such as the second-order condition being greater than 0 for both variable 1 and variable 2.

\subsubsection{Hessian}
We define the Hessian $H(x,y)$ to be:
\begin{center}
	$H(x,y) = f_{xx}f_{xx}-(f_{xy})^2$
\end{center}
Conditions for the Hessian are below:
\begin{itemize}
	\item If $H(X,Y) > 0$ and $f_{xx}(X,Y) > 0$ it is a minimum
	\item If $H(X,Y) > 0$ and $f_{xx}(X, Y) < 0$ it is a maximum
	\item If $H(X,Y) < 0$ it is a saddle point
	\item Other cases, this fails
\end{itemize}


\newpage
\section{Homogeneous Functions}
We say that a function $f(x, y)$ is homogeneous of degree $r$ if:
\begin{center}
	$f(\lambda x, \lambda y ) = \lambda^rf(x, y)$
\end{center}
When looking to see if a function is homogeneous, replace all variables with the same variable but a coefficient of $\lambda$ and see if they factorise out neatly into the form above.

\subsection{Applications}
We can use these functions for production, for example, capital and labour in a function $q{k,l}$. If it is homogeneous of degree $r$ we have:
\begin{center}
	$q(\lambda k, \lambda l) = \lambda^r q(k,l)$
\end{center}
If we have $\lambda > 1$ we have:
\begin{itemize}
	\item $q(k,l)$ giving the amount produced given inputs $k$ and $l$
	\item $q(\lambda k, \lambda l)$ giving the amount produced given inputs $\lambda k$ and $\lambda l$
\end{itemize}
Further, here are some important things to note:
\begin{itemize}
	\item If $r>1$ we have $\lambda^r > \lambda$ hence \textbf{increasing returns to scale}
	\item If $r=1$ we have $\lambda^r = \lambda$ hence \textbf{constant returns to scale}
	\item If $r<1$ we have $\lambda^r < \lambda$ hence \textbf{decreasing returns to scale}
\end{itemize}

\subsection{Euler's Theorem}
If $f(x,y)$ is a homogeneous function of degree $r$ then:
\begin{center}
	\Large{$x\frac{\delta f}{\delta x} + y \frac{\delta f}{\delta y} = rf(x, y)$}\\
\end{center}
In this course you will be asked to verify or explain its economic interpretation. Also notice that:
\begin{itemize}
	\item If capital $k$ is paid for at a rate of: $\frac{\delta Q}{\delta k}=$ Marginal Product of Capital per unit
	\item If labour $l$ is paid for at a rate of: $\frac{\delta Q}{\delta l} =$ Marginal Product of Labour per unit
\end{itemize}

\newpage
\section{Constrained Optimisation}
This is related to producing with a constraint. Here, we want to minimise costs whilst maximising out output. This can be found by the point at which the constraint curve is tangential to the cost curve.
\subsection{Lagrangean}
We use the Lagrangean to solve this constrained optimisation questions:
\begin{center}
	$L(x,y, \lambda) = f(x,y) - \lambda g(x,y)$
\end{center}
When we are asked to minimise $f(x,y)$ subject to $g(x,y) = 0$ we seek to find the points where we minimise the Lagrangean such that:
\begin{center}
	\Large{$\frac{\delta L}{\delta x}=0, \; \frac{\delta L}{\delta y}=0\; and \; \frac{\delta L}{\delta \lambda}=0$}
\end{center}
This gives us three partials:
\begin{enumerate}
	\item $L_x(x,y,\lambda) = f_x(x,y) - \lambda g_x(x,y) = 0$
	\item $L_y(x,y, \lambda) = f_y(x, y) - \lambda g_y(x,y) = 0$
	\item $L_\lambda(x,y, \lambda) = -g(x,y) = 0 \rightarrow g(x,y) = 0$
\end{enumerate}
\begin{center}
	\fbox{\begin{minipage}{.8\textwidth}
			\begin{center}
				Minimise $8l + 10k$ subject to $kl=320$ and $k, l \geq 0$\\
				Constraint function: $g(k,l) = kl-320 = 0$ from $kl=320$\\
				Lagrangean: $L(x,y, \lambda) = 8l + 10k - \lambda(kl-320)$\\
				Then differentiate:
				\begin{itemize}
					\item $L_k = 10 - \lambda l =0$
					\item $L_l = 8-\lambda k = 0$
					\item $l_\lambda = -(kl-320)=0$
				\end{itemize}
				Eliminating $\lambda$ from the first two equations gives us: $\lambda=\frac{10}{l}={8}{k} \rightarrow 5k=4l$\\
				Then substitute into the next equation: $k(\frac{5}{4}k)=320 \rightarrow k^2 = 256 \rightarrow k=\pm 16$\\
				Since $k > 0$ we have $k=16$ and $l=\frac{5}{4}(16)=20$\\
				Hence: $(k^*, l^*) = 16,20)$

			\end{center}	
	\end{minipage}}
\end{center}
\subsection{Lagrangean Multiplier}
This is effectively the \textbf{marginal cost} of production. $\frac{dC}{dQ} = \lambda$. When we differentiate $C$ with respect to $Q$ we get:
\begin{center}
	$\frac{dC}{dQ}=w\frac{dl}{dQ}+v\frac{dk}{dQ}$ from $C(k(Q),l(Q)) = wl(q) + vk(Q)$\\
	Using the Lagrangean form, you can figure out Lambda independently through following the steps previously.
\end{center}

\subsection{Utility Functions}
With budget constraints and budget sets, we can see if an individual is able to afford a bundle. For example, for two goods: $p_1x_1+p_2x_2 \leq M$ we have a constraint of $M$ that the individual cannot spend over.
Individuals may also prefer some bundles over others as well.This is written as:
\begin{center}
	$(a_1, a_2) \succ (b_1, b_2)$
\end{center}
There are certain ways in which we can do this:
\begin{enumerate}
	\item Monetary Value: $p_1a_1 + p_2a_2 > p_1b_1 + p_2b_2$
	\item Quantity: $a_1 + a_2 > b_1 + b_2$
	\item Geometric Average: $\sqrt{a_1a_2} > \sqrt{b_1b_2}$
	\item Substitution: $ma_1 +na_2 > mb_1 + nb_2$
\end{enumerate}
We define a utility function to differentiate between preferences such that $U(a_1, a_2) > U(b_1, b_2)$. If they are equal such that $U(a_1, a_2) = U(b_1, b_2)$ we say that the consumer is indifferent between the two. These curves are all shaped differently. They all give us \textbf{convex sets} which means that any two points in the set connected with a straight line will be contained within the set.Thus, we do the same as the Lagrangean:
\begin{center}
	$p_1x_1+p_2x_2-M=0$\\
	$L(x_1, x_2, \lambda) = U(x_1, x_2) - \lambda(p_1x_1+p_2x_2-M)$
\end{center}
Lambda in this scenario is the \textbf{marginal utility of income} i.e. how much utility we get for the additional amount we spend.

\section{Matrices}




\end{document}
