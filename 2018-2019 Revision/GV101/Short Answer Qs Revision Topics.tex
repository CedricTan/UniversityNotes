\documentclass[12pt, letterpaper]{article}
\usepackage[margin=.5in]{geometry}
\usepackage{graphicx}
\usepackage{tikz}

\title{
	GV101 Intro to PolSci\\
	\large{Simon Hix}\\
	\large{Short Answer Questions Revision Document}
}
\author{Cedric Tan}
\date{April 2019}


\begin{document}
\maketitle
\abstract{This document covers the 8 topics that could be asked in the short answer section of the GV101 paper. There is a section dedicated to each topic and each section will begin with the definitions related to the subject then go into moderate depth on the relevant material before concluding with a half page summary on the topic. Note that this is a collation of readings, lecture notes and further supplementary material that can be found online. All work is not necessarily authentic and has been modified by me to fit my needs for this course}

\newpage
\tableofcontents
\newpage


\section{The Modernisation Theory of Democratisation}

\textbf{Definition:} Classic modernisation theory argues that countries are more likely to become democratic and stay democratic as they develop economically $\rightarrow$ this theory is more prevalent in high-income countries. (Clark, Golder and Golder 2017)

\subsection{Key Ideas}
Modernisation theory argues that all societies pass through the same historical stages of economic development. Those who are not democratic are simply \textbf{labelled as underdeveloped.} Rostow (1960) and Gerschenkron (1962) believed that African, Asian and Latin American countries were simply underdeveloped versions of countries in Europe.

These 'primitive' countries were characterised by large agricultural sectors and small industrial and service sectors. Eventually these countries are supposed to modernise by expanding their industrial and service sectors while reducing their emphasis on the level of agriculture.

\subsection{The Political Science Approach}
Helmed by Lipset (1959, 1960), modern societies, he says, need an appropriate type of government. Przeworski (1998, 2000) states that dictatorships and other types of government are replaced by democracies because:
\begin{itemize}
	\item Social structure becomes more complex
	\item New groups emerge and organise along various lines
	\item Labour requires active cooperation by employees hence the system can no longer be \textbf{command driven.}
	\item Dictatorships lose control and effectiveness as:
		\begin{itemize}
			\item Technological change endows private information and autonomy
			\item Civil democratic society tends to emerge as a result
		\end{itemize}
\end{itemize}
Hence for Przeworski, modernisation theory highlights the idea that as a country develops economically, they will also democratise due to the plurality of opinions that are formed along with the independence individuals develop with their new information.

Lipset also argues that higher income countries will also tend to maintain their democratic status as:
\begin{itemize}
	\item Sustaining the democracy is done through the people and their interests
	\item Interest in democracy is a main concern for these people and it persists as long as Przeworksi's reasons of diverse social structure remain
\end{itemize}

\subsection{Summary of Modernisation Theory}
Summary to be written for extended revision.

\newpage
\section{The Selectorate Theory for Non-Democratic Regimes}

\textbf{Definition:} Selectorate theory helps to explain why we observe tremendous variation in the economic performance of dictatorships. Rather than categorise governments as either democratic or dictatorial, selectorate theory characterises all governments by their location in a two-dimensional institutional space:
\begin{itemize}
	\item One dimension is the size of the selectorate: those with a say in selecting the leader
	\item The second is the size of the winning coalition: those in the selectorate whose support is essential for the leader to stay in office
\end{itemize}

\begin{center}
	$W = Winning\;Coalition$ and $S = Selectorate$
\end{center}

\subsection{Types of Selectorate and Winning Coalition}
Here are various forms of the relationship between Selectorate and Winning Coalition:
\begin{itemize}
	\item Large W and Large S
		\begin{itemize}
			\item Democracies usually
			\item Incentive to produce public goods
			\item Good government performance
			\item High levels of wealth, efficient governane and low rates of corruption and kleptocracy
		\end{itemize}
	\item Small W and Large S
		\begin{itemize}
			\item Personalist dictatorships and dominant party democracies
			\item Incentives to provide rewards to their relatively small winning coalition to stay in power
			\item Rigged voting hence the large selectorate but small winning coalition
			\item Poor government performance
			\item Low levels of wealth, inefficient governance with high levels of corruption and kleptocracy
		\end{itemize}
	\item Small W and Small S
		\begin{itemize}
			\item Monarchic and military dictatorships
			\item Produces middling government performance
		\end{itemize}
\end{itemize}

The basic assumption of selectorate theory is that all political leaders are motivated by the desire to gain office. The competitive nature of politics forces all leaders to behave in this way. With that in mind, government performance is derivative of the ruling power making their selectorate happy. As shown above, personalistic or dominant party regimes will have lacking performance due to a small W that needs to be appeased. Dictatorships tend to focus on private goods to be given to their W as a result of this. For democracys, who have large Ws, public goods are the focus.

\subsection{Loyalty Norm}
The \textbf{Loyalty Norm} extends the idea of keeping the respective Ws happy. The Loyalty Norm is determined by $W/S$ which is effectively the probability that someone in the selectorate is in the winning coalition.

\textbf{Low Probability:}

There is less chance of a member of W defecting as the odds that they could form a new winning coalition is low, hence the loyalty here is high. Strong loyalty norm regimes tend to have a \textbf{greater chance in engaging in corruption and kleptocracy.} The amount to pay W is lower to keep their loyalty.
\textbf{High Probability:}

These is a higher chance of a member of W defecting as the odds that they could form a new winning coalition is high, hence the loyalty here is low. Low loyalty norm regimes tend to have a \textbf{lower chance in engaging in corruption and kleptocracy.} The amount required to pay W is higher to keep their loyalty. 

\begin{center}

We can recognise the probabilities by constructing a simple equation:

% insert equation and example from Clark, Golder and Golder
$R_L = Reward\;for\;Loyalty$ ;
$R_D = Reward\;for\;Defecting$ \\
$P_L = Probablity\;of\;Staying$ ;
$P_D = Probability\;of\;Defecting$ \\

\end{center}

\subsection{Size of the Winning Coalition}
The size of W can affect the eocnomic performance heavily through either investment into public or private goods. Leaders \textbf{always prefer to use private goods to satisfy the winning coalition rather than public goods} as it is easier and more effective. However, \textbf{increasing W should lead to more public good production.} This is simply because there is more people to please and public goods, since they are \textbf{non-rivalrous} \textit{(one person's consumption does not hinder another's ability to consume)} and \textbf{non-excludable} \textit{(one cannot prevent another from accessing this good entirely)}, appease everyone. Private goods, on the other hand, can only entertain a few people and so would not be able to maintain W.

\subsection{Summary of Selectorate Theory}

\newpage
\section{Differences between Strategic and Expressive Voting}


\newpage
\section{Down's Theory of Party Competition}


\newpage
\section{Olson's Theory of Collective Action}


\newpage
\section{Tsbelis' theory of veto-players and agenda setters}


\newpage
\section{Office vs Policy Seeking Theories of Coalitions}


\newpage
\section{Principal-Agent theory of Independent Institutions}


\end{document}