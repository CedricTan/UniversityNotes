\documentclass[12pt, letterpaper]{article}
\usepackage[margin=1in]{geometry}
\usepackage{graphicx}
\usepackage{amsmath}
\usepackage{amssymb}
\usepackage{tikz}

\title{
	{GV100 Intro to PolTheory}\\
	{\large{Professor Katrin Flikschuh}}\\
	{\large{Essay Plan Document}}
}
\author{Cedric Tan}
\date{April 2019}

\begin{document}
\maketitle
\begin{abstract}

This is a construction of some essay plans for MT and LT topics from the Introduction to Political Theory Module. The notes are mine fully and may not be authentic to the lecturer's as they have been modified.

The format of this material is usually recounted slide by slide but some slides may be merged together as the material fits appropriately with one another.
\end{abstract}
\newpage
\tableofcontents
\newpage

\section{Plato}
We will begin with a recap on Plato's main ideas before moving onto essay constructions.
\subsection{Plato and Justice}
Plato's motivation is to find the just city and to figure out what the just city entails. 
\subsubsection{Plato and Thrasymachus}
We begin with the dialogue between Plato and Thrasymachus. Platonic justice is \textbf{unknown in the beginning,} what we do have is a conception of Thrasymachean Justice:
\begin{itemize}
	\item Whatever is right for the stronger
	\item To the advantage of the stronger
\end{itemize}
However, even with Thrasymachus we are unsure about his motivations for justice. Are they \textbf{factual i.e. descriptive} or are they \textbf{normative i.e. what we should strive to do?} He could be advertising both in his theory, which is perfectly acceptable. Despite this, Plato is unsatisfied with Thrasymachus' conclusion. Injustice produces faction and hatred. For Plato, this means that the soul is ultimately unbalanced.

\subsection{Study Questions and Plans}

\newpage
\section{Aristotle}


\newpage
\section{Machiavelli}


\newpage
\section{Thomas Hobbes}


\newpage
\section{John Stuart Mill}


\newpage
\section{Kwame Nkrumah}



\end{document}