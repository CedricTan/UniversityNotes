\documentclass[12pt, letterpaper]{article}
\usepackage[margin=1in]{geometry}
\usepackage{graphicx}
\usepackage{amsmath}
\usepackage{amssymb}
\usepackage{tikz}
\usepackage{hyperref}
\hypersetup{
	colorlinks = true,
	linkcolor = black,
	urlcolor = blue
}

\title{
	{GV100 Intro to PolTheory}\\
	{\large{Professor Katrin Flikschuh}}\\
	{\large{Essay Plan Document}}
}
\author{Cedric Tan}
\date{April 2019}

\begin{document}
\maketitle
\begin{abstract}

This is a construction of some essay plans for MT and LT topics from the Introduction to Political Theory Module. Essay plans are sparse throughout the document and can be found at the end of \textbf{parts} of each thinker. If they are longer, that means more time has been dedicated to them due to particular interest. The notes are mine fully and may not be authentic to the lecturer's as they have been modified.

The format of this material is usually recounted lecture by lecture, which is to say a week by week recount. Material may be merged together as it fits appropriately with one another such as multiple weeks on the same thinker.
\end{abstract}
\newpage
\tableofcontents
\newpage

\section{Plato}
We will begin with a recap on Plato's main ideas before moving onto essay constructions. This section is split into two parts, the first concerning \textbf{Plato and Justice} and the second concerning \textbf{Plato and his Preservation of Justice.}
\subsection{Part 1: Plato and Justice}
Plato's motivation is to find the just city and to figure out what the just city entails. 
\subsubsection{Plato and Thrasymachus}
We begin with the dialogue between Plato and Thrasymachus. Platonic justice is \textbf{unknown in the beginning,} what we do have is a conception of Thrasymachean Justice:
\begin{itemize}
	\item Whatever is right for the stronger
	\item To the advantage of the stronger
\end{itemize}
However, even with Thrasymachus we are unsure about his motivations for justice. Are they \textbf{factual i.e. descriptive} or are they \textbf{normative i.e. what we should strive to do?} He could be advertising both in his theory, which is perfectly acceptable. Despite this, Plato is unsatisfied with Thrasymachus' conclusion. Injustice produces faction and hatred. For Plato, this means that the soul is ultimately unbalanced.

\subsubsection{Characteristics of Justice}
For Plato, there are certain characteristics that must be inherent in justice: 
\begin{enumerate}
	\item There needs to be a desirability of justice so as to make people act in a justifiable manner without the need for coercion
	\item Justice must be a good in itself, this means that it isn't a means to an ends, good or bad, but rather the idea of justice is itself good. The products of justice do not matter in this discussion. However, if justice is a good in itself, it should follow that its products are good too
	\item Justice is also the excellence of the soul which means that the soul is harmonious in a metaphysical sense
	\item Justice also has positive effects on other people, referencing point 2
\end{enumerate}

\subsubsection{Conventional and Platonic Justice}
Thus, moving on from its characteristics, we can argue that types of justice may appear to have these characteristics so we need to ensure that we can distinguish these \textbf{conventional} forms of justice from our \textbf{Platonic} form.
\begin{itemize}
	\item Conventional $\rightarrow$ justice in terms of appearance and practice
	\item Platonic $\rightarrow$ justice as a good in itself in an idealistic form
\end{itemize}
So, the confusion can be illustrated as follows:
\begin{center}
	\noindent\fbox{\begin{minipage}{.9\textwidth}
	Take the characteristics of \textbf{intelligence} and \textbf{courage.} Now, let's compare these characteristics, respectively, to \textbf{slyness} and \textbf{cunning.} Here, we can see that these characteristics have some similar traits and can be conflated at times. Yet, the latter traits are seen to have unjust motivations. This is the concern when attempting to distinguish between \textbf{Conventional} and \textbf{Platonic} forms of justice.
	\end{minipage}}
\end{center}
To add further complication to this matter. There are motives to be just without a harmonious soul, hence, again, on appearance we might look just but the motivations are simply not in the right order. A key exmaple of this would be \textbf{Hobbesian fear of punishment} as you act in a just manner to obtain security rather than \textit{acting for justice itself,} which is what Plato says is the important thing to do.

Hence, it is \textbf{very difficult to distinguish} between the two as, on the surface, they look as if they are the same thing entirely.


\subsection{Part 1: Study Questions and Plans}
Below are study questions and short, planned responses to them.
\subsubsection{Thrasymachus}
\textbf{Question:} How does Thrasymachus defend his claims? Are his arguments convincing?\\\\
We begin by understand Thrasymachus' claims:
\begin{itemize}
	\item Justice is whatever is good for the stronger
	\item Where the power lies is where justice lies $\rightarrow$ governments, democratic, tyrannical and so on, make laws which are specific and to the benefit of their government type, hence their actions are just
	\item The subjects in these governments are to follow these laws and rules to the benefit of their government type, if they act accordingly, they are labelled as just
	\item Justice is obedience to laws
	\item Justice is nothing but the advantage of another
\end{itemize}
\textbf{Arguments}\\
This is, on face value, a very attractive conception of justice. We conform to a system of rules created by people in power and believe in them because they hold us and everyone else accountable to the system whilst making the system itself function. Hence, it would seem convincing, in a rather cynical sense, that we are weak people following the rule of those who dictate and just have the \textbf{belief that what we are doing is considered just.}

This idea is just as plausible today as it was with Thrasymachus. Think about big government in places such as Russia (less so) or North Korea, the subjects in those countries believe their actions are just as long as they conform to the set of rules laid out. Subjects are literally strong-manned into conforming to that system of 'justice'.

Thus, this means that the subjects, from Thrasymachus point of view, are seen to be just while \textbf{the rulers who take advantage of their just behaviour are seen to be unjust.} Thrasymachus goes onto list various examples of where the just are far worse off than the unjust. The first example he uses is the idea of \textbf{contracts} $\rightarrow$ where the unjust person will leave better off than the unjust due to some abuse of the contract that the unjust person is willing to commit that the just person would not.

Further, the idea of \textbf{justice as nothing but the advantage of another} advocates some sort of ethical egoism. This argument, again from a cynical point of view, seems to work out. However, to prescribe egoism is a difficult task and if attacking Thrasymachus from this point of view, it would be easy to pick apart his argument. Let us not take this argument with as much weight to be charitable to his other arguments. \\\\
\textbf{Counter-arguments}\\
For Plato, this seems questionable and concerning for a few reasons:

One of them is because of the fact that even rulers are \textbf{fallible,} which means that they themselves are prone to making mistakes. If subjects, willing to be just, aim to conform to the ruler's specifications at all times, then when rulers incorrectly institute a law that is not in their advantage, we find ourselves in a self-contradicting situation whereby the just subjects, in their attempts to be just, \textbf{are acting to the disadvantage of the stronger,} despite the stronger instituting said laws. This means that \textbf{obedience is occasionally not in the interests of the stronger.} The issue that Plato must face is Thrasymachus' claims that \textbf{rulers who do not act in their own advantage are not effectively ruling.} However, this claim seems to be absurd in the sense that ruling is a continuous activity and if we were to segment it into successes and failures in ruling, it would seem that the ruler has not ruled continuously at all.

Another is Plato's concern with the soul and its excellence. If we are to simply look at justice as the advantage of the stronger, it might not be the case that our soul is harmonious at all. Taking the unjust argument from Thrasymachus, where the unjust man is always better off than the just, we recognise the possibility that under the Thrasymachean system, we are approaching justice in a strictly conventional manner rather than a Platonic manner, hence Plato's dissatisfaction.  \\\\
\textbf{Evaluation}\\
Are Plato's arguments against Thrasymachus satisfactory? To a certain extent, they seem to be. The argument from self-contradiction appears to defeat the Thrasymachus' initial argument and further, the harmonious excellence of the soul which is emphasised in the Platonic doctrine has its merits.

Yet, this still does not account for the skeptic, that justice which surrounds us is purely conventional and that the idealistic conception of justice makes no sense to adopt because of its very idealistic nature. If we take the material perspective, that is, to disregard all idealistic conceptions, we might have to give Thrasymachus a lot more credit than he is given.

Thus the side of the argument you take is entirely dependent on your world-view. If you think that the conventional justice really \textbf{is not justice at all} and that an idealistic justice is the only form of justice that is truly \textit{justified} then you are able to disregard the Thrasymachean argument. Yet, in a very \textbf{practical sense,} it would seem that Thrasymachus appears to tap into our doubts of the very systems in which we live.

\subsubsection{Soul and City, an analogy}
\textbf{Question:} Towards the end of Book 1, Plato has Socrates introduce an analogy between 'justice as a virtue of the soul' and 'justice as a virtue of the well-ordered polis.' Why does he draw this analogy? Is the idea that the just polis is comparable to the just person? Or is the thought, rather, that for a just polis to be possible, we have to have just persons?\\\\
We begin by understanding the two analogies that Socrates puts together. Below are both the \textbf{Soul} and the \textbf{City} with their \textbf{constituent parts.}
\begin{center}
	\begin{tabular}{c|c}
		\textbf{Soul} & \textbf{City}\\
		\hline
		Appetites & Artisans\\
		\hline
		Spirit & Guardians (Auxiliaries)\\
		\hline
		Reason & Philosopher Kings\\
		\hline
	\end{tabular}
\end{center}
Here is a refresher on the roles of each:
\begin{itemize}
	\item Appetites or the Artisans $\rightarrow$ here to meet our daily needs and bodily well-being
	\item Spirit or the Guardians $\rightarrow$ here to give us our intuitive guidance and establish our virtues as people or as a city
	\item Reason or the Philosopher Kings $\rightarrow$ here to govern us
\end{itemize}
\textbf{Think of society as a ship at sea:}
\begin{itemize}
	\item It is dependent on who is the captain and what roles each individual has on the boat
	\item The true helmsman wants to sail smoothly and safely but still towards a set direction
	\item This can be achieved in a \textbf{just} manner
	\item Necessarily, this involves all components to be working \textbf{harmoniously}
\end{itemize}
% Underfull hbox - fix later
\textbf{Just Polis as comparable to the Just Person}\\
Recall that justice comes from the harmony of components within an entity. Plato discussed this idea with justice in an individual. This is seen through the balancing act of all components of the soul whereby if each individual component is doing its part, through the guidance of reason, which is a part in itself, we will be operating as a just person. This is because we have some necessities which are fulfilled by our appetites such as fulfilling our need to sleep or eat whilst also establishing our virtues and guidance on where to go with our spirit.

This is comparable to the just polis in a sense due to the idea of how a polis is constructed in a similar manner. Roles are given within society, that much is given. When you fulfil these roles and everyone else does the same, we will be operating as a just city due to the fact that there is a certain harmony within the city itself. This harmony subsequently means that the soul, in a sense, of the city is inevitably excellent. \\\\
\textbf{Just Polis requires Just Persons}
The second possibility of this analogy is a hint towards a political ideology of instituting a just city by creating just people. This makes intuitive sense as the requirement of harmony, to create a just city, flows easily from the individuals in said city being just. If the purpose of an individual is to fulfil a particular role in society, their spirit and reason will lead them there \textbf{on the assumption that the individual is already just.} This will lead to all roles, the artisans, guardians and philosopher kings, being fulfilled and operating at the best level they can be operating at.

Taking these factors into account, which we can label as a bottom up approach, the just individual will lead into the just city. This argument is strong because it does not leave out any individual in the polis whatsoever. It appears that it is a necessary requirement for all individuals in the polis to be just for the polis itself to be just. \\\\
\textbf{Evaluation on analogies}\\
The \textbf{first argument,} a comparison of the soul to the city, appears to be approachable in a strictly functionalist sense. If all constituent parts of the soul, like all constituent parts of the city, are working strictly by their function, we can see that the argument makes sense. From that point of view, the analogy makes some sort of sense.

Yet, it would be hard-pressed to say that the constitution of one person's soul is the same as the constitution of a city. The plethora of moving parts that exist in an operating polis seems to be incomparable or reducible to the three components of the soul. Within the Artisan class, for example, a great number of artisan crafts exist which might all have different idealistic purposes and so on. Hence, from a practical sense, the idea seems absurd to compare the two (Ferrari 2003). Further, could they be mutually exclusive? If I aimed to maintain a harmonious soul, could I still fulfil my function? Perhaps not, and if that is the case, then it'd be unlikely for a just individual to live in a just polis.

Does this take away from the strength of the analogy though? If the practicality is such that it is not only function that is required to be fulfilled, it might diminish Plato's argument. However, the simplistic nature of the analogy makes the argument approachable and sensible in a clarified manner. It makes sense that for a city to be just, its internal gears all need to be operating smoothly in some manner.

The \textbf{second argument,} just individuals leading into a just city, is convincing to a large extent because of this bottom-up approach that I had touched on. By necessarily requiring all individuals to be just to constitute a just city, we nail down the functionality argument along with the harmonious and excellence requirement from the get go. Thus, it would make a lot of sense to start a city, with the aim for it to be just, if we have just individuals founding it.

Yet, we have a \textbf{chicken and egg problem} because although we would love to nail down these requirements initially, it seems idealistic to believe all individuals will be just before founding the city. It would seem more plausible to have a just city, more or less, that trains people to be just. But then again, who would constitute that just city? Perhaps we can reduce the requirements for the creation of a just city by having \textbf{some just individuals} instead of them all being just. That's a point of contention that might need to be studied further.

\subsubsection{Arguments to Remember}
Below are some arguments to remember as reasons to be just or reasons as to why justice is preferred to injustice:
\begin{enumerate}
	\item Argument of Craft
	\item Argument from Ignorance
	\item Argument from Happiness
\end{enumerate}
\textbf{Argument of Craft}\\
The basic premise of this argument is related to ruling for the ruled or ruling for some other gain. Plato draws this analogy: there is a distinction to be made between \textbf{shepherding} and \textbf{shepherding for money.} The function of the shepherd is to shepherd the sheep and that is intrinsic to the activity, the profitability that arises from shepherding is not part of the function but just a (positive) side-effect of it.

Hence, every activity has some sort of intrinsic function that is tied to it which is found in its own objective nature. This function is not set by the agent but found in Plato's forms. Hence, \textbf{ruling is misunderstood if they do not rule for the benefit of the ruled.} Plato believes that this is the \textbf{objective function} of ruling as an activity.\\\\
\textbf{Argument from Ignorance}\\
This argument discusses how the unjust are not better off than the just. This is because a just person \textbf{wants to be better than an unjust person} but \textbf{does not want to be better than another just person.} Meanwhile, an unjust person \textbf{wants to be better than all other persons.}

Plato believes that no-one who properly understands a given practice wants to exceed another who also properly understands it. Think of a baker, if they have perfected the craft to the point where there is nothing else to learn then why would they attempt to outdo another who is at the same point as them? Hence, the excesses in the practice of justice shows ignorance of what practising justice means. These excesses are seen through those who are unjust.\\\\
\textbf{Argument from Happiness}\\
The unjust are not happier, Plato argues, as they have factionalised souls. A soul which has factions is unhappy from a Platonic point of view because of this lack of harmony that is require for a soul to be excellent and subsequently happy.

\subsection{Part 1: Appendix}
Here are some additional resources to aid revision:
\begin{itemize}
	\item \href{https://plato.stanford.edu/entries/plato-ethics-politics/#PoliPartOneIdeaCons}{Plato's Ethics and Politics in the Republic}
\end{itemize}

\subsection{Part 2: Preservation of Justice}
This part will begin to explore the institutions and systems that are required to create a just city. Below are some key ideas to remember when discussing Plato's construction and the subsequent preservation of justice:
\begin{itemize}
	\item Craft Analogy $\rightarrow$ stating that each individual has a soul which is assigned to a specific purpose
	\item Censorship $\rightarrow$ to maintain the craft analogy
	\item Abolishing private property $\rightarrow$ reducing faction and promoting cooperation for the guardians and philosopher kings especially
	\item Rigged Lotteries $\rightarrow$ placing blame on chance rather than regulation or legislation
\end{itemize}
From these ideas, one can begin to see an almost tyrannical twist to Plato's thinking.

\subsubsection{Craft Analogy}
This section will focus specifically on Plato's craft analogy. Again we revisit the three types of citizen that can be found in the just polis, \textbf{Kallipolis,} and Plato's assignment of the value of their soul:
\begin{center}
	\begin{tabular}{c|c}
		\textbf{Function} & \textbf{Value of their Soul}\\
		\hline
		Artisan & Bronze\\
		\hline
		Guardian & Silver\\
		\hline
		Philosopher King & Gold\\
		\hline
	\end{tabular}
\end{center}
Hence, when you are born, your soul is seen to be of one particular type of value and that part of your nature is immutable and objective. However, having parents of a particular soul type \textbf{does not exclude the possibility of giving birth to a different soul type.} Artisans can give birth to potential philosopher kings whilst philosopher kings can give birth to potential artisans.

The craft analogy is adopted to ensure that a role is assigned to each individual and each individual is properly trained to fulfil that role. Again, recognise that this is important for the harmonious excellence that a city requires for it to be just.

\subsubsection{Censorship}
Plato discusses various forms of censorship throughout \textbf{\textit{The Republic.}} Some common examples are listed below:
\begin{itemize}
	\item Poetry is limited so as to not provide stories that appear to be fake or dull the mind
	\item Music is limited so as to not provide sounds that dull the senses and ease individuals into a sense of complacency
	\item Authors and Poets of fiction can also be expunged if the need is felt (Books 3 and 8)
\end{itemize}
Again, censorship is here to ensure that all individuals maintain a strict adherence to the roles that they are given through the craft analogy. Hence, censorship is used to maintain the analogy from the educational stage such that this idea of purpose and adherence is driven into each individual's system. This is to ensure the excellence of a just city that Plato desires to be constantly maintained.

\subsubsection{Private Property}
Plato removes private property within the polis when training the philosopher kings and guardian class. Again, the premise of this removal is to provide a preemptive measure to faction that might occur. The premise of private property is that it defines property as excludable and rivalrous, if we take this into account, then when private property is created and taken by individuals, the contest for private property may, itself, lead to faction. Hence, to reduce disagreement or avoid it entirely and to maintain the concept of harmony and excellence, Plato takes the whole concept of private property away.

\subsubsection{Rigged Lotteries}
Another form of maintaining harmony is to ensure that people believe all that happens is up to chance. Hence, the use of rigged lotteries to dictate the allocation of resources and allocation of titles and so on is a disguise for the reality of the rulers controlling all facets of society.

\subsubsection{Why Reason?}
From the arguments above, we get the sense that the use of reason is what dictates a lot of these motives to be harmonious and strive for excellence and justice. Although it might seem intuitive for reason to rule, simply because it strives to take the best course of action, I will list below some of the reasons as to why we should not allow \textbf{appetite} and even \textbf{spirit} to \textit{take charge.}\\\\
\textbf{Appetite}\\
The issues with appetite involve always looking for more than what they already have. The fact that the appetitive part, without reason and spirit to guide it, is always hungry shows a \textbf{lack of control} and perhaps a desire to work towards \textbf{personal gain.} This is compounded by its instinctive nature which involves \textbf{no planning ahead} with it also being \textbf{heavily reactionary.} Thus, the appetitive part of the soul suffers from a short-termism and a \textbf{prioritisation of instant gratification.} Here are some examples of appetite in power:
\begin{itemize}
	\item Zimbabwe's Mugabe and appetite for power
	\item Malaysia's Najib and appetite for gain leading to corruption and kleptomancy
\end{itemize}
\textbf{Spirit}\\
The issues with spirit might not be as intuitive since the spirited part of our souls provide us with virtues and values and an instinctive guidance. However, it is the case that sometimes these values and virtues can be taken too far. If there is no restraint put on spirits worldview by reason, we can see the values that, taken too far, to be dangerous. Further, \textbf{spirit is restless} and has no limitations if taking charge, it does not care for appetite or reason and only for its \textbf{ideology.} Here are some examples of ideology that has been taken too far:
\begin{itemize}
	\item Hitler and Nazism
	\item Mao and his yellow army
	\item Stalin and his Soviet dream
\end{itemize}

\subsection{Part 2: Study Questions and Plans}
Below are study questions and short, planned responses to them:

\subsubsection{Classes of Citizen}
\textbf{Question:} Why does Plato distinguish between 3 classes of citizens?\\\\
The main argument to approach this question with is the argument from function. As we have described above, adherence to a function and performing the best you can at that function is what leads to a well-ordered polis. This is all in effort to maintain the harmonious soul and strive for the excellence of a just city. Below are further reasons why:
\begin{itemize}
	\item It happens to be that certain people are just born with a certain skillset that they cannot change
	\item To be unauthentic to that skillset would not be harmonious in the soul
	\item To be unauthentic would also mean that you are not fulfilling your function for the just city
	\item This function and conforming to function is a necessary condition for a just city which is related to the craft analogy
	\item Hence the distinction of classes of citizen provides a necessary structural component to Plato's city
	\item by creating this organised structure, the just city is able to operate without concessions or great faults meaning that there is more room to strive for the excellence that it pursues
\end{itemize}

\subsubsection{Philosophers and Ruling}
\textbf{Question:} Should philosophers be political rulers?\\
\textbf{n.b. This could be a thematic question and can take different forms but it is likely to be touched upon in the exam.}\\\\
First, a question must be asked on the context in which we apply this style of ruling. Plato's context would have been similar to a city state but now we have governments that encompass a more more unhomogenized group of peoples that might be more difficult to control than the city state.

Yet, a question we can also ask is \textbf{if the context even matters for philosopher kings?} Assuming that they have some sort of absolute knowledge, we can recognise their ability to provide us with a strict direction that is possibly independent of governments we have today.\\\\
\textbf{Arguments}\\
One argument for the philosopher king is based on their knowledge and, supposedly, superior understanding of political structures and frameworks. If the faith we put into their understanding is valid, and their knowledge is valid, they would be in the best place to direct how we construct our political systems. Plato's idea of the craft analogy and certain forms of censorship along with the dissolution of private property shows a, if radical, example of how a philosopher king would go about de-constructing our state and then subsequently rebuilding it for a particular purpose. This is one potential route of argument for the philosopher king.

Another is the supposedly \textbf{unbiased perspective} philosopher kings have on ruling. Due to their level of reason exceeding anyone elses, they would be in the best position to make decisions about their citizen body. Their unbiased perspective would also be implementing a just system for all parties, including themselves. If they were perfectly rational, as Plato would put it, the system, barring some major exogenous shocks, would operate smoothly. This is, perhaps, an idealistic argument for the philosopher king case.

In addition to the previous point, their level of education would usually exceed the average politician. This can be a strong case for the philosopher king, especially in our current context (April 2019), since there would be more deliberate action rather than reaction that might be put through in government. If they are in a position of power, these philosopher kings would not be playing a simple politician's game of \textbf{whoever shouts the loudest is seen as correct} but rather constructing a government that strives for the goal of excellence and justice in Platonic terms.\\\\
\textbf{Counter-arguments}\\
Theory is not practice and this shows in various contexts whereby what you think can be applied, in a structural way, cannot be applied whatsoever when you really have a practical attempt at it. Michael Ignatieff (2014), a Canadian philosopher and one-time politician, is a living example of this type of failure. Although he does mention that a certain \textbf{hubris} drove him to attempt a political campaign, he recognises that in the end, his vast knowledge of political philosophy does not apply whatsoever to the real world of politics. This is because the individuals you are catering towards are not as rational as you think and the ideas that you attempt to espouse do not come across as very valid in their eyes. Individuals are also selfish, not caring for the excellence of the polis as a whole but rather, caring for themselves first. \textbf{These are just a few examples of practicality clashing with a theoretical mindset.} This does not mean, however, that we should abandon theory entirely, just that adopting a purely rational and theoretical approach would not be so applicable in reality.

Further, a simple Machiavellian counter-argument can really augment the strength of Ignatieff's experience. As Machiavelli describes in \textbf{\textit{The Prince,}} the reality of politics is heavily decided by the force of \textbf{fortuna} which, boiled down, is simply just luck. Although you can attempt to control Fortuna with some sort of \textbf{virtú,} this set of behaviours is definitely not so related to the theoretical knowledge of Plato's Philosopher Kings. Philosophers may simply not know how to adapt to fortuna and their frameworks, which can sometimes be strict and immutable, might not be fit for the randomness of reality that Machiavelli emphasises. Although there is opportunity, politically, the Philosopher King might not make the best out of it.

Moreover, continuing the Machiavellian line of thought, \textbf{politics is more about experience rather than pure thought.} Both in a practical and perhaps intuitive sense, this makes sense as politics, as in the action of making or doing politics, is a craft and not a purely academic subject. Crafts, in this sense, require a lot of experience to be perfected and one cannot simply think their way there. Although this does not mean that Plato's Philosopher Kings are without experience, to draw that conclusion so quickly would be uncharitable and brash. Yet, the starting point of creating the just city and subsequently ruling it from a purely rational perspective would, from experience as Ignatieff puts it, lead to quite the disaster.\\\\
\textbf{Evaluation}\\
We have defined two sides of the spectrum and they have been two extreme cases:
\begin{itemize}
	\item \textbf{The Philosopher King} without any practical experience
	\item \textbf{The Real Politician} with pure experience and little knowledge
\end{itemize}
To conclude that either one is the solution would be, for lack of a better word, idiotic. Thus, framing the Philosopher King as such would not be charitable in argument either. Hence, the question is raised: could it be the case that a Philosopher King, filled with practical experience, get the best of both worlds? Here, we would be adopting a certain flexibility in the way we do politics as we are not always conforming to theoretical structures or striving for a fully idealistic conception of Kallipolis. Rather, the ruler has to be adaptive to certain situations cast down by \textbf{fortuna} whilst still realising some sort of \textit{just goal.}

This is difficult as ruling now means knowing how to weigh the plethora  of opinions and interests that exist in a society. To state that we have a Philosopher King or Philosopher Kings in charge means that one value system is instated. The issue with this is \textbf{how we judge which value system is the best one.} Although we can vouch for Plato, the next philosopher might not and their opinion might just be as valid.

It is a hard case to make for absolute justice and a fix-all system that might come from the Philosopher King. If it is truly a system that is fix-all, then it might be a case to make but that concept is, in itself, idealistic.

\subsubsection{Plausibility of Ideal Justice}
\textbf{Question:} is the ideal of justice plausible even if it is not realised in a polis/practicality?\\\\
This is a difficult question because for a concept to be valid we can see that there is a contention between a metaphysical validity versus a  tangible validity. It is hard to know whether an ideal concept is valid if we cannot see it in a tangible form as we are unable to judge its \textit{idealness.}

However, we cannot invalidate the argument for its idealness either because its in a realm of forms that cannot be simply dismissed. The thought of it exists, that much is true, hence it might or might not be an ideal $\rightarrow$ there could be many configurations of the ideal as well.

Thus, we can argue this:
\begin{itemize}
	\item We can agree that a concept's validity can exist without an actual manifestation
	\item However, it is hard to judge this concept without its manifestation
	\item If we do have a manifestation, we do not know if it is fully authentic to its concept either
	\item What we can conclude is that we are uncertain of the validity of the concept but we cannot dismiss it
	\item \textbf{Ultimately, it is plausible that an ideal for justice can exist without its manifestation, yet we cannot simply state that one conception of this ideal form is truly correct.} The existence of other multiple forms espoused by other philosophers complicates the issue entirely.
\end{itemize}

\subsubsection{Justice and Other Virtues}
\textbf{Question:} What is the function of justice in relation to other virtues?\\\\
We will begin with an understanding of Platonic virtues. Below are some standard ones that are preached by Plato. These can be related to the cardinal virtues:
\begin{itemize}
	\item \textbf{Courage:} associated with the guardians, this virtue is a mixture of belief and steadfastness of character
	\item \textbf{Wisdom:} exclusive to the rulers, this virtue is an appraisal of intellect
	\item \textbf{Temperance:} belief and a support for order with a conviction for harmony to avoid chaos
	\item \textbf{Justice:} close to moderation but is a virtue found when all other virtues are in harmony, it is found as part of 'doing your own thing', an authenticity to oneself and one's purpose
\end{itemize}
There are a few possible routes to see Justice in relation to the other virtues. \textbf{One} is to recognise justice as a regulator, keeping all other pieces of the puzzle in check. \textbf{The other} is to see justice as something that comes out of the harmony and excellence of all the other virtues in play.

These two possibilities give two wholly different arguments as we recognise again the top-down and bottom-up approaches. Hence, the question could be seen as a paradoxical one as we can question which set of virtues comes first. Does justice come as a result of harmony in other parts or do we only get harmony if we are aiming to be just?

For this question, I am not sure if there is a right or wrong answer but it is likely that the soul needs to be constituted in a well ordered manner first to achieve excellence and I am of that opinion. Perhaps we can recognise the virtue of justice as \textbf{a virtue of maintenance} whereby justice, known as doing ones own, works to ensure that we are continually striving for excellence once we have initially obtained it.

\subsection{Part 2: Appendix}
Here are some additional resources to aid revision:
\begin{itemize}
	\item \href{https://plato.stanford.edu/entries/plato-ethics/#MidPerJusOthVir}{Plato's Ethics: An Overview}
\end{itemize}

\newpage
\section{Aristotle}


\newpage
\section{Machiavelli}


\newpage
\section{Thomas Hobbes}
We will begin with a discussion on Hobbes' philosophical project and his methodology before arriving at his solution through the absolute sovereign imposed upon the state.

\subsection{Part 1: Hobbes' Philosophical Project}
Here is a brief introduction to his project:
\begin{itemize}
	\item Hobbes sought to discover rational principles for the construction of a civil polity that would not be subject to destruction from within e.g. civil war and civil disobedience
	\item He believe that any government would be better than civil war
	\item All governments except absolute ones are prone to dissolution
	\item Stability requires a refrain from actions that undermine the regime
	\item He aimed to demonstrate the reciprocal relationship between political obedience and peace
\end{itemize}

\subsubsection{Thought and Methodology}
Here is a summary of his thought and methodology:
\begin{itemize}
	\item Physicalist notions with no Platonic duality with an intense use for empirical definitions in his method
	\item Based our actions on \textbf{Appetites} and \textbf{Aversions} separated into:
		\begin{itemize}
			\item \textbf{Vital action:} continued without interruption and with no help of imagination
			\item \textbf{Voluntary:} to move in a manner fancied in our minds
		\end{itemize}
		\textbf{Small beginnings to motion are commonly called endeavour}
	\item Appetites means your attraction to goods
	\item Aversions means your distancing from evils
	\item Deliberation is simple the last appetite or aversion in your will: everything is determined in that sense and to will is to follow your last appetite or aversion i.e. \textbf{not entirely free will for Hobbes}
	\item Appetites and aversions are imperative: we do not necessarily will it, in the sense of the verb, as it is deterministic
	\item Appetites and aversions pave way for different ends:
		\begin{itemize}
			\item Desire for security allows for a compulsion to obey a common power
			\item Fear of death paves way for a compulsion to obey a common power
			\item Desire for honour, glory and power paves way for conflict due to individuals conflicting over interests
		\end{itemize}
\end{itemize}

\subsubsection{Hobbes on Power}
Taken from Chapter 10 of the Leviathan:
\begin{itemize}
	\item Power of man is his present means to obtain some future apparent good and is either original (from oneself) or instrumental (an aid to oneself)
	\item Natural power is the eminence of the faculties of body or mind
	\item Forms of power include:
		\begin{itemize}
			\item Faction
			\item Riches joined with liberality i.e. ability to do what you want with your riches
			\item Reputation of power and other forms of reputation
			\item Eloquence
		\end{itemize}
	\item Forms of honour include:
		\begin{itemize}
			\item Praise and its various forms
			\item Trust and belief
			\item Harkening a man's counsel as said man would be honourable
		\end{itemize}
\end{itemize}
Hence, from this, we can see that power is a central concept to the Hobbesian methodology as it is necessarily present in the interactions we have with one another.

\subsubsection{Hobbes' State of Nature}
\textbf{The State of Nature (SoN)} is a hypothetical created by Hobbes as a contra case to government. We can define it as \textbf{a condition of society without a government where each person is themselves their own judge, jury and executioner.}
To summarise further:
\begin{itemize}
	\item This is a state of perfectly private judgement with no agency to arbitrate or enforce judgement decisions
	\item There is a lack of agencies to enforce anything
	\item Everyone is a constant threat to everyone as:
		\begin{itemize}
			\item Men are equal in the SoN: even the weakest has the capacity to kill the strongest
			\item Desires for the same things lead to conflicts of interest
			\item There is a desperate requirement of some sort of power to protect oneself and advance one's own interests			
			\item For certain peoples, a desire for glory and power leads to a motivation for conflict
			\item Every individual in the SoN has the right to preserve oneself $\rightarrow$ this is practically a right to do all things given the correct context
		\end{itemize}
	\item To put it simply, it is a 'not a place for industry as the fruit thereof is uncertain'
\end{itemize}
From these classifications we are given by Hobbes above, he concludes, in a further note, that the SoN is a \textbf{State of War (SoW).} This is because of the divisive struggle between individuals competing for their own interests with no agencies to mediate between them. Flikschuh (2018) gives three main principles for quarrel which were touched on before:
\begin{enumerate}
	\item \textbf{Competition:} scarce resources with conflicting interests hence we cannot satisfy everyone's needs
	\item \textbf{Diffidence:} a lack of credibility hence a constant distrust between individuals
	\item \textbf{Glory:} some seek power and have an enjoyment in conflict hence providing a self motivation to go to war
\end{enumerate}
Thus, the sad reality is that \textbf{life is nasty, brutish and short in the SoN.}

\subsubsection{Laws of Nature}
There are two critical laws that Hobbes proposes which are inherent or should be inherent in our dispositions:
\begin{enumerate}
	\item Men are commanded to endeavour peace
	\item Men would lay down rights for peace as long as they can do it safely $\rightarrow$ this would come from a mutual covenanting to submit to the authority of some government
\end{enumerate}
The laws are related to human's common appetites and aversions. The question of whether or not they are determined by reason or if they simply exist in nature is open to interpretation but should not diminish their in argument. These are also called articles of peace: one should follow them to attain a peaceful arrangement outside the SoW.

\subsubsection{Establishing the Sovereign}
There exists two methods"
\begin{enumerate}
	\item When people mutually covenant each to one another to obey a common authority, this is called \textbf{sovereignty by institution}
	\item When threatened to covenant, they have established \textbf{sovereignty by acquisition}
\end{enumerate}
Both these methods are valid ways of creating sovereignty as they are willed by individuals:
\begin{itemize}
	\item By institution: willed by all individuals to consent into the contract without the need for force, this is all an appetite for security more-so than an aversion from war but both play a part
	\item By acquisition: is legitimate too albeit intuitively not. This is because the last will or aversion, in this case the likely aversion from death due to force, is willed in favour of contracting rather than dying, hence for Hobbes, it is valid even if at first it is against your will $\rightarrow$ effectively the last thing you willed was for the sovereign hence the validity
\end{itemize}
The underlying notion of fear, of others or of the ruler, is what drives individuals to contract with one another. Political legitimacy then comes from not how the government is formed but rather how it protects its contracting people. \textbf{Political obligation ends when this protection by the sovereign ends.}

\subsubsection{Absolutism}
For Hobbes, a government must be absolute for it to be effective. Hence absolute authority is a necessary condition for successful government. Here are two conditions for this absolute authority which we call \textbf{essential rights of sovereignty}
\begin{itemize}
	\item Powers must not be divided nor limited
	\item Powers include: legislation, adjudication, enforcement, taxation, war-making and control of normative discipline
\end{itemize}
Imposition of limits leaves room for interpretation of the overstepping of these limits which invites further trouble. Disagreement between parties for and against may arise and, in the worse cases, could lead to some sort of civil war. Hence we can avoid this by ensuring that no grey areas exist through absolute power.

\subsubsection{Interpretations of the SoN}
Lloyd and Sreedhar (2018) present questions around the idea of the State of Nature. Here are a few examples:\\\\
\textbf{Individuals and Natural Associations}
\begin{itemize}
	\item What about family and friends in the SoN? Where do they stand with respect to competition and diffidence?
	\item Are natural associations, such as relationships through procreation, included in the Hobbesian analysis?
	\item Do these natural associations even exist for Hobbes?
	\item If they do not, how plausible is the SoN?
	\item To what extent is the Hobbesian analysis constrained to individuals who are isolated?\\
\end{itemize}
\textbf{Rationality and War}
For Hobbes, the crazy conclusion that no other philosopher really makes is that \textbf{it is rational to go to war.} However, Lloyd and Sreedhar question this rationality as follows:
\begin{itemize}
	\item Why would communal life be prone to disaster? Is this an assumption that all actors will be rational in a Hobbesian manner? $\rightarrow$ An example of this failure in rationality is prioritising a short term gain over a long term one.
	\item Is it also possible that we can be trapped in situations where it is rational to act in a sub-optimal way? $\rightarrow$ Game theory and cooperation providing better outcomes for all parties is an example of this
\end{itemize}

\subsection{Part 1: Study Questions and Plans}


\subsection{Part 1: Appendix}

\subsection{Part 2: Hobbes and Dilemmas}

\subsection{Part 2: Study Questions and Plans}

\subsection{Part 2: Appendix}

\newpage
\section{John Stuart Mill}


\newpage
\section{Kwame Nkrumah}



\end{document}