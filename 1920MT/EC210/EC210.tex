\documentclass[12pt, letterpaper]{article}
\usepackage[margin=1in]{geometry}
\usepackage{graphicx}
\usepackage{amsmath}
\usepackage{amssymb}
\usepackage{tikz}
\usepackage{hyperref}
\hypersetup{
	colorlinks = true,
	linkcolor = black,
	urlcolor = blue
}

\setlength{\headheight}{15pt}

\usepackage{fancyhdr}
\pagestyle{fancy}
\fancyhf{}
\rhead{\leftmark}
\lhead{EC210}

\title{
	{EC210 Macroeconomic Principles MT}\\
	{\large{Dr Kevin Sheedy}}\\
	{\large{Lecture Notes}}
}
\author{Cedric Tan}
\date{September 2019}
\begin{document}
\maketitle
{\small
  \noindent\textbf{Concept}\\
  Concept. \hspace*{\fill}[1]

  \vspace{10pt}
  \noindent\textbf{Concept}\\
  Concept.\hspace*{\fill}[2]

\newpage
\tableofcontents
\newpage

\section{Measuring GDP}
A few questions surrounding Gross Domestic Product (GDP):
\begin{itemize}
	\item What is national income and what are its components?
	\item How is it measured?
	\item How should it be interpreted?
\end{itemize}
To answer these questions, we take the example island economy with a description of its various components outlined below:
\begin{itemize}
	\item Markets and Goods: fish (\$2); restaurant meals (\$10); boats (\$1,000)
	\item Firms: produce goods by hiring labour (island limited to 20 workers), renting land (totalling 4 units), and using their capital (1 boat)
	\item Households: buy goods, work (wage per worker is \$100) and rent out land (\$50 per unit)
	\item Government: tax and provide public services
	\item There is also trade with the rest of the world
\end{itemize}
So from the list above, we can try to create a set of national accounts for the economy.

\section{National Income and Product Accounts}
The NIPA is produced by the Bureau of Economic Analysis (BEA), providing information on the value and composition of output produced in the United States during a given period and on the types and uses of the income generated by that production. NIPA begins by considering the transactions that occur in a simple economy in order to introduce the economic concepts that underlie the NIPAs. We also describe the NIPA sectors for which economic activity is emasured and the use of T-accounts to illustrate economic flows.

\subsection{Conceptual Basis of the Accounts}
This section covers concepts used within the NIPA framework.
\subsubsection{The circular flow}
The fundamental idea of a working economy can be illustrated between just two parties: individuals and businesses. Between these two parties, we can illustrate the flow of income such that:
\begin{itemize}
	\item Individuals provide labour and in return businesses provide goods and services or,
	\item Individuals are given income from their labour and in return, they expend their income on the business.
\end{itemize}
This basic form of exchange is called the circular flow of income and is self-contained within this example.

\subsubsection{Economic concepts in the NIPAs}
The circular flow illustrates the interdependence of the flows or activities that occur in the economy, such as the production of goods and services and the income generated from that production. The circular flow also illustrate the equality between the income earned from production and the value of goods and services produced.

\vspace{10pt}
\noindent However, we did mention that the actual economy is much more complicated. What is not included is also the state, local governments and the rest of the world (trade). Other parts are also not discussed in this simply flow, such as investment in capital (PP\&E), flows of financial capital (stocks, bonds, deposits) and the contribution of these flows to the accumulation of fixed assets.

\subsubsection{Output}
The featured measure of output in the NIPAs is GDP. GDP measures the market value of goods, services and structure produced by the nation's economy in a particular period. The following points are things to keep in mind when considering the output of the economy.

\vspace{10pt}
\noindent\textbf{GDP includes market production and some nonmarket production}\\
This means that GDP accounts for things sold in the market, your normal goods and services any consumer could purchase, and of things that are not sold in the market such as defence services provided by the government, education provided by the local government or the state and so on. Data aims to include this to gain a true measure

\vspace{10pt}
\noindent\textbf{Whenever Possible, GDP is valued at Market Prices}\\
The NIPAs value market goods and services using prices set by the market. This means that there is a common unit of measurement that facilitates the comparison of goods and services that make up economic activity. Using market values also facilitates the analysis of the impacts on the economy of events such as the implementation of government programs or the occurrence of natural disasters e.g. boost in education spending or destruction of infrastructure valued at $x$.

In some cases, market prices do not fully reflect the \textbf{value} of a good or service and may include some types of services where an actual exchange has not occurred. In these cases, the value is \textit{imputed} from similar market transactions. Imputations measure the value of goods and services that are not fully reflected in market prices. Examples of imputed measure in the NIPAs include the value of compensation-in-kind (such as meals provided by employers).

\vspace{10pt}
\noindent\textbf{filler}\\

\end{document}