\documentclass[12pt, letterpaper]{article}
\usepackage[margin=1in]{geometry}
\usepackage{graphicx}
\usepackage{amsmath}
\usepackage{amssymb}
\usepackage{tikz}
\usepackage{hyperref}
\hypersetup{
	colorlinks = true,
	linkcolor = black,
	urlcolor = blue
}

\setlength{\headheight}{15pt}

\usepackage{fancyhdr}
\pagestyle{fancy}
\fancyhf{}
\rhead{\leftmark}
\lhead{EC210}

\title{
	{EC210 Macroeconomic Principles MT}\\
	{\large{Dr Kevin Sheedy}}\\
	{\large{Lecture Notes}}
}
\author{Cedric Tan}
\date{September 2019}
\begin{document}
\maketitle
{\small
  \noindent\textbf{Concept}\\
  Concept. \hspace*{\fill}[1]

  \vspace{10pt}
  \noindent\textbf{Concept}\\
  Concept.\hspace*{\fill}[2]

\newpage
\tableofcontents
\newpage

\section{Measuring GDP}
A few questions surrounding Gross Domestic Product (GDP):
\begin{itemize}
	\item What is national income and what are its components?
	\item How is it measured?
	\item How should it be interpreted?
\end{itemize}
To answer these questions, we take the example island economy with a description of its various components outlined below:
\begin{itemize}
	\item Markets and Goods: fish (\$2); restaurant meals (\$10); boats (\$1,000)
	\item Firms: produce goods by hiring labour (island limited to 20 workers), renting land (totalling 4 units), and using their capital (1 boat)
	\item Households: buy goods, work (wage per worker is \$100) and rent out land (\$50 per unit)
	\item Government: tax and provide public services
	\item There is also trade with the rest of the world
\end{itemize}
So from the list above, we can try to create a set of national accounts for the economy.

\subsection{National Accounts}
Listed below are the national accounts created from the island economy above.

\subsubsection{Corporate Sector: Fish Producers}

\begin{center}
	\begin{tabular}{|c|c|c|c|}
	\hline
	& Item & Value & units\\
	\hline
	1 & Sales of Fish (Revenues) & \$1,800 & 900\\
	\hline
	2 & Net addition to inventory & \$200 & 100\\
	\hline
	3 & Production (1+2) & \$2,000 & 1,000 \\
	\hline
	& & &\\
	\hline
	4 & Wages paid to workers & \$1,000 & 10\\
	\hline
	5 & Interest on debt & \$200 & \\
	\hline
	& & &\\
	\hline
	6 & Gross Operating Profits [=(3)-(4)-(5)] & \$800 & \\
	\hline
	7 & Net Borrowing & \$500 & \\
	\hline
	8 & Investment (New Boat) & \$1,000 & 1 \\
	\hline
	9 & Dividends [=(6)+(7)-(2)-(8)] & \$100 & \\
	\hline
	\end{tabular}
\end{center}

\subsubsection{Corporate Sector: Restaurants}

\begin{center}
	\begin{tabular}{|c|c|c|c|}
	\hline
	& Item & Value & units\\
	\hline
	1 & Sales of Meals (Revenues) & \$2,500 & 250\\
	\hline
	&&&\\
	\hline
	2 & Purchase of intermediate goods (fish) & \$1,000 & 500\\
	\hline
	3 & Wages paid to workers & \$500 & 5 \\
	\hline
	4 & Rent of land & \$200 & 4\\
	\hline
	5 & Tax on sales & \$250 & \\
	\hline
	& & &\\
	\hline
	6 & Gross Operating Profits [=(1)-(2)-(3)-(4)-(5)] & \$550 & \\
	\hline
	7 & Retained Earnings & \$100 & \\
	\hline
	8 & Dividends [=(6)-(7)] & \$450 & \\
	\hline
	\end{tabular}
\end{center}

\subsubsection{Government Sector}

\begin{center}
	\begin{tabular}{|c|c|c|c|}
	\hline
	& Item & Value & units\\
	\hline
	1 & Income tax on wages (direct) & \$200 & \\
	\hline
	2 & Sales tax (indirect) & \$250 & \\
	\hline
	3 & Tax Revenue (1+2) & \$450 & \\
	\hline
	& & &\\
	\hline
	4 & Wages of government employees & \$500 & 5\\
	\hline
	5 & Interest on debt & \$150 & \\
	\hline
	& & &\\
	\hline
	6 & Budget deficit [=(4)+(5)-(3)] & \$200 & \\
	\hline
	\end{tabular}
\end{center}

\subsubsection{Household Sector: Income}


\subsubsection{Household Sector: Expenditure}


\subsubsection{International Trade and Income}

\subsection{GDP}
Gross Domestic Product is the most common measure of the size of an economy. The definition is as follows: \textit{the value of all goods and services produced in a country in a period of time}. A few things about the definition:
\begin{itemize}
	\item It is a gross measure: that means it is the opposite of 'net' i.e. it does not account for depreciation
	\item It is domestic: that means it covers the geographical area of a country irrespective of ownership or nationality
	\item Product: it is based on the amounts of goods newly produced, not just sold
\end{itemize}
Having understood that, there are further points to note about GDP.
\begin{itemize}
	\item GDP aims to include all output for sale in the market and also some non-market output
	\item Value means market value (where possible)
	\item Only final goods are counted which means we exclude intermediate goods
	\item GDP is a flow not a stock so it is measured in a period of time
\end{itemize}
There are three equivalent approaches to measure GDP:
\begin{enumerate}
	\item Production approach
	\item Expenditure approach
	\item Income approach
\end{enumerate}

\subsubsection{Production Approach}
This is when we sum all the \textit{value added} over all industries producing goods and services. Here are some key things to remember:
\begin{itemize}
	\item Value added equals value of output \textbf{minus} value of intermediate goods used in production. This is to avoid double counting.
	\item Use market value where possible which can be Price $\times$ Quantity
	\item For government services where there is no market value, use cost of production as a proxy
	\item Impute values for some non-market outputs as well such as services from owner-occupied housing
\end{itemize}

\subsubsection{Expenditure Approach}
The expenditure approach sums all the expenditures within the economy. This means household consumption, investment, government consumption and net exports.
\begin{itemize}
	\item Households consumption is equal to purchases of goods and services by households
	\item Investment is equal to purchases of new capital goods by firms plus purchases of new residential structures plus net change in inventories of goods
	\item Government consumption is equal to purchases of goods to provide public services including their own production
	\item Net exports is equal to the value of exports minus the value of imports
\end{itemize}

\subsubsection{Income Approach}
The income approach is the sum of all income derived from the production process. These include:
\begin{itemize}
	\item Wages
	\item Rents
	\item Net interest paid by firms
	\item Profits
	\item Indirect taxes
\end{itemize}
Incomes are paid by firms (or the government) and is related to the value of production. Indirect taxes deducted from profits are added back for consistency in relation to the value of production.

\subsection{Comparing GDP}
We usually compare GDP through price indices and between nominal and real GDP values. We also compare GDP across time and countries. Here is a run down of what we mean by Nominal and Real:
\begin{itemize}
	\item \textbf{Nominal} variables are expressed in units of money, with no chance, just as it is recorded
	\item \textbf{Real} variables are adjusted for changes in the value of money so they aim to capture changes in quantities only and remove the effects of inflation
\end{itemize}
When calculating nominal growth, we just compare the figures from year to year. For example:
\begin{center}
	Year 1 GDP in US\$: $10,000$\\
	Year 2 GDP in US\$: $22,000$\\
	$\frac{22,000-10,000}{10,000} = 1.2$\\
	That is a 120\% growth in nominal GDP between Year 1 and Year 2.
\end{center}

We use a base year to calculate the changes in real

\subsubsection{Comparing GDP Across Countries}
One soln: use market exchange rates to convert values in different currency into those of a common currency
Example:
- Country A uses \$
- Country B uses \£
- The market exchange is rate is \$1 = \£2

Is the comparison on market exchange rates a fair one?

Price of computers in country B expressed in terms of country A's currency - same as the price of computers in country A
Computers satisfy the law of one price: the price is the same everywhere once expressed in terms of a common currency (plausible for traded goods - computers)

Haircuts are probably not the same
Haircuts can be relatively cheap - they do not satisfy the law of one price, they cannot be traded across countries
Theory: Balassa-Samuelson effect

Cost of living is relatively low in Country B

\subsubsection{Purchasing Power Parity (PPP)}
Compare the cost of the same basket of goods across the two countries

Use the same quantities with the same prices e.g. Country A as the base.

Relative price of the two currencies is the ratio of the two prices.

Use of country A's basket is arbitrary.

Again, use Fisher Ideal's Index (geometric average) to get a better measure of both baskets


\section{Macroeconomic Models}
We use microeconomic principles to inform our macroeconomic models.

Equilibrium: in equilibrium the consumer and the firm face the same market real wage and the marginal rate of substitutions (from consumer's preference) is equal to the marginal rate of transformation of time into good (marginal product of labour from the production function)

Equilibrium is efficient: since $MRS_{l,c} = MP_n$ the competitive equilibrium is Pareto efficient. Not possible to make any improvements

\subsection{Taxation}
Assumed lump-sum tax
In practice, should be income tax
Implications of a proportional labour income tax at rate t
- Supply labour up to the point where MRS is equal to the after-tax wage
- Inefficiency: tax drives a wedge between MRSlc and MPn 

Laffer Curve
- With taxes that are not lump-sum, the amount of revenue does not necessarily increase when the tax rate rises
- Higher tax rate discourages working which could lead to lower tax revenue
- For very high tax rates, this negative effect on revenue is likely to be dominant
- For low tax rates the direct effect of higher tax rates increasing revenue is more important i.e. raise taxes to gain more revenue


\section{Economic Growth}
Vast difference in outputs per worker across the world e.g. United States Worker, in just over ten days, produced as much as an average worker in Niger has produced in one year.

Rich were only ahead by a factor of 2 instead of 35. Historically unprecedented. Growth in perpetuity?

\subsection{Measuring Growth}
Growth in output per head gives us an idea of standards of life

Could also be divided by workers.

Taking the natural logarithm of the raw data is a common way to plot data

The slope of the graph of the natural log of a time series is a good approximation to its growth rate

There are other important measures of development (welfare) such as:
\begin{itemize}
	\item Life expectancy
	\item Infant mortality
	\item Literacy and education
\end{itemize}
Measures such as the HDI can go in that direction - holistic measure. Though we will be narrow in this course and focus on GDP per person, there is more comparable data available at that time.

\subsubsection{Why growth matters}
Compounding rates on the level of growth
1.75\% per year in the US from 1870-2000. This grew to \$36,000
If g=0.75\% would lead to \$10,000
If g=2.75\% would lead to \$120,000
Is g=2.75\% unreasonable? No - but sustaining that growth could be hard

From stagnation to modern growth
Countries that have experienced growth in per capita output also experienced a long period of stagnation
Increase in population offset the increase in output
Will study Malthusian Model and the Solow Model

Convergence?


\newpage
\appendix

\section{National Income and Product Accounts}
The NIPA is produced by the Bureau of Economic Analysis (BEA), providing information on the value and composition of output produced in the United States during a given period and on the types and uses of the income generated by that production. NIPA begins by considering the transactions that occur in a simple economy in order to introduce the economic concepts that underlie the NIPAs. We also describe the NIPA sectors for which economic activity is measured and the use of T-accounts to illustrate economic flows.

\subsection{Conceptual Basis of the Accounts}
This section covers concepts used within the NIPA framework.
\subsubsection{The circular flow}
The fundamental idea of a working economy can be illustrated between just two parties: individuals and businesses. Between these two parties, we can illustrate the flow of income such that:
\begin{itemize}
	\item Individuals provide labour and in return businesses provide goods and services or,
	\item Individuals are given income from their labour and in return, they expend their income on the business.
\end{itemize}
This basic form of exchange is called the circular flow of income and is self-contained within this example.

\subsubsection{Economic concepts in the NIPAs}
The circular flow illustrates the interdependence of the flows or activities that occur in the economy, such as the production of goods and services and the income generated from that production. The circular flow also illustrate the equality between the income earned from production and the value of goods and services produced.

\vspace{10pt}
\noindent However, we did mention that the actual economy is much more complicated. What is not included is also the state, local governments and the rest of the world (trade). Other parts are also not discussed in this simply flow, such as investment in capital (PP\&E), flows of financial capital (stocks, bonds, deposits) and the contribution of these flows to the accumulation of fixed assets.

\subsubsection{Output}
The featured measure of output in the NIPAs is GDP. GDP measures the market value of goods, services and structure produced by the nation's economy in a particular period. The following points are things to keep in mind when considering the output of the economy.

\vspace{10pt}
\noindent\textbf{GDP includes market production and some nonmarket production}\\
This means that GDP accounts for things sold in the market, your normal goods and services any consumer could purchase, and of things that are not sold in the market such as defence services provided by the government, education provided by the local government or the state and so on. Data aims to include this to gain a true measure

\vspace{10pt}
\noindent\textbf{Whenever Possible, GDP is valued at Market Prices}\\
The NIPAs value market goods and services using prices set by the market. This means that there is a common unit of measurement that facilitates the comparison of goods and services that make up economic activity. Using market values also facilitates the analysis of the impacts on the economy of events such as the implementation of government programs or the occurrence of natural disasters e.g. boost in education spending or destruction of infrastructure valued at $x$.

In some cases, market prices do not fully reflect the \textbf{value} of a good or service and may include some types of services where an actual exchange has not occurred. In these cases, the value is \textit{imputed} from similar market transactions. Imputations measure the value of goods and services that are not fully reflected in market prices. Examples of imputed measure in the NIPAs include the value of compensation-in-kind (such as meals provided by employers).

\vspace{10pt}
\noindent\textbf{GDP is a measure of current production, not sales}\\
In the NIPAs, the measure of output refers to output produced within that period, regardless of when that output is sold. For example, if a car-maker produces a car in Period 1 but does not sell it, we record that as GDP. When it sells the car in Period 2, we subtract from inventory and also record it as an expense, these two cancel out so GDP is no affected.

\vspace{10pt}
\noindent\textbf{GDP is equal to the value of goods and services for \textit{final} users}\\
This means that measurement of GDP captures the value of products that are consumed and not used in a later stage of production, those that are sold, given away, or otherwise transferred to foreign residents, those that are used to produce other goods and that last mote than a year, and those that may be inventoried for future consumption. When considering the production process for the entire economy, the value of intermediate products - goods and services used in the production process (and will not contribute to future production) - is excluded, so that the measure of output is an un-duplicated total. 

\vspace{10pt}
\noindent\textbf{GDP can be measured in three different ways}\\
Refer to notes above on more detail but a quick reminder:
\begin{enumerate}
	\item Expenditures approach
	\item Income approach
	\item Value-added approach
\end{enumerate}

\vspace{10pt}
\noindent\textbf{GDP is a gross measure}\\
This means that we do not account for depreciation of consumption of fixed capital. Setting aside this amount from GDP would give us the Net Domestic Product which is a measure of current production excluding these other costs. Net domestic product is a measure that indiciates how much of the Nation's output is available for consumption or for adding to the Nation's wealth.

\subsection{The T-account}
A T-Account offers another way to illustrate the flows of the economy. More detailed than the circular flow diagram, it is a two-sided table that matches source of funds on the right (credit) with uses on the left (debit). The example below is a simple "Income and Outlay Account" for an individual.

\begin{center}
\begin{tabular}{|c|c|c|c|}

\multicolumn{2}{c}{Uses of income} & \multicolumn{2}{c}{Sources of income}\\
\hline
Consumption & 50 & Compensation & 70\\
Tax payments & 20 & Interest received & 20\\
Saving & 30 & Dividends received & 10\\
\hline
Total Expenditures and Savings & 100 & Total Income & 100\\
\hline

\end{tabular}
\end{center}
The right side of the account shows an individual's sources of income: compensation (wages and salaries) and the interest and dividends received from the ownership or assets (bonds and stocks). The sum of these sources is total income. The left side shows the individual's uses for this income: consumption (purchases of goods and services, tax payments and savings. The sum of these uses is total expenditures and savings.

\vspace{10pt}
\noindent The structure of the T-account provides two analytical benefits. First, because it is an identity, it enables one to identify and estimate a balancing item between the two sides of the account. For example, we can take away expenditure items with known values from total income to arrive at values of other expenditures. Second, when constructed for more than one economic sector, the T-accounts provide a double-entry system in which a source of income in an account for one sector also appears as a use of income in the account of another sector. 

\subsection{Seven NIPA accounts}

Here are the seven accounts listed:
\begin{enumerate}
	\item Domestic Income and Product Account
	\item Private Enterprise Income Account
	\item Personal Income and Outlay Account
	\item Government Receipts and Expenditures Account
	\item Foreign Transactions Current account
	\item Domestic capital Account
	\item Foreign Transactions Capital Account
\end{enumerate}

\end{document}