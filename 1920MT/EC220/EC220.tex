\documentclass[12pt, letterpaper]{article}
\usepackage[margin=1in]{geometry}
\usepackage{graphicx}
\usepackage{amsmath}
\usepackage{amssymb}
\usepackage{tikz}
\usepackage{hyperref}
\hypersetup{
	colorlinks = true,
	linkcolor = black,
	urlcolor = blue
}

\setlength{\headheight}{15pt}

\usepackage{fancyhdr}
\pagestyle{fancy}
\fancyhf{}
\rhead{\leftmark}
\lhead{EC220}

\title{
	{EC220 Introduction to Econometrics}\\
	{\large{Dr Canh T. Dang}}\\
	{\large{Lecture Notes}}
}
\author{Cedric Tan}
\date{September 2019}
\begin{document}
\maketitle
{\small
  \noindent\textbf{Concept}\\
  Concept. \hspace*{\fill}[1]

  \vspace{10pt}
  \noindent\textbf{Concept}\\
  Concept.\hspace*{\fill}[2]

\newpage
\tableofcontents
\newpage
\section{Introduction to Econometrics: Michaelmas Term}
The initial focus of MT is applied econometrics, particularly causal questions such as "what-if" questions. Mislabelling causality as correlation can be a critical error that people make when analysing data. The course intends to teach you how to analyse data and answer economic questions using "econometrics" and data.

\vspace{10pt}
\noindent Examples of what-ifs are below:
\begin{itemize}
	\item What happens to a country if it withdraws from a trade agreement?
	\item What is the effect of parents' education on children's education?
	\item What is the impact on your health if you go to a hospital?
\end{itemize}

\subsection{Causality}
\textbf{A causes B:} A contributes (or influences) to the occurrence of event B. The \textit{cause} A is partly responsible for the \textit{effect}
 B, and the effect B is partly dependent on the cause A. A can be necessary for the occurrence of B, but A can simply lead to fluctuations in B, this is still a causal relationship.
 
\vspace{10pt}
\noindent So we can take two definitions for causality:
\begin{itemize}
	\item A is a necessary condition for B to occur
	\item A can cause fluctuations in B
\end{itemize}
Labels are also necessary for the structure of causation:
\begin{itemize}
	\item Event A is called Treatment
	\item Event B is called the Outcome
	\item A third variable that causes the two events to happen is called a Confounder
\end{itemize}

\noindent We can have reverse causality - A causes B but also B causes A. An example would be \textit{Umbrellas} and \textit{Rain} where bringing umbrellas is caused by the possibility of rain. Sometimes timing helps to establish causality - as \textit{Rain} happens before \textit{Umbrellas}, \textit{Umbrellas} cannot cause \textit{Rain}.

\vspace{10pt}
\noindent Reasons that A and B are correlated:
\begin{enumerate}
	\item A causes B (direct causation)
	\item B causes A (reverse causation)
	\item A and B are consequences of a common cause but do not cause each other (confounder)
	\item A causes B and B causes A (bidirectional causation)
	\item A causes C which then causes B (indirect causation)
	\item No connection between A and B, the correlation is pure coincidence
\end{enumerate}
All statistical techniques only establish associations, causation requires interpretation.\\
Correlation: the extent to which A and B tend to decrease and increase at the same time.

\vspace{10pt}
\noindent Causation can occur without correlation, here is an example for medicine:\\
Illness (A) can cause death (B), but nowadays healthcare (C) can eliminate the correlation between common illness and death.

\subsection{Why Causality?}
This is the economist's comparative advantage, the ability to infer causality from correlation.

\vspace{10pt}
\noindent Examples of causality are listed below with classifications like the above.
\begin{itemize}
	\item Direct Causation:
	\item Reverse Causation:
	\item Confounder Problem:
	\item Bidirectional Causation:
	\item Indirect Causation
	\item Pure Coincidence:
\end{itemize}

\end{document}