\documentclass[12pt, letterpaper]{article}
\usepackage[margin=1in]{geometry}
\usepackage{graphicx}
\usepackage{amsmath}
\usepackage{amssymb}
\usepackage{tikz}
\usepackage{hyperref}
\hypersetup{
	colorlinks = true,
	linkcolor = black,
	urlcolor = blue
}

\setlength{\headheight}{15pt}

\usepackage{fancyhdr}
\pagestyle{fancy}
\fancyhf{}
\rhead{\leftmark}
\lhead{GV263}

\title{
	{GV263 Public Policy Analysis}\\
	{\large{Dr Daniel Berliner, Dr Charlotte Haberstroh, Professor Edward Page}}\\
	{\large{Lecture Notes}}
}
\author{Cedric Tan}
\date{September 2019}
\begin{document}
\maketitle
{\small
  \noindent\textbf{Concept}\\
  Concept. \hspace*{\fill}[1]

  \vspace{10pt}
  \noindent\textbf{Concept}\\
  Concept.\hspace*{\fill}[2]

\newpage
\tableofcontents
\newpage

\section{Introduction to Public Policy and the Policy Cycle}

\subsection{Structure}
Policy process and policy-making
\begin{itemize}
	\item Agenda-setting
	\item Policy formulation and decision-making styles
	\item Implementation
	\item Policy evaluation
	\item Public opinion
	\item Interest groups
	\item Political parties
	\item Bureaucracies
\end{itemize}
Contested issues in public policy
\begin{itemize}
	\item Science and Public Policy
	\item Policy fiascos
	\item Nudging
	\item Austerity
	\item ICTs and Public Policy
	\item Corruption
	\item Transparency
	\item Developing countries
	\item Cross-national learning
\end{itemize}

\subsection{Public Policy}
Some of the questions covered in the course:
\begin{itemize}
	\item Why are we talking about developing policy for an issue now when it has been ignored for years?
	\item Who shapes the policy responses to problems?
	\item Why do we appear to make only small changes to existing policies?
	\item Why do many policies seem to achieve little?
	\item Do laws work better than persuasion? (Nudge)
	\item When do policy makers listen to evidence? (Thunberg)
\end{itemize}
There is a troubled relationship between academic study of policy and its practice. Enthusiasm for academic research has shaped policy practice from the 18th century Cameralism through Fabianism to Blairite "evidence based policy". However, results of academic impact has been modest, even where invited.

\vspace{10pt}
\noindent The What Works initiative aims to improve the way government and other organisations create, share and use high quality evidence for decision-making. Academic papers aim to inform public policy but the way in which they are written makes public policy implementation difficult. The evaluation aspect of academic papers can confuse the application of public policy. Hence we could ask: \textbf{what has the academic study of policy got to offer then?}

\vspace{10pt}
\noindent Is the academic idea useless? Pursuing sets of questions that have cultural impact and value, acquiring and developing insights, acquiring and developing skills and techniques that have practical pay-off, these are all aspects of the analysis idea.

\vspace{10pt}
\noindent The cultural importance has a few questions:
\begin{itemize}
	\item Who sets the agenda in public policy?
	\item Does public opinion shape policy?
	\item Does bureaucracy constrain politics? (Brexit and lawmakers becoming Anti-Brexit constraining policy)
	\item Do parties make a difference? (Does it matter for a variety of public policy?)
\end{itemize}
There are practical payoffs: Lindblom and Cohen (2009): When one is making policy, you are attempting to solve a problem. You have two types of knowledge: 1. Ordinary Knowledge - the things we know about people e.g. people do not want to go to jail, people want to earn money i.e. things we do not need to research so heavily and 2. Professional Social Inquiry (PSI) - knowledge that requires more in depth analysis e.g. police presence and their impact on crime in a certain area.

\vspace{10pt}
\noindent From Lindblom and Cohen we can infer that a lot of Public Policy is created from Ordinary Knowledge and not Professional Social Inquiry. But we can see three methods of PSI impact public policy:
\begin{itemize}
	\item Results of PSI can shape policy alongside Ordinary Knowledge: methods of combating corruption, the beneficiaries of "open government"
	\item Knowledge produced from PSI can become Ordinary Knowledge: incrementalism, bounded rationality e.g. cancer research, new public management and thermostat theory - public opinion reacts to public policy like a thermostate (up, down, up again style fluctuation)
	\item PSI can debunk Ordinary Knowledge: (possibly controversial) could lead to policy learning, checks on corruption and the role of the private sector
\end{itemize}

\vspace{10pt}
\noindent There are also practical contributions:
\begin{itemize}
	\item Forensic skills: where problems/issues arise (e.g. analysis of stages of policy making)
	\item Techniques of assessing and evaluating policies (e.g. Weiss on Evaluation)
	\item Exploring alternative ways of assessing policies (e.g. debate on "success" of policy)
	\item Skepticism: bias in academic literature is that things do not work as expected
\end{itemize}
A lot of debate is centered around this.

\subsection{What is Public Policy?}
There is no one definition but there are several famous ones:
\begin{itemize}
	\item Anything that government chooses to do or not to do - Dye
	\item Set of interrelated decisions - Jenkins
\end{itemize}
There are many meanings of public policy:
\begin{itemize}
	\item It is a field of activity
	\item A form of Intention and desire
	\item A bundle of measures/Specific proposals/Decisions by government e.g. tobacco
	\item It is strategy e.g. blame avoidance
	\item Formal authorisation
	\item Output and Outcome
	\item Cause/Effect assumptions
\end{itemize}

\subsection{Policy process and policy cycle}
Analyst Perspective
\begin{itemize}
	\item Problem formulation
	\item Selection of criteria
	\item Comparison of alternatives and selection
	\item Consideration of constraints
	\item Implementation and evaluation
\end{itemize}
Policy Process Perspective
\begin{itemize}
	\item Agenda-setting
	\item Policy formulation
	\item Decision-making
	\item Policy implementation
	\item Policy evaluation
	\item Termination
\end{itemize}
Policy cycle and questions:
\begin{itemize}
	\item Agenda Setting: problem recognition and issue selection
		\begin{itemize}
			\item What issues receive political attention?
			\item Why are issues framed in particular ways?
			\item What processes lead to mobilisation?
		\end{itemize}
	\item Policy formulation and decision making
		\begin{itemize}
			\item Who was involved in making the decision?
			\item What did they want
			\item How powerful were they?
			\item What resources and support could they mobilise?
			\item How was conflict handled and resolved?
		\end{itemize}
	\item Implementation
		\begin{itemize}
			\item Why did things not work out as intended?
			\item Were there unforeseen hitches and obstacles?
			\item Were street-level bureaucrats selective in how they implemented instructions? (Policy and cattling)
		\end{itemize}
	\item Evaluation
		\begin{itemize}
			\item How clear were the objectives behind the policy?
			\item How do you measure the impact of the policy?
			\item Was there a hidden set of motives behind the policy?
		\end{itemize}
\end{itemize}
Yet is the policy cycle a useful device? Policy cannot be differentiated into discrete stages, sequencing is empirically misleading.
\begin{itemize}
	\item Implementation is policy formulation on the ground.
	\item Interaction among multiple cycles and policies: there are trade-offs and conflicts between goals (Garbage Can Theory - haphazard development of policy)
	\item No theory or cause
	\item Implicit rational and top-down assumption
	\item Operates at what level?
\end{itemize}
Simply put it is:
\begin{itemize}
	\item A useful friend
	\item Offers way to cut complexity and to ask distinct questions
	\item Has theoretical openness
	\item Has process-tracing (from start to finish)
\end{itemize}


\end{document}