\documentclass[12pt, letterpaper]{article}
\usepackage[margin=1in]{geometry}
\usepackage{graphicx}
\usepackage{amsmath}
\usepackage{amssymb}
\usepackage{tikz}
\usepackage{hyperref}
\hypersetup{
	colorlinks = true,
	linkcolor = black,
	urlcolor = blue
}

\setlength{\headheight}{15pt}

\usepackage{fancyhdr}
\pagestyle{fancy}
\fancyhf{}
\rhead{\leftmark}
\lhead{GV249}

\title{
	{GV249 Research Design in Political Science}\\
	{\large{Dr Florian Foos}}\\
	{\large{Lecture Notes}}
}
\author{Cedric Tan}
\date{September 2019}
\begin{document}
\maketitle
{\small
  \noindent\textbf{Concept}\\
  Concept. \hspace*{\fill}[1]

  \vspace{10pt}
  \noindent\textbf{Concept}\\
  Concept.\hspace*{\fill}[2]

\newpage
\tableofcontents
\newpage

\section{Module}
Check moodle and fill in

\section{Introduction to Research Design}
Aims of this module:
\begin{itemize}
	\item Make you competent consumers of research: how do you tell whether you can trust a study's results or what does this technical language within the scientific paper actually mean? There is also an aim to better understand claims to causality, providing a methodology to evaluate such claims.
	\item Enable you to design your own studies and conduct your own research by equipping students with the tools used within political science. 
\end{itemize}

What does it mean to be fair?
\begin{itemize}
	\item The deck should not be stacked in favour of your preferred hypothesis!
	\item Choose a method that is \textit{unbiased}.
	\item If you find the opposite to what you expect to find, you need to say so and you need to report it.

\end{itemize}
\subsection{Causes of effects}
These questions are never answerable beyond a reasonable doubt because the event has already happened and the number of potential causes can be infinite.

Take the example of Hilary Clinton and the 2016 Presidential Election:
\begin{itemize}
	\item What is the effect of misogynistic remarks on voters' evaluation of political candidates?
	\item What is the effect of anti-trade rhetoric on party support?
\end{itemize}
Manipulation is also a key factor in cause and effect.
\begin{center}
	\textbf{Manipulation is the assignment to different values of the treatment/explanatory variable.}
\end{center}

\subsubsection{Terminology}
Variables
Categorical
Ordinal
Continuous

Outcome and Explanatory Variables:
\begin{itemize}
	\item Outcome: dependent variable (DV) and denoted by $Y$.
	\item Explanatory: independent variable (IV) or "treatment" and denoted by $X$ or $Z$.
\end{itemize}

\subsubsection{Description}
Sometimes political science is concerned with describing rather than finding cause and effect. Examples include:
\begin{itemize}
	\item Did the share of marriages 
	\item From 1990 to now, how 
\end{itemize}
Try to look for questions that ask 'what is'.

\subsubsection{Research Integrity}
Research as the search for the truth.
Challenges related to bad science:
\begin{itemize}
	\item Underreporting: researchers do not report all the tests that they conducted
	\item P-hacking: researchers choose the model specification based on which model gives the best results
	\item Biased research outputs that are unethical
\end{itemize}
Here are some steps to maintain integrity.
\begin{itemize}
	\item Pre-register our hypotheses
	\item Report null findings
	\item Replicate each others studies
	\item Correct mistakes
	\item Learn together
\end{itemize}

\subsection{What makes Political Science}
This section aims to cover an introduction to the scientific process found within the Political Science discipline. We refer to Kellstedt and Whitten (20xx) with their approach to politics being the search for causal explanations. There is a need to have a willingness to consider new evidence, and on the basis of that new evidence, change what you thought you \textit{knew} to be true. There is a road to scientific knowledge that political scientists may embark on. The stages are as follows:
\begin{enumerate}
	\item Causal Theory
	\item Hypothesis
	\item Empirical Tests
	\item Evaluation of Hypothesis
	\item Evaluation of Causal Theory
	\item Scientific Knowledge
\end{enumerate}
Listed below is a short explanation on each one:

\vspace{10pt}
\noindent \textbf{Causal Theory}: this is the tentative conjecture about the causes of some phenomenon of interest. The development of causal theories about the political world requires thinking in new ways about familiar phenomena. As such, theory building is part art and part science.

\vspace{10pt}
\noindent \textbf{Hypothesis}: once a theory has been developed, we want to test it. The first step in testing is to restate our theory as one or more testable hypotheses. A hypothesis is a theory-based statement about a relationship that we expect to observe. For every hypothesis, there is a corresponding \textbf{null hypothesis} which is the theory-based statement but it is about what we would observe if there were no relationship between the independent and dependent variables we have identified.

\vspace{10pt}
\noindent \textbf{Empirical Tests}: here we look towards the real world for observable evidence that might prove or disprove our hypotheses. We evaluate our hypotheses with reference to the null hypotheses.

\vspace{10pt}
\noindent \textbf{Evaluation}: from our tests, we look to evaluate our hypotheses and our causal theories to understand whether or not a plausible link can be found between the two. There has to be a significant link to disprove the null hypothesis. If there was a 50-50 evaluation, the null hypothesis usually takes precedent. If there was a 80-20 evaluation, the null hypothesis still takes precedent! This is important to understand because accepting new found scientific knowledge is critical to creating further knowledge. Thus, to disprove the null hypothesis, the experiment's results must be tried and tested.

\vspace{10pt}
\noindent Once a theory has been established as part of scientific knowledge in a field of study, researched can build upon the foundation that this theory provides. This accumulation in knowledge and acceptance of theories forms a \textbf{paradigm.} Once researches in a scientific field have widely accepted a paradigm, they can pursue increasingly technical questions that make sense only because of the work that has come beforehand. The state of research under an accepted paradigm is referred to as \textbf{normal science}.

\vspace{10pt}
\noindent When a major problem is found with the accepted theories and assumptions of a scientific field, that field will go through a revolutionary period during which new theories and assumptions replace the old paradigm to establish a new paradigm. 

\subsubsection{Variables and Causal Explanations}
Political scientists like to think of the world in variables and causal explanations. We should think of each variable in terms of its \textit{label} and its \textit{values}. The \textbf{variable label} is a description of what the label is, and the \textbf{variable values} are the denominations in which the variable occurs. An example is age, the label being age, and the denomination being the number of years, days or even hours.

\vspace{10pt}
\noindent Variables are also labelled as independent or dependent, depending on their role within our causal theory. Independent variables are the variables which change and cause the dependent variable to shift in one way or another. A good way of remembering this is that \textit{dependent variables are \textbf{dependent} on the independent variable}. Variables are important because they help us think about causal explanations behind events. For example, how Variable A \textbf{causes} a change in Variable B, here this is a statement of causality.

\vspace{10pt}
\noindent From our Causal Theory, we move from this general statement to a more specific statement on how we view the world, and more specifically, how we think the world works in a certain way. We want to measure our variables so we begin to \textbf{operationalize} them. In this context, we can use the words measure and operationalize interchangeably. Here, we look at the variables and use data to inform ourselves on how the relationships work. 


\subsection{Big Data, Formal Theory and Causal Inference}

\end{document}